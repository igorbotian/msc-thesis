\section*{ABSTRACT}

\begin{doublespace}

\thesisschool

%\vspace*{1em}

\thesisauthor

\textbf{\thesistitle}

\thesissubject

\currentyear

%\pageref{LastPage}
\todo{??} pages, \todo{??} figures, \todo{??} tables and \todo{??}
appendices

\begin{tabbing}
Examiners:\quad\= \thesisfirstexaminer\\
\> \thesissecondexaminer
\end{tabbing}

Keywords: \thesiskeywords
\end{doublespace}

\vspace*{1em}
\todo{Abstract}
%
\begin{comment}
FINAL THESIS INSTRUCTIONS.

The abstract is a concise (one A4 sheet), objective, independent summary
of the Master’s thesis. It should be intelligible as such, without the original document. 

It explains the contents of the thesis: the objective, methodologies, results and conclusions. 
A good abstract is written in complete and concise sentences. 
The author does not express his or her opinions, but describes the thesis as would an outside reporter. 
No direct references are made to the original text. 

The abstract is a public document, and therefore all confidential information must be excluded
from it. The abstract is prepared in Finnish and English. 
Both the Finnish and English abstracts are included in the thesis. 

The abstracts are also submitted to the faculty study affairs services as an annex to the assessment 
application of the thesis.
Foreign nationals do not need to prepare an abstract in Finnish.

The author sends electronic copies of the abstracts or the entire thesis to the LUT library. 
More details are available from the library and its web site.

=== 

An abstract is prepared on all Master's theses. You should favour the passive voice or the 3rd
person active in case the abstract is published separately. Unestablished abbreviations, symbols or
technical terms should be explained. Tables, equations etc. are used only if they are necessary for
the sake of clarity. No direct references are made to the original text.

The abstract is done in both Finnish and English (equivalent contents). In the Finnish abstract, the
title is in Finnish and in the English one in English. Foreign students do not need to prepare an
abstract in Finnish.

The complete identification information should be included in the beginning of both the Finnish and
the English abstract.

Author’s name
Title of thesis
Faculty
Degree programme and/or major subject
Year of completion
Master’s Thesis University
Number of pages, figures, tables and appendices
Examiners (1st and 2nd)
Keywords in Finnish
Keywords in English

The keywords must be informative and describe the contents of the thesis accurately. 
Concrete concepts (e.g. equipment) are in plural, abstract ones (e.g. methods) in singular. 
A good title should include at least some of the most important keywords.
The number of keywordsshould be three to five.
\end{comment}