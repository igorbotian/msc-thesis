\SectionTitle{CONCLUSIONS AND FUTURE WORK}
\todo{Contribution, future work}
\Paragraph{}
\todo{Conclusions}
%
\begin{comment}
FINAL THESIS INSTRUCTIONS.

Depending on the nature and scope of the study, the report ends either with the chapter
"Conclusions", or two separate chapters, e.g. "Conclusions" and "Summary". The conclusions
analyse the observations and results drawn from the research. The conclusions examine and reflect
on e.g. the compatibility of the theory and measurements, the reasons for possible differences, and
summarise the conclusions drawn from the results. The need for further research and possible
practical applications may also be argued here.

INSTRUCTIONS FOR WRITING A MASTER'S THESIS

In most cases it is reasonable to present new findings and main results in a separate
section. This section also further analyzes observations and results. This section is also the
place to include the synthesis of the results, the future possibilities in the research topic or
the in the application area. After discussion and consideration one should present the
conclusions. This section contains the reasoning for those conclusions. In the next section,
in the summary, one just lists the conclusions.

HOW TO ORGANIZE YOUR THESIS

You generally cover three things in the Conclusions section, and each of these usually merits 
a separate subsection:

1. Conclusions 
2. Summary of Contributions 
3. Future Research

Conclusions are not a rambling summary of the thesis: they are short, concise statements 
of the inferences that you have made because of your work. It helps to organize these as short 
numbered paragraphs, ordered from most to least important. 
All conclusions should be directly related to the research question stated in Section 4.

Examples:
1. The problem stated in Section 4 has been solved: as shown in Sections ? to ??, 
an algorithm capable of handling large-scale Zylon problems in reasonable time has been developed.
2. The principal mechanism needed in the improved Zylon algorithm is the Grooty mechanism.
3. Etc.

The Summary of Contributions will be much sought and carefully read by the examiners. 
Here you list the contributions of new knowledge that your thesis makes. Of course, 
the thesis itself must substantiate any claims made here. There is often some overlap with the 
Conclusions, but that's okay. Concise numbered paragraphs are again best. 
Organize from most to least important. Examples:

1. Developed a much quicker algorithm for large-scale Zylon problems.
2. Demonstrated the first use of the Grooty mechanism for Zylon calculations.
3. Etc.

The Future Research subsection is included so that researchers picking up this work in future 
have the benefit of the ideas that you generated while you were working on the project. 
Again, concise numbered paragraphs are usually best.

WRITING THE THESIS

Results

This is a narrative presentation of your findings. This is where you present your statistics, 
tables, figures, etc. that show what the specific findings of your study are. Present them in 
detail. Remember that someone should be able to duplicate your study based solely on this 
document. This requires considerable description. 
 
It is very important not to try and combine this chapter with the next one.  You need to
carefully present your results first with no further interpretation.  Once you have presented the 
data you are ready to move on to the next section.

Discussion

This chapter should begin with a concise restatement of your study’s purpose along with any 
needed background information.  You should restate each of your hypotheses.  Now that you 
have presented the results in the previous section, discuss them in this section.  What, 
specifically, do the results mean?  How can they be interpreted?  Can they be interpreted in 
multiple ways?  What do the findings tell you about your hypothesis?  Do not claim more for 
your results than the data really shows.  Avoid speculation. 

Conclusions

This should summarize your results and discussion.  You should include a list of the most 
important findings of your study in descending order of importance.  You should also provide a 
statement about the possibility of future study.  What needs to be done and what does this 
study contribute? 

HOW TO WRITE A THESIS

The General Discussion or Conclusions integrate the whole thesis and present its main points at one
place. This should be done in the context of the unifying hypothesis of the thesis.
The Introduction and this chapter along with the Summary or Abstract are the most important parts of
the thesis.

The Conclusions record the power of your scientific thinking. You have to unite all
that has gone before with a “thread of unified perspective”. This is where you say why you
think your story is a good one and present evidence from your work to support your claim.
The fate of your hypothesis is revealed here: did it stand, fall, or require modification?
You may briefly compare your work with that of others, present whatever new knowledge
has been gained from your work, and suggest what may be done to further new knowledge.
The Conclusions should give a sense of fulfilment and finality to your thesis, and give the
reader some satisfaction that the time spent on reading it has not been in vain.
\end{comment}