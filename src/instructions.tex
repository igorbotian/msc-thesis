\preamblesectiontitle{Instructions}

\preamblesubsectiontitle{Критерии оценки диплома}

\begin{suggestions}
Можно расписать, что в каждом пункте сделано и на каком уровне
\end{suggestions}

\begin{itemize}
  \item Постановка проблемы, цели, определения и установки границ диссертации
  \item Связь с предыдущими исследованиями
  \item Подход к исследованию, выбранные методы исследования и используемые материалы
  \item Расписание проведения исследование
  \item Полученные результы и их анализ
  \item Организация и логичность работы
  \item Глубина исследования
  \item Достоверность работы
  \item Язык и компоновка работы
  \item Независимость выбранного методы исследования и его применение
\end{itemize}

\preamblesubsectiontitle{What is a master's thesis}
Your  thesis is a research project that you have spent considerable time in preparatory research
(literature review), project design (formulation of a hypothesis), data collection (field and or 
laboratory), analysis (statistical examination of the data), and finally presentation and synthesis
(examination of the statistical results in the context of your hypothesis and literature review). 

\preamblesubsectiontitle{Thesis should be well-written}
\begin{enumerate}
	\item Organized, with a logical flow 
	\item Concise, but also complete
	\item Good grammar
	\item It’s usually a good idea to have a colleague read it before giving it to the advisor,
	especially if they have already submitted their first draft or successfully defended their
	proposal. Often little errors or small changes will be identified and addressed. 
	They can also be some the best sources of information for “why” or “how”. 
\end{enumerate}

\preamblesubsectiontitle{General structure}
\begin{enumerate}
	\item Introduction (Background, Motivations, Literature review)
	\item Objective/Purposes/Hypothesis (need not be a separate section, but often is)
	\item Methods
	\item Preliminary Results
\end{enumerate}

\preamblesubsectiontitle{General tips}
\begin{itemize}
  \item Each paragraph proceeds from general to specific.
  \item Some have suggested that reading the first sentence of every paragraph in the document
  should convey the essential meaning of the whole.
  \item Vary the structure of your sentences and paragraphs.
  \item Use transitions between paragraphs (either the last sentence of the proceeding one
  or the first sentence of the subsequent one, should tie the two together).
  \item Avoid one-sentence paragraphs (generally at least 3 sentences comprise a paragraph)
  \item Consider optional presentation methods (always using good HF knowledge and
  practice). Often the same thing can be conveyed by text, graphs, tables, diagrams,
  etc. Pick what is the most effective, but avoid duplication.
  \item Get in the habit of writing (and reading, in a special way, as noted earlier). As a student,
  it helped me to write something every day, even if it was brief, and even if I didn’t later use it. 
  It also helped (and still does) to write down my thoughts.
\end{itemize}

\preamblesubsectiontitle{Objectives/Purposes of the thesis}
\begin{itemize}
  \item Non-quantitative, but specific and clearly filling some hole/need addressed in the
  Introduction.
  \item The Intro should have motivated and “scoped” the stated objectives and purposes.
\end{itemize}

\preamblesubsectiontitle{Hypotheses of the thesis}
\begin{itemize}
  \item Non-quantitative, but again specific and clear.
  \item There should be obvious connections to the objectives; addressing (proving) your hypotheses
  supports achieving your objectives
  \item There must be clear (though not stated here) indications of how statistical methods would be
  used to evaluate the hypotheses. In the methods, your statistical tests should make reference to 
  these hypotheses.
  \item Not every statistical test should have an associated hypothesis (otherwise it would be
  unwieldy); instead, the hypotheses can be general (e.g. there will be an association among several 
  variables; factors A and B will have effects on several measures of performance).
  \item Don’t use words like ‘significant’, save this for the description of statistical methods.
\end{itemize}

\preamblesubsectiontitle{Methods of the thesis}
\begin{itemize}
  \item What will be done, how, and why? In particularly “why” (why this IV, why these levels, why
  this measure, ...)
  \item With respect to how and why, there is typically more than one way to do something, and you
  must explain (and sometimes justify) your choice.
  \item The methods should have clear connections to the hypotheses.
  \item The Methods tends to be a difficult and sometimes complicated section. In general, proceed
  from broad to specific, but also ensure that a context is provided before specific details are
  raised. For example, don’t describe specific experimental treatments before you’ve even explained
  the overall approach and the different independent and dependent variables.
  \item For widely-used and generally accepted approaches, just summarize with reference to the
  literature. For other approaches, more explanation and justification needed.
  \item Note that ‘repeated measures’ refers to a study design, while within- and between- subjects
  refers to specific independent measures (or treatments). Nested and between-subjects factors are 
  synonymous.
  \item The reader should be able to understand what you’re talking about, given what was provided
  before (use of a colleague again helps here).
  \item Subsections are often used such as: Overview; Participants; Procedures; Instrumentation;
  Experimental Design; Data Reduction; Analysis (stats)
  \item The specific ordering of the sections in g., should achieve the goals of d. and f.
  \item Somewhere (typically in Experimental Design), there should be an explicit statement of the
  independent and dependent variables (or factors, or measures)
  \item 
\end{itemize}

\preamblesubsectiontitle{Some common mistakes}
\begin{itemize}
  \item Repetitive sentence structure (The... The... The... or However, ... Additionally, ...
  Therefore, ...)
  \item Avoid complex words and convoluted sentence constructions, where simpler ones will convey
  the information (like utilize vs. use; cognizant vs. aware; though personal style always has a role). 
  Eschew obfuscation!
  \item There is no advantage to be gained by making something obscure. The scientific value is not
  enhanced by complicated words and prose, and to someone that knows the field, you don’t sound any
  more knowledgeable. If you look at some of the best journals, they are typically written in a very 
  dry, boring, direct, and terse style. It tends to be the weaker journals where creative writing 
  flourishes! 
\end{itemize}

\preamblesubsectiontitle{Tips}
\italic{Always keep the reader's backgrounds in mind.} Who is your audience? How much can you
reasonably expect them to know about the subject before picking up your thesis? 
Usually they are pretty knowledgeable about the general problem, 
but they haven't been intimately involved with the details over the last couple of years like 
you have: spell difficult new concepts out clearly. 
It sometimes helps to mentally picture a real person that you know who has the appropriate 
background, and to imagine that you are explaining your ideas directly to that person.
\nextparagraph{}
\italic{Don't make the readers work too hard}! This is fundamentally important. You know what few
questions the examiners need answers for (see above). 
Choose section titles and wordings to clearly give them this information. 
The harder they have to work to ferret out your problem, your defence of the problem, 
your answer to the problem, your conclusions and contributions, the worse mood they will be in, 
and the more likely that your thesis will need major revisions.
\nextparagraph{}
A corollary of the above: \italic{it's impossible to be too clear}! Spell things out carefully,
highlight important parts by appropriate titles etc. 
There's a huge amount of information in a thesis: make sure you direct the readers to the 
answers to the important questions.
\nextparagraph{}
Remember that \italic{a thesis is not a story}: it usually doesn't follow the chronology of things
that you tried. It's a formal document designed to answer only a few major questions.
\nextparagraph{}
Avoid using phrases like "Clearly, this is the case..." or "Obviously, it follows that ..."; 
these imply that, if the readers don't understand, then they must be stupid. 
They might not have understood because you explained it poorly.
\nextparagraph{}
Avoid \italic{red flags}, claims (like "software is the most important part of a computer system")
that are really only your personal opinion and not substantiated by the literature or the 
solution you have presented. Examiners like to pick on sentences like that and ask questions like, 
"Can you demonstrate that software is the most important part of a computer system?"

\preamblesubsectiontitle{Rationale for structure}
Introduction/Aim. \italic{What did you do and why?}\\
Materials and Methods. \italic{How did you do it?}\\
Observations/Results. \italic{What did you find?}\\
Discussion. \italic{What do your results mean to you and why?}\\
Conclusions. \italic{What new knowledge have you extracted from your experiment?}

\preamblesubsectiontitle{What Graduate Research is All About}
Your thesis must show two important things:

\begin{itemize}
	\item you have identified a worthwhile problem or question which has not been previously 
	answered,
	\item you have solved the problem or answered the question.
\end{itemize}

\preamblesubsectiontitle{Hypothesis}
The hypothesis is all important. It is the foundation of your thesis. It gives coherence
and purpose to your thesis. Your hypothesis must fit the known facts and be testable.
To comply with the first, you must have read the literature. To comply with the second, you must do
the experiment. This is why the hypothesis is central to scientific investigation.

\begin{itemize}
	\item The hypothesis defines the aim or objective of an experiment, that if some likely but
unproven proposition were indeed true, we would expect to make certain observations or measurements.
	\item A hypothesis is an imaginative preconception of what might be true in the form of
a declaration with verifiable deductive consequences.
\end{itemize}

\preamblesubsectiontitle{Examiner's questions}
Examiners ask the following questions when reading a thesis:

\begin{itemize}
  \item Has the student read all the references?
  \item What questions does this thesis raise?
  \item What richness does it contain that can spawn other work?
  \item What is the quality of flow of ideas?
\end{itemize}