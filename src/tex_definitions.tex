% === DEFINITIONS FOR TEXT ITEMS ===

% to use citations to bibliography items
% requires the 'color' package

\newcommand{\reference} [1] {
	\cite{#1}
}

\newcommand{\webresource} [1] {
	\textcolor{red}{\cite{#1}}
}

% to define bibliography items

\makeatletter
\def\url@leostyle{%
	\@ifundefined{selectfont}{\def\UrlFont{\sf}}{\def\UrlFont{\small\sf}}}
 	 %\@ifundefined{selectfont}{\def\UrlFont{\sf}}{\def\UrlFont{\small\ttfamily}}}
\makeatother

% Now actually use the newly defined style.
\urlstyle{leo}

% to define a bibliography item used in the thesis

\newcommand{\bibbook} [9] {
	\bibitem{#1} % key
		\ifthenelse{\isempty{#2}}
			{} % author(s) is not presented
			{#2,} % author(s)
		\emph{#3}, % title
		\ifthenelse{\isempty{#4}}
			{} % first edition
			{#4 edition,} % edition
		#5, % publisher
		#6, % country
		#7, % date of publication		
		#8 pages,
		ISBN: \mbox{#9}
}

\newcommand{\bibarticle} [9] {
	\bibitem{#1} % key
		#2, % author(s)
		\emph{#3}, % title
		#4, % journal
		\ifthenelse{\isempty{#5}}
			{}
			{Vol. #5,} % volume
		\ifthenelse{\isempty{#6}}
			{}
			{No. #5,} % number
		#7, % country
		\ifthenelse{\isempty{#9}}
			{#8} % date
			{#8, pages #9} % date and pages
}

\newcommand{\bibdissertation} [7] {
	\bibitem{#1} % key
		#2, % author(s)
		\emph{#3}. % title
		Dissertation, #4, % school
		#5, % address
		#6 % date
		%#7 pages,
}

\newcommand{\bibmastersthesis} [7] {
	\bibitem{#1} % key
		#2, % author(s)
		\emph{#3}. % title
		Master's thesis, #4, % school
		#5, % address
		#6 % date
		%#7 pages,
}

\newcommand{\bibproceedings} [6] {
	\bibitem{#1} % key
		#2, % author(s)
		\emph{#3}, % title
		#4, % conference
		#5, % address
		#6 % date
}

\newcommand{\bibmanual} [5] {
	\bibitem{#1} % key
		#2, % author(s)
		\emph{#3}, % title
		#4. % date
		Available at: \url{#5}, % URL
		retrieved \bibdateformat\today
}

\newcommand{\bibtechreport} [6] {
	\bibitem{#1} % key
		#2, % author(s)
		\emph{#3}, % title
		#4, % institution/organization
		#5, % country
		#6 % date
}

\newcommand{\bibwebarticle} [5] {
	\bibitem{#1} % key
		#2, % author(s)
		\emph{#3}, % title
		#4. % year
		Online document, from: \url{#5}, % URL
		retrieved \bibdateformat\today
}

\newcommand{\bibwebdocument} [4] {
	\bibitem{#1} % key
		\emph{#2},  % title
		#3. % year
		Online document, from: \url{#4}, % URL
		retrieved \bibdateformat\today
}

\newcommand{\bibwebsite} [3] {
	\bibitem{#1} % key
		\emph{#2}.  % title
		Internet page, from: \url{#3}, % URL
		referred \bibdateformat\today
}

% to define and add references

\newcommand{\placereferencetosection} [1] {
	\label{sec:#1}
}

\newcommand{\placereferencetofigure} [1] {
	\label{fig:#1}
}

\newcommand{\placereferencetotable} [1] {
	\label{tab:#1}
}

\newcommand{\referencetotable} [1] {
	\tablename~\ref{tab:#1}
}

\newcommand{\referencetofigure} [1] {
	\figurename~\ref{fig:#1}
}

\newcommand{\referencetosection} [1] {
	Section~\ref{sec:#1}
}

% to place an image
% requires the 'graphicx' package

\newcommand{\tablefigure} [3] {
	\begin{table}[placement=h]
		\begin{center}
			#3
			\caption{#1}
			\placereferencetotable{#2}
		\end{center}
	\end{table}
}

\newcommand{\imagefigure} [2] {
	\begin{figure}[placement=h]
  		\begin{center}
    		\includegraphics[scale=0.65]{#2}
    		\caption{#1}
    		\placereferencetofigure{#2}
  		\end{center}
	\end{figure}
}

% to print the given date (format is '<month short name> <year>')
% requires the 'datetime' package

\newcommand{\bibdate} [2] {%
	\monthname[#1] #2
}

% to print the given date (format is 'dd.mm.yyyy')
% requires the 'datetime' package

\newdateformat{bibdateformat}{%
	\twodigit{\THEDAY}.\twodigit{\THEMONTH}.\THEYEAR
	%\twodigit{\THEDAY}~\monthname~\THEYEAR
}

% to print the date of the thesis submission
% requires the 'datetime' package

\newdateformat{englishdateformat}{%
	\monthname[\THEMONTH] \ordinal{DAY}, \THEYEAR
}

\newcommand{\thesisdate}{
	\englishdateformat\today
}

% to print the current year
% requires the 'datetime' package

\newdateformat{yeardateformat}{%
	\THEYEAR
}

\newcommand{\currentyear}{
	\yeardateformat\today
}

% to use abbreviations in the text
% requires the 'acronym' package

\newcommand{\abbrv} [1] {
	\ac{#1}
}

% === DEFINITONS FOR APPENDICES ===

% the counter for appendices
\newcounter{appendixcounter}
\setcounter{appendixcounter}{0}
\newcommand{\appendixnumber}{\todo{\Alph{appendixcounter}}} % \arabic{appendixcounter}

% to format headers of the document pages
% requires the 'fancyhdr' package

\newcommand{\theappendixtitle}{}

\newcommand{\clearpagestyle} {
	\pagestyle{fancy}
	\fancyhf{} % to reset existing page format
	\renewcommand{\headrulewidth}{0pt} % to remove a page header line
}

\newcommand{\applygeneralpagestyle} {
	\clearpagestyle
	\cfoot{\thepage} % a page number center-aligned in the page footer
}

% to format a header of an appendix page
% requires the 'fancyhdr' package

\newcommand{\applyonepageappendixpagestyle} {
	\clearpagestyle	
	\rhead{\theappendixtitle}
}

% to format a header of an appendix page
% requires the 'fancyhdr' package

\newcommand{\applyappendixfirstpagestyle} {
	\clearpagestyle
	\rhead{\theappendixtitle}
	\rfoot{(continue)}
}

% to format a header of an appendix page
% requires the 'fancyhdr' package

\newcommand{\applyappendixpagestyle} {
	\clearpagestyle
	\rhead{APPENDIX \appendixnumber. (continued)}
	\rfoot{(continue)}
}

% to format a header of an appendix page
% requires the 'fancyhdr' package

\newcommand{\applyappendixlastpagestyle} {
	\clearpagestyle
	\rhead{APPENDIX \appendixnumber. (continued)}
}

% to format an appendix page and to add it's title to the Table of Contents
% doesn't require any additional package

\newcommand{\theappendixpagetitle} [1] {
	APPENDIX \appendixnumber. #1
}

\newcommand{\theappendixtoctitle} [1] {
	APPENDIX \appendixnumber. #1\par
}

\newcommand{\theappendix} [1] {
	\ifthenelse { \equal{\arabic{appendixcounter}}{0} } {
		\cleardoublepage
		\phantomsection
		\addcontentsline{toc}{section}{APPENDICES}
	}
	{
		% nothing
	}

	\addtocounter{appendixcounter}{1}
	\addtocontents{toc}{\theappendixtoctitle{#1}}
	\section*{\theappendixpagetitle{#1}}
	
	\renewcommand{\theappendixtitle}{\theappendixtoctitle{#1}}
}
