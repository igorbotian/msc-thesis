% === DEFINITIONS FOR TEXT ITEMS ===

% to define own format of a section's title

\newcommand{\SectionTitle} [1] {%
	\section{\MakeUppercase{#1}}%
	\vspace*{1em}%
}

\newcommand{\SubSectionTitle} [1] {%
	\subsection{#1}%
	\vspace*{0.5em}%
}

\newcommand{\SubSubSectionTitle} [1] {%
	\subsubsection{#1}%
	\vspace*{0.5em}%
}

\newcommand{\PreambleSectionTitle} [1] {%
	\section*{\MakeUppercase{#1}}%
	\vspace*{1em}%
}

\newcommand{\PreambleSubSectionTitle} [1] {%
	\subsection*{#1}%
	\vspace*{1em}%
}

% to define own delimiters between paragraphs and delimiters

\newcommand{\Paragraph} [1] {
	\ifthenelse{\isempty{#1}}
		{\par\vspace*{1em}}
		%{\par\vspace*{1em}}
		{\par\colorbox{paragraphdescriptionbgcolor}{#1}\par} % {\par\vspace*{1em}}
}

\newcommand{\Sentence} {}

% to define new text colors
% requires the 'color' package

\definecolor{commentcolor}{gray}{0.2}
\definecolor{paragraphdescriptionbgcolor}{RGB}{255,248,220} % pale yellow (cornsilk)
\definecolor{remarkcolor}{gray}{0.2}
\definecolor{suggestioncolor}{RGB}{235,140,0} % dark orange

% to use citations to bibliography items
% requires the 'color' package

\newcommand{\Reference} [1] {%
	\cite{#1}%
}

\newcommand{\WebSite} [1] {%
	\textcolor{blue}{\cite{#1}}%
}

% to define bibliography items

\makeatletter
\def\url@leostyle{%
	\@ifundefined{selectfont}{\def\UrlFont{\sf}}{\def\UrlFont{\small\sf}}}
 	 %\@ifundefined{selectfont}{\def\UrlFont{\sf}}{\def\UrlFont{\small\ttfamily}}}
\makeatother

% Now actually use the newly defined style.
\urlstyle{leo}

% to define a bibliography item used in the thesis
% requires the 'xstring' package

\newcommand{\BibAnd} {%
	~\AndWord~%\space and \space
}

\newcommand{\GetFirstLetter} [1] {%
	\StrLeft{#1}{1}%
}

\newcommand{\BibAuthor} [3] {%
	\GetFirstLetter{#2}. % first name
	\ifthenelse{\isempty{#3}}
		{#1}% only last name
		{\GetFirstLetter{#3}. #1}% middle name and last name
}

\newcommand{\BibBook} [9] {
	\bibitem{#1} % key
		\ifthenelse{\isempty{#2}}
			{} % author(s) is not presented
			{#2,} % author(s)
		\emph{#3}, % title
		\ifthenelse{\isempty{#4}}
			{} % first edition
			{#4 edition,} % edition
		#5, % publisher
		#6, % country
		#7, % date of publication		
		#8 pages,
		ISBN: \mbox{#9}
}

\newcommand{\BibArticle} [9] {
	\bibitem{#1} % key
		#2, % author(s)
		\emph{#3}, % title
		#4, % journal
		\ifthenelse{\isempty{#5}}
			{}
			{Vol. #5,} % volume
		\ifthenelse{\isempty{#6}}
			{}
			{No. #5,} % number
		#7, % country
		\ifthenelse{\isempty{#9}}
			{#8} % date
			{#8, \PagesWord\ #9} % date and pages
}

\newcommand{\BibDissertation} [7] {
	\bibitem{#1} % key
		#2, % author(s)
		\emph{#3}. % title
		\DoctoralThesisWord, #4, % school
		#5, % address
		#6 % date
		%#7 pages,
}

\newcommand{\BibMastersThesis} [7] {
	\bibitem{#1} % key
		#2, % author(s)
		\emph{#3}. % title
		\MastersThesisWord, #4, % school
		#5, % address
		#6 % date
		%#7 pages,
}

\newcommand{\BibProceedings} [6] {
	\bibitem{#1} % key
		#2, % author(s)
		\emph{#3}, % title
		#4, % conference
		#5, % address
		#6 % date
}

\newcommand{\BibManual} [5] {
	\bibitem{#1} % key
		#2, % author(s)
		\emph{#3}, % title
		#4. % date
		\AvailableAtWord: \url{#5}, % URL
		retrieved \BibDateFormat\today
}

\newcommand{\BibTechReport} [6] {
	\bibitem{#1} % key
		#2, % author(s)
		\emph{#3}, % title
		#4, % institution/organization
		#5, % country
		#6 % date
}

\newcommand{\BibWebArticle} [5] {
	\bibitem{#1} % key
		#2, % author(s)
		\emph{#3}, % title
		#4. % year
		\OnlineDocumentWord, \RetrievedFromWord: \url{#5}, % URL
		\RetrievedWord~\BibDateFormat\today
}

\newcommand{\BibWebDocument} [4] {
	\bibitem{#1} % key
		\emph{#2},  % title
		#3. % year
		\OnlineDocumentWord, \RetrievedFromWord: \url{#4}, % URL
		\RetrievedWord~\BibDateFormat\today
}

\newcommand{\BibWebSite} [3] {
	\bibitem{#1} % key
		\emph{#2}.  % title
		\InternetPageWord, \RetrievedFromWord: \url{#3}, % URL
		\ReferredWord~\BibDateFormat\today
}

% to define and add references

\newcommand{\PlaceReferenceToSection} [1] {%
	\label{sec:#1}%
}

\newcommand{\PlaceReferenceToFigure} [1] {%
	\label{fig:#1}%
}

\newcommand{\PlaceReferenceToTable} [1] {%
	\label{tab:#1}%
}

\newcommand{\ReferenceToTable} [1] {%
	\tablename~\ref{tab:#1}%
}

\newcommand{\ReferenceToFigure} [1] {%
	\figurename~\ref{fig:#1}%
}

\newcommand{\ReferenceToSection} [1] {%
	Section~\ref{sec:#1}%
}

% to place an image
% requires the 'graphicx' package

\newcommand{\TableFigure} [3] {
	\begin{table}[placement=h]
		\begin{center}
			#3
			\caption{#1}
			\PlaceReferenceToTable{#2}
		\end{center}
	\end{table}
}

\newcommand{\PlaceImageFigure} [3] {
	\begin{figure}[placement=h]
  		\begin{center}
    		\includegraphics[scale=0.65]{#3} % file
    		\caption{#1} % caption
    		\PlaceReferenceToFigure{#2} % reference ID
  		\end{center}
	\end{figure}
}

\newcommand{\ImageFigure} [2] {%
	\PlaceImageFigure{#1}{#2}{#2}%
}

\newcommand{\ReproducedImageFigure} [3] {%
	\PlaceImageFigure{#1, reproduced from \Reference{#3}}{#2}{#2}%
}

% to print the given date (format is '<month short name> <year>')
% requires the 'datetime' package

\newcommand{\BibDate} [2] {%
	\monthname[#1] #2%
}

% to print the given date (format is 'dd.mm.yyyy')
% requires the 'datetime' package

\newdateformat{BibDateFormat} {%
	\twodigit{\THEDAY}.\twodigit{\THEMONTH}.\THEYEAR
	%\twodigit{\THEDAY}~\monthname~\THEYEAR
}

% to print the current year
% requires the 'datetime' package

\newdateformat{YearDateFormat} {%
	\THEYEAR%
}

\newcommand{\CurrentYear}{
	\YearDateFormat\today%
}

% to use abbreviations in the text
% requires the 'acronym' package

\newcommand{\Abbreviation} [1] {%
	\EnglishText{\ac{#1}}
}

\newcommand{\RussianAbbreviation} [1] {%
	\RussianText{#1}%
	%\RussianText{\ac{#1}}
}

\newcommand{\RussianAbbreviationShort} [1] {%
	\RussianText{#1}%
	%\RussianText{\acs{#1}}
}

% === DEFINITONS FOR APPENDICES ===

% the counter for appendices
\newcounter{AppendixCounter}
\setcounter{AppendixCounter}{0}
\newcommand{\AppendixNumber}{\TODO{\Alph{AppendixCounter}}} % \arabic{AppendixCounter}

% to format headers of the document pages
% requires the 'fancyhdr' package

\newcommand{\TheAppendixTitle}{}

\newcommand{\ClearPageStyle} {
	\pagestyle{fancy}
	\fancyhf{} % to reset existing page format
	\renewcommand{\headrulewidth}{0pt} % to remove a page header line
}

\newcommand{\ApplyGeneralPageStyle} {
	\ClearPageStyle
	\cfoot{\thepage} % a page number center-aligned in the page footer
}

% to format a header of an appendix page
% requires the 'fancyhdr' package

\newcommand{\ApplyOnePageAppendixPageStyle} {
	\ClearPageStyle	
	\rhead{\TheAppendixTitle}
}

% to format a header of an appendix page
% requires the 'fancyhdr' package

\newcommand{\ApplyAppendixFirstPageStyle} {
	\ClearPageStyle
	\rhead{\TheAppendixTitle}
	\rfoot{(continue)}
}

% to format a header of an appendix page
% requires the 'fancyhdr' package

\newcommand{\ApplyAppendixPageStyle} {
	\ClearPageStyle
	\rhead{\MakeUppercase{\AppendixTitle} \AppendixNumber. (continued)}
	\rfoot{(continue)}
}

% to format a header of an appendix page
% requires the 'fancyhdr' package

\newcommand{\ApplyAppendixLastPageStyle} {
	\ClearPageStyle
	\rhead{\MakeUppercase{\AppendixTitle} \AppendixNumber. (continued)}
}

% to format an appendix page and to add it's title to the Table of Contents
% doesn't require any additional package

\newcommand{\TheAppendixPageTitle} [1] {
	\MakeUppercase{\AppendixTitle} \AppendixNumber. #1
}

\newcommand{\TheAppendixTocTitle} [1] {
	\MakeUppercase{\AppendixTitle} \AppendixNumber. #1\par
}

\newcommand{\TheAppendix} [1] {
	\ifthenelse { \equal{\arabic{AppendixCounter}}{0} } {
		\cleardoublepage
		\phantomsection
		\addcontentsline{toc}{section}{\MakeUppercase{\AppendicesTitle}}
	}
	{
		% nothing
	}

	\addtocounter{AppendixCounter}{1}
	\addtocontents{toc}{\TheAppendixTocTitle{#1}}
	\section*{\TheAppendixPageTitle{#1}}
	
	\renewcommand{\TheAppendixTitle}{\TheAppendixTocTitle{#1}}
}
