% === DEFINITIONS FOR TEXT ITEMS ===

% to format headers of the document pages
% requires the 'fancyhdr' package

\newcommand{\ClearPageStyle} {
	\pagestyle{fancy}
	\fancyhf{} % to reset existing page format
	\renewcommand{\headrulewidth}{0pt} % to remove a page header line
}

\newcommand{\ApplyGeneralPageStyle} {
	\ClearPageStyle
	\cfoot{\thepage} % a page number center-aligned in the page footer
}

% to define own format of a section's title

\newcommand{\SectionTitle} [2] {%
	\section{\MakeUppercase{#1}}%
	\PlaceReferenceToSection{#2}%
	\vspace*{1em}%
}

\newcommand{\SubSectionTitle} [2] {%
	\subsection{#1}%
	\PlaceReferenceToSection{#2}%
	\vspace*{0.5em}%
}

\newcommand{\SubSubSectionTitle} [2] {%
	\subsubsection{#1}%
	\PlaceReferenceToSection{#2}%
	\vspace*{0.5em}%
}

\newcommand{\PreambleSectionTitle} [1] {%
	\section*{\MakeUppercase{#1}}%
	\vspace*{1em}%
}

\newcommand{\PreambleSubSectionTitle} [1] {%
	\subsection*{#1}%
	\vspace*{1em}%
}

% to define own delimiters between paragraphs and delimiters

\newcommand{\Paragraph} [1] {
	\ifthenelse{\isempty{#1}}
		{\par\vspace*{1em}}
		%{\par\vspace*{1em}}
		{\par\colorbox{paragraphdescriptionbgcolor}{#1}\par} % {\par\vspace*{1em}}
}

\newcommand{\Sentence} {}

% to define new text colors
% requires the 'color' package

\definecolor{commentcolor}{gray}{0.2}
\definecolor{paragraphdescriptionbgcolor}{RGB}{255,248,220} % pale yellow (cornsilk)
\definecolor{remarkcolor}{gray}{0.2}
\definecolor{suggestioncolor}{RGB}{235,140,0} % dark orange
\definecolor{websitecolor}{gray}{0.4}

% to use citations to bibliography items
% requires the 'color' package

\newcommand{\Reference} [1] {%
	\cite{#1}%
}

\newcommand{\WebSite} [1] {%
	\textcolor{websitecolor}{\cite{#1}}%
}

% to define and add references

\newcommand{\PlaceReferenceToSection} [1] {%
	\label{sec:#1}%
}

\newcommand{\PlaceReferenceToAppendix} [1] {%
	\label{sec:#1}%
}

\newcommand{\PlaceReferenceToFigure} [1] {%
	\label{fig:#1}%
}

\newcommand{\PlaceReferenceToTable} [1] {%
	\label{tab:#1}%
}

\newcommand{\ReferenceToTable} [1] {%
	\tablename~\ref{tab:#1}%
}

\newcommand{\ReferenceToFigure} [1] {%
	\figurename~\ref{fig:#1}%
}

\newcommand{\ReferenceToSection} [1] {%
	\SectionWord~\ref{sec:#1}%
}

\newcommand{\InReferenceToSection} [1] {%
	\InSectionWord~\ref{sec:#1}%
}

\newcommand{\ReferenceToAppendix} [1] {%
	\AppendixWord~\ref{sec:#1}%
}

% to place an image
% requires the 'graphicx' package

\newcommand{\TableFigure} [3] {
	\begin{table}[placement=h]
		\begin{center}
			#3
			\caption{#1}
			\PlaceReferenceToTable{#2}
		\end{center}
	\end{table}
}

\newcommand{\PlaceImageFigure} [3] {
	\begin{figure}[placement=h]
  		\begin{center}
    		\includegraphics[scale=0.65]{#3} % file
    		\caption{#1} % caption
    		\PlaceReferenceToFigure{#2} % reference ID
  		\end{center}
	\end{figure}
}

\newcommand{\ImageFigure} [2] {%
	\PlaceImageFigure{#1}{#2}{#2}%
}

\newcommand{\ReproducedImageFigure} [3] {%
	\PlaceImageFigure{#1 \Reference{#3}}{#2}{#2}%
}

\newcommand{\Listing} [3] {
	\lstinputlisting[
		caption={#1}, 
		basicstyle=\footnotesize\ttfamily, 
		language=#2, 
		frame=single, 
		showstringspaces=false,
		tabsize=2]
		{#3}
}

\newcommand{\NonTitledListing} [3] {
	\lstinputlisting[
		title={\Bold{#1}}, 
		basicstyle=\footnotesize\ttfamily, 
		language=#2, 
		showstringspaces=false,
		tabsize=2]
		{#3}
}

% to use abbreviations in the text
% requires the 'acronym' package

\newcommand{\Abbreviation} [1] {%
	\EnglishText{\ac{#1}}
}

\newcommand{\RussianAbbreviation} [1] {%
	\RussianText{#1}%
	%\RussianText{\ac{#1}}
}

\newcommand{\RussianAbbreviationShort} [1] {%
	\RussianText{#1}%
	%\RussianText{\acs{#1}}
}

% to print the current year
% requires the 'datetime' package

\newdateformat{YearDateFormat} {%
	\THEYEAR%
}

\newcommand{\CurrentYear}{
	\YearDateFormat\today%
}
