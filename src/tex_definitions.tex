% === DEFINITIONS FOR TEXT ITEMS ===

% to format headers of the document pages
% requires the 'fancyhdr' package

\newcommand{\ClearPageStyle} {
	\pagestyle{fancy}
	\fancyhf{} % to reset existing page format
	\renewcommand{\headrulewidth}{0pt} % to remove a page header line
}

\newcommand{\ApplyGeneralPageStyle} {
	\ClearPageStyle
	\cfoot{\thepage} % a page number center-aligned in the page footer
}

% to define own format of a section's title

\newcommand{\SectionTitle} [2] {%
	\section{\texorpdfstring{\MakeUppercase{#1}}{#1}}%
	\PlaceReferenceToSection{#2}%
	\vspace*{1em}%
}

\newcommand{\SubSectionTitle} [2] {%
	\subsection{#1}%
	\PlaceReferenceToSection{#2}%
	%\vspace*{0.5em}%
}

\newcommand{\SubSubSectionTitle} [2] {%
	\subsubsection{#1}%
	\PlaceReferenceToSection{#2}%
	%\vspace*{0.5em}%
}

\newcommand{\PreambleSectionTitle} [1] {%
	\section*{\MakeUppercase{#1}}%
	\vspace*{1em}%
}

\newcommand{\PreambleSubSectionTitle} [1] {%
	\subsection*{#1}%
	%\vspace*{1em}%
}

% to define own delimiters between paragraphs and sentences

\newcommand{\Paragraph} [1] {
	\ifthenelse{\isempty{#1}}
		{\par\vspace*{1em}}
		{\par\vspace*{1em}}
		%{\par\colorbox{paragraphdescriptionbgcolor}{#1}\par} % {\par\vspace*{1em}}
}

\newcommand{\Sentence} {}

\newcommand{\Remark} [1] {
	\par\colorbox{remarkbgcolor}{#1}\par%
}

\NewEnviron{remarks} {%
	\par\colorbox{remarkbgcolor}{\BODY}\par%
}

% to define new text colors
% requires the 'color' package

\definecolor{commentcolor}{gray}{0.2}
\definecolor{paragraphdescriptionbgcolor}{RGB}{255,248,220} % pale yellow (cornsilk)
\definecolor{remarkcolor}{gray}{0.2}
\definecolor{remarkbgcolor}{RGB}{255,160,122}
\definecolor{suggestioncolor}{RGB}{235,140,0} % dark orange
\definecolor{websitecolor}{gray}{0.4}
\definecolor{possiblebadtranslationcolor}{RGB}{64,0,0}

% to mark a text translated possibly bad

\newcommand{\T} [1] {%
	\textcolor{possiblebadtranslationcolor}{\underline{#1}}%
}

\newcommand{\A} {%
	\T{a}
}

\newcommand{\Aa} {%
	\T{A}
}

\newcommand{\An} {%
	\T{an}
}

\newcommand{\AAn} {%
	\T{An}
}

\newcommand{\The} {%
	\T{the}
}

\newcommand{\TThe} {%
	\T{The}
}

% to use citations to bibliography items and citations under chapter titles
% requires the 'color' package

\newcommand{\Reference} [1] {%
	\cite{#1}%
}

\newcommand{\WebSite} [1] {%
	%\textcolor{websitecolor}{\cite{#1}}%
	\Reference{#1}%
}

\newcommand{\Citation} [2] {%
	\hfill \Italic{#1}.

	\hfill #2
}

% to define and add references

\newcommand{\PlaceReferenceToSection} [1] {%
	\label{sec:#1}%
}

\newcommand{\PlaceReferenceToAppendix} [1] {%
	\label{sec:#1}%
}

\newcommand{\PlaceReferenceToFigure} [1] {%
	\label{fig:#1}%
}

\newcommand{\PlaceReferenceToTable} [1] {%
	\label{tab:#1}%
}

\newcommand{\ReferenceToTable} [1] {%
	\TableWord~\ref{tab:#1}%
}

\newcommand{\InReferenceToTable} [1] {%
	\InTableWord~\ref{tab:#1}%
}

\newcommand{\ReferenceToFigure} [1] {%
	\FigureWord~\ref{fig:#1}%
}

\newcommand{\OnReferenceToFigure} [1] {%
	\OnFigureWord~\ref{fig:#1}%
}

\newcommand{\ReferenceToSection} [1] {%
	\SectionWord~\ref{sec:#1}%
}

\newcommand{\InReferenceToSection} [1] {%
	\InSectionWord~\ref{sec:#1}%
}

\newcommand{\ReferenceToAppendix} [1] {%
	\AppendixWord~\ref{sec:#1}%
}

\newcommand{\InReferenceToAppendix} [1] {%
	\InAppendixWord~\ref{sec:#1}%
}

% to place an image
% requires the 'graphicx' package

\newcommand{\TableFigure} [3] {
	\Paragraph{}
	\begin{table}[placement=h]
		\begin{center}
			#3
			\caption{#1}%
			\PlaceReferenceToTable{#2}%
		\end{center}
	\end{table}
}

\newcommand{\PlaceImageFigure} [3] {
	\Paragraph{}
	\Paragraph{}
	\begin{figure}[placement=h]
  		\begin{center}
    		\includegraphics[scale=0.65]{#3} % file
    		\caption{#1} % caption
    		\PlaceReferenceToFigure{#2} % reference ID
  		\end{center}
	\end{figure}
}

\newcommand{\ImageFigure} [2] {%
	\PlaceImageFigure{#1}{#2}{#2}%
}

\newcommand{\ReproducedImageFigure} [3] {%
	\PlaceImageFigure{#1 \Reference{#3}}{#2}{#2}%
}

% to use source code listings in the thesis

\newcommand{\Listing} [3] {
	\Paragraph{}
	\lstinputlisting[
		caption={#1}, 
		basicstyle=\footnotesize\ttfamily, 
		language=#2, 
		frame=single, 
		showstringspaces=false,
		tabsize=2]
		{#3}
}

\newcommand{\AppendixTitledListing} [3] {
	\lstinputlisting[
		title={\Bold{#1}}, 
		language=#2, 
		breaklines=true, 
		basicstyle=\footnotesize\ttfamily, 
		commentstyle=\footnotesize\ttfamily, 
		identifierstyle=, 
		keywordstyle=, 
		stringstyle=, 
		showstringspaces=false, 
		tabsize=2]
		{#3}
}

\newcommand{\AppendixNonTitledListing} [2] {
	\lstinputlisting[
		language=#1, 
		breaklines=true, 
		basicstyle=\footnotesize\ttfamily, 
		commentstyle=\footnotesize\ttfamily, 
		identifierstyle=, 
		keywordstyle=, 
		stringstyle=, 
		showstringspaces=false, 
		tabsize=2]
		{#2}
}


% to use abbreviations in the text
% requires the 'acronym' package

\newcommand{\Abbreviation} [1] {%
	\EnglishText{\ac{#1}}%
}

\newcommand{\RussianAbbreviation} [1] {%
	#1%
	%\ac{#1}
}

% to print the current year
% requires the 'datetime' package

\newdateformat{YearDateFormat} {%
	\THEYEAR%
}

\newcommand{\FourDigitsDate} [3] {%
	\twodigit{#3}.\twodigit{#2}.#1
}

\newcommand{\CurrentYear}{
	\YearDateFormat\today%
}
