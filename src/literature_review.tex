\SectionTitle{LITERATURE REVIEW}
\begin{comment}
WRITING THE THESIS

The literature review should discuss all of the research that has been done on the subject.  
How you group the discussion will depend on your project, but be sure to come up with a 
logical organization before you begin writing.  This is the discussion and analysis of the library 
research you've been doing for the past 12 months.  How many studies should be included will 
depend on the topic, but a general baseline is 75 to 100 references (although many topics will 
appropriately have many more than this). 

The literature review should begin with a reiteration of the purpose of your study.  This 
should be followed by a preview of what is to come in the literature review.  This lays out the 
overall organization of specific topics you will cover. 

The purpose of the literature review is to concisely demonstrate your level of understanding 
of the research related to your project.  You should not discuss all of the literature in‐depth.  
Rather you should group your literature according to some general topics and only discuss 
specific studies if they are “landmark” studies for your area of research (there should be 6‐10 of 
these).  Each of these specific discussions should include specific information about the group 
involved in the research project, data, and results reported.  Often a review of literature will 
include several of these in‐depth reviews with “mini‐reviews” of studies that came to the same 
or similar conclusions.  The literature review should end with a discussion of how the literature 
relates to your study.

HOW TO ORGANIZE YOUR THESIS

1. Review of the State of the Art
Here you review the state of the art relevant to your thesis.
The idea is to present (critical analysis comes a little bit later) the major ideas in the state 
of the art right up to, but not including, your own personal brilliant ideas.
You organize this section by idea, and not by author or by publication.

2.Research Question or Problem Statement
Engineering theses tend to refer to a "problem" to be solved where other disciplines talk in terms 
of a "question" to be answered. In either case, this section has three main parts:

1. a concise statement of the question that your thesis tackles 
2. justification, by direct reference to section 3, that your question is previously unanswered
3. discussion of why it is worthwhile to answer this question.

Item 2 above is where you analyze the information which you presented in Section 3. 
For example, maybe your problem is to "develop a Zylon algorithm capable of handling very large 
scale problems in reasonable time" (you would further describe what you mean by "large scale" and 
"reasonable time" in the problem statement). Now in your analysis of the state of the art you 
would show how each class of current approaches fails (i.e. can handle only small problems, 
or takes too much time). In the last part of this section you would explain why having a large-scale 
fast Zylon algorithm is useful; e.g., by describing applications where it can be used.

Since this is one of the sections that the readers are definitely looking for, highlight it by 
using the word "problem" or "question" in the title: e.g. "Research Question" or "Problem Statement", 
or maybe something more specific such as "The Large-Scale Zylon Algorithm Problem."

HOW TO WRITE A THESIS

This should be a critical synthesis of the state of the knowledge. Especially important are the
areas needing further investigation: what has not been done, as well as what has been done, but 
for which there is a conflict in the literature. The examiner finds out how the candidate thinks
from reading this section.

The literature review is the backdrop on which you present your work. It must be selective,
but substantial enough for the merits of your work to be judged in relation to what is
known. It is especially critical for a PhD thesis where the claim of originality should be
defended with a thorough and critical review of the literature, especially in your specific
area of research. You should capture the essence of current knowledge and comment
critically on where the interesting questions and inconsistencies lie. The literature review
is vital to justify your hypothesis, which must be consistent with what is known. If you
present your literature review objectively but selectively, so that it does not stick out as
an extraneous chapter, but merges into the larger story of your thesis, you would have
done well.
\end{comment}