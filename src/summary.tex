\section{SUMMARY}

\subsection{Тема}
Это тема

\subsection{Цель}
Это цель

\subsection{Критерии оценки диплома}

\begin{suggestions}
Можно расписать, что в каждом пункте сделано и на каком уровне
\end{suggestions}

\begin{itemize}
  \item Постановка проблемы, цели, определения и установки границ диссертации
  \item Связь с предыдущими исследованиями
  \item Подход к исследованию, выбранные методы исследования и используемые материалы
  \item Расписание проведения исследование
  \item Полученные результы и их анализ
  \item Организация и логичность работы
  \item Глубина исследования
  \item Достоверность работы
  \item Язык и компоновка работы
  \item Независимость выбранного методы исследования и его применение
\end{itemize}

\begin{comment}
INSTRUCTIONS FOR WRITING A MASTER'S THESIS

The last section of the work is the summary. This section summarizes the results and the
conclusions from the results. This section gives answers for the promises given in the
abstract and in the introduction. Items are only listed, the presentation becomes short in
this sense, since everything is proved, shown, motivated, and presented in earlier sections
of the thesis. There is nothing new given in this section, no new conclusions, no new
proposals for the future work. Maybe the only new thing is the criticism directed at the
work performed by the author. In the thesis this section occupies only one page, the
presentation becomes short and assertive.
\end{comment}