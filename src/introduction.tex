\section{INTRODUCTION}
\todo{Limitations, expected results}
\nextparagraph{}
\todo{Introduction}

\subsection{Background}

\todo{Background}
%
\begin{comment}
INSTRUCTIONS FOR WRITING A MASTER'S THESIS

The goal of this report is to give instructions how to write a master’s thesis. The report is
only an outline, a model which will be extended according to the contents of the actual
work for the thesis. For more information, read “Final thesis instructions” in the study
guide, discuss with your supervisor, visit appropriate course web pages, see for example
[1]. When you are referring to pages in the World Wide Web, see the instruction for citing
and referencing from the LUT Library web site. Do not use footnotes or do not write URLs
within the text.

Introduction contains three subsections: background, goals and delimitations, and
description of the structure of the thesis. Use this sectioning. Subsection 1.1 includes an
introduction to the background for the work. Remember that the abstract is a separate
piece of text. The introduction should be written independently such that one does not need
to read the abstract to understand the introduction. The introduction is written in a general
level instead of many details present. These details will be explained later starting from
section 2. The paragraphs have more than only one sentence. In the thesis, this subsection
occupies from 1 to 2 pages.

Remember to introduce the abbreviations when they are used in the text for the first time.
For example: ”This thesis is about the games played in National Hockey League (NHL) in
seasons 1900-2000. The annual penalties in NHL have ... “.

The introduction is written such that the reader is interested to continue to read the full
thesis. And if this interest is arisen then the author is ready to give some general
descriptions for the contents of the thesis and reading guidelines for the rest of the text in
the thesis.

\end{comment}

\subsection{Goals and deliminations}

\todo{Goals and deliminations}
%
\begin{comment}
INSTRUCTIONS FOR WRITING A MASTER'S THESIS

Express the goals for the work, include also the delimitations. This way the reader knows
when the results are valid and she can place the work in a proper framework and scope. 
It is also important to say, what is not done during the work for the thesis. Then the
thesis will show how the goals are met. In the thesis this subsection occupies from 1 to 2
pages.
\end{comment}

\subsection{Structure of the thesis}

\todo{Structure of the thesis}
%
\begin{comment}
INSTRUCTIONS FOR WRITING A MASTER'S THESIS

This subsection contains a short description for the contents of the thesis. The contents of
each section are characterized with one or two sentences. For example: ”Section 2 contains
a description of the ...”. In the thesis this subsection occupies at most one page, in many
cases half a page is enough. At this point, one should thoroughly consider the structure of
work. Discuss with your supervisor about the structure.
\end{comment}
%
\begin{comment}
FINAL THESIS INSTRUCTIONS.

The actual research report is opened with an introduction. The purpose of the introduction is to
introduce the topic and awaken the reader's interest. The introduction briefly describes the
background, material extent and aims of the thesis. The introduction relates the thesis to other
research and sources and presents the research methodology applied. It also describes the key
points and organisation of the research report. It does not, however, include detailed descriptions 
of the theory, methods or results. A good introduction is, nevertheless, significantly longer than a
couple of pages, and is organised in a logical manner.

JONI'S HOWTO FOR WRITING MSC/BSC THESIS

Start by writing the introduction (even before any coding) and make the following section:
Background and motivation.
   
Write this section very carefully. Think: 
1) Why am I doing this? and 
2) Why solving this problem
is important? and especially
3) How the same thing has been done elsewhere or how similar things
have been done elsewhere?

I.e. start by motivating why you are doing what you are doing, find what others have done, and write
this down very carefully. This section sets the main basis for all remaining chapters. Trust me,
doing this first will benefit your work!

Rule-Of-Thumb-1: There is no such topic that nobody has not done it before
OR nobody has not done a similar thing before.

Good thesis is not a 'great innovation of your own', but a great view to related works and
utilisation of the best existing methods to solve your problem.

Therefore, google out what has been done. From web pages look for articles explaining what they have
done. If you find a book on your topic, have it as a bed side book and read it. If you want to be a
real professional, spend 2 hours in library and use their search engines to find appropriate
articles (journals only), e.g. from IEEE and Elsevier (these articles are often found with Google as
well, but you may need library services to access the articles).
   
Rule-Of-Thumb-2: The more time you spend on thinking and reading instead of doing, the better your
thesis will be.

HOW TO WRITE A (THESIS / DISSERTATION) PROPOSAL

a. It’s not a literature review! It should be a summary of existing evidence that
motivates your specific, proposed work.
b. Start broad (e.g. injuries, need for ergonomics, etc.), become increasingly specific
c. End with a review, and broaden out to discuss potential applications (importance)
of the proposed work
d. Topics to be addressed: what’s been done; what hasn’t; what is needed and why;
indicate your part or contribution (scoping your domain)
e. Intro should contain some statements of objectives, purposes, and hypothesis.
Placement is flexible, though, and these could be in separate sections between
Intro and Methods, or even part of the Methods. Depending on the specifics, not
all of these (objective, purposes, and hypotheses) will always been relevant.
More important that it be clear and readable.
f. How long should it be? Long enough to satisfy the above goals. Typically 10-30
pages for an MS, longer for a PhD proposal.
g. When summarizing existing literature, it is not enough just to describe what
authors X, Y, and Z did. Results should be interpreted, in the context of the
overall review and study objectives.
h. In particular, discuss contrasting evidence, possible sources for discrepancies
(experimental design, lack of controls, sensitivity of measures, etc.), and the
importance of resolving the differences.
i. Summarize evidence, rarely individual studies

HOW TO ORGANIZE YOUR THESIS

This is a general introduction to what the thesis is all about - it is not just a description of
the contents of each section. Briefly summarize the question (you will be stating the question 
in detail later), some of the reasons why it is a worthwhile question, and perhaps give an overview 
of your main results. This is a birds-eye view of the answers to the main questions answered in 
the thesis (see above).

WRITING THE THESIS

The introduction should introduce the thesis.  This is not a summary of the thesis.  It is not a 
brief version of each chapter.  It is an introduction to the topic.  Introduce the subject.  In 
general terms, what does your study address?  Why is it important?  Where does is fit in the 
overall field?  Be sure to include in the introduction a clear statement of your hypothesis and 
how you are going to address it.  Throughout the introduction you should use citations from the 
research literature to support your study.  These citations should include but not be limited to 
research presented in the Literature Review.
The following are suggested topics that are usually covered in the introduction.
- Statement of the Problem.  You should succinctly state the problem that your thesis is 
going to address.  You should also present relevant information about why this is an 
important problem. 
- Background and Need.  You should present relevant literature that supports the need for 
your project.  Research articles, books, educational and government statistics are just a 
few sources that should be used here.  This section can include brief overviews of articles 
covered in the literature review that support the need for your project 
- Rationale.  You should carefully present the model or theory that underlies the project.  
The rationale should define the larger problem being investigated, summarize what is known about 
the problem, define the gap(s) in the knowledge, and state what needs to  be done to address the
gap(s).
- Purpose of the Project. Based on the above background information, explain the purpose 
of the study.  Explain what you hope the study will accomplish and why you chose to do 
this particular study.  This should be supported with citations and specific information 
related to the study. 
- Research Questions/Hypotheses. Given the background above, you carefully state the 
hypothesis(ses) that will be tested in your thesis. 
- Methods. Briefly (as you will cover this in‐depth in a later chapter) describe the methods 
that were used in your study (i.e., research methods, variables, instrumentation, 
participants, pilot, analysis of data). 
- Limitations. Begin this with a summary of the document thus far to provide a background 
for any limitations to this study.  Be very specific, for example the population to which 
your findings will be limited. 

HOW TO WRITE A THESIS

The Introduction should embody the (unified) hypothesis. The reader finds in a clearly expressed
hypothesis the skeleton of the thesis on which hangs all of the skin and meat that will be 
presented later.

Remember that the introductory pages are important because they create the first, and perhaps
lasting, impression on the examiner. Use flow diagrams, headings, sub-headings etc., to create 
and sustain interest.

The Introduction is where you “soft launch” your reader on the work described in your
thesis. Lead the reader from the known to the unknown. State the hypothesis clearly.
Give a preview of your thesis, globally and chapter by chapter. Your Introduction has
done its work if you have captured the reader’s curiosity and interest in this first chapter.
\end{comment}