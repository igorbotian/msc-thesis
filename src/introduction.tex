\section{INTRODUCTION}

\todo{Introduction}

\subsection{Background}

\todo{Background}
%
\begin{comment}
INSTRUCTIONS FOR WRITING A MASTER'S THESIS

The goal of this report is to give instructions how to write a master’s thesis. The report is
only an outline, a model which will be extended according to the contents of the actual
work for the thesis. For more information, read “Final thesis instructions” in the study
guide, discuss with your supervisor, visit appropriate course web pages, see for example
[1]. When you are referring to pages in the World Wide Web, see the instruction for citing
and referencing from the LUT Library web site. Do not use footnotes or do not write URLs
within the text.

Introduction contains three subsections: background, goals and delimitations, and
description of the structure of the thesis. Use this sectioning. Subsection 1.1 includes an
introduction to the background for the work. Remember that the abstract is a separate
piece of text. The introduction should be written independently such that one does not need
to read the abstract to understand the introduction. The introduction is written in a general
level instead of many details present. These details will be explained later starting from
section 2. The paragraphs have more than only one sentence. In the thesis, this subsection
occupies from 1 to 2 pages.

Remember to introduce the abbreviations when they are used in the text for the first time.
For example: ”This thesis is about the games played in National Hockey League (NHL) in
seasons 1900-2000. The annual penalties in NHL have ... “.

The introduction is written such that the reader is interested to continue to read the full
thesis. And if this interest is arisen then the author is ready to give some general
descriptions for the contents of the thesis and reading guidelines for the rest of the text in
the thesis.

\end{comment}

\subsection{Goals and deliminations}

\todo{Goals and deliminations}
%
\begin{comment}
INSTRUCTIONS FOR WRITING A MASTER'S THESIS

Express the goals for the work, include also the delimitations. This way the reader knows
when the results are valid and she can place the work in a proper framework and scope. 
It is also important to say, what is not done during the work for the thesis. Then the
thesis will show how the goals are met. In the thesis this subsection occupies from 1 to 2
pages.
\end{comment}

\subsection{Structure of the thesis}

\todo{Structure of the thesis}
%
\begin{comment}
INSTRUCTIONS FOR WRITING A MASTER'S THESIS

This subsection contains a short description for the contents of the thesis. The contents of
each section are characterized with one or two sentences. For example: ”Section 2 contains
a description of the ...”. In the thesis this subsection occupies at most one page, in many
cases half a page is enough. At this point, one should thoroughly consider the structure of
work. Discuss with your supervisor about the structure.
\end{comment}
%
\begin{comment}
FINAL THESIS INSTRUCTIONS.

The actual research report is opened with an introduction. The purpose of the introduction is to
introduce the topic and awaken the reader's interest. The introduction briefly describes the
background, material extent and aims of the thesis. The introduction relates the thesis to other
research and sources and presents the research methodology applied. It also describes the key
points and organisation of the research report. It does not, however, include detailed descriptions 
of the theory, methods or results. A good introduction is, nevertheless, significantly longer than a
couple of pages, and is organised in a logical manner.

JONI'S HOWTO FOR WRITING MSC/BSC THESIS

Start by writing the introduction (even before any coding) and make the following section:
Background and motivation.
   
Write this section very carefully. Think: 
1) Why am I doing this? and 
2) Why solving this problem
is important? and especially
3) How the same thing has been done elsewhere or how similar things
have been done elsewhere?

I.e. start by motivating why you are doing what you are doing, find what others have done, and write
this down very carefully. This section sets the main basis for all remaining chapters. Trust me,
doing this first will benefit your work!

Rule-Of-Thumb-1: There is no such topic that nobody has not done it before
OR nobody has not done a similar thing before.

Good thesis is not a 'great innovation of your own', but a great view to related works and
utilisation of the best existing methods to solve your problem.

Therefore, google out what has been done. From web pages look for articles explaining what they have
done. If you find a book on your topic, have it as a bed side book and read it. If you want to be a
real professional, spend 2 hours in library and use their search engines to find appropriate
articles (journals only), e.g. from IEEE and Elsevier (these articles are often found with Google as
well, but you may need library services to access the articles).
   
Rule-Of-Thumb-2: The more time you spend on thinking and reading instead of doing, the better your
thesis will be.
\end{comment}