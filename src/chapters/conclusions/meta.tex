\newcommand{\ConclusionsFolderName}{conclusions}
%
\SectionTitle{\ConclusionsTitle}{conclusions}
\IncludeSection{\ConclusionsFolderName}{conclusions}
%
\begin{comment}
INSTRUCTIONS FOR WRITING A MASTER'S THESIS

The last section of the work is the summary. This section summarizes the results and the
conclusions from the results. This section gives answers for the promises given in the
abstract and in the introduction. Items are only listed, the presentation becomes short in
this sense, since everything is proved, shown, motivated, and presented in earlier sections
of the thesis. There is nothing new given in this section, no new conclusions, no new
proposals for the future work. Maybe the only new thing is the criticism directed at the
work performed by the author. In the thesis this section occupies only one page, the
presentation becomes short and assertive.
\end{comment}
%
\begin{comment}
Заключение суммирует все результаты и выводы, полученне из результатов.
Заключение даёт ответы на вопросы и ожидания, перечисленные во введении.
Пункты перечисляются кратко, так как всё уже продемонстрировано и доказано в предыдущих разделах.
Никаких новых выводов не делается, ничего нового, за исключением критики в адрес проведённого исследования.
В заключении содержится оценка соответствия теории и измерений.
Указываются причины необходимости дальнейшего исследования и  возможные варианты практического применения.

Заключение можно организовать следующим образом:
1. Выводы (утверждения, кратко перечислить, что получено в данной работе, начиная с наиболее важного и заканчивая наименее важным, должны быть чётко связаны с вопросом исследования и проблемами, поднимаемые в разделах)
2. Краткое изложение вклада автора диссертации (тоже нумерованным списком, начинать с наиболее важного и заканчивая наименее важным, могут пересекаться с выводами)
3. Дальнейшее исследование (что можно сделать в будущем, преимущества идей, полученных в ходе данной работы)

=== 

- отразить суть и ценность проделанной работы;
- отметить преимущества предлагаемого варианта решения или предмета рассмотрения;
- привести конкретные предложения и рекомендации для практического внедрения результатов 
исследования или указать место и время уже состоявшегося внедрения работы;
- указать семинары, конференции, публикации и.т.д., где уже была представлена данная работа или ее 
части;
- дать рекомендации по перспективе дальнейшей работы в этой области;
- отметить нерешенные задачи.

===

Рябченко:
ВЫВОДЫ
- необходимо много фотографий, так как куча всего влияет на результаты
- цель работы была такая-то, чтобы получить то-то
- судя по фоткам, можно сказать, что алгоритм работает хорошо. но у него есть недостаток такой-то, который может привести к тому-то. однако если что-то улучшить, всё будет пучком. но в любом случае это важно, так как влияет на что-то
- алгоритм был использован в такой цветовой схеме, но нет причин, чтобы его не использовать в других схемах
- метод был разработан из работ какого-то чувака
- есть ещё похожее исследование в данной области, но там что-то немного другое, но может быть использовано в совокупности
ДАЛЬНЕЙШЕЕ ИССЛЕДОВАНИЕ
- вывод о том, что необходимо разработать алгоритм
- при этом было бы очень полезно что-то принять во внимание
- есть один путь получить более лучший результат
- один элемент, который не был рассмотрен в данной диссертации
- необходимо провести намного больше экспериментов, чтобы получить более точные результаты и улучшить алгоритм
ВОЗМОЖНЫЕ ПРИМЕНЕНИЯ
- одно из возможных применений - паспортный контроль
- алгоритм может быть использован в кинопроизводстве
- ну и его можно применить в медицинском оборудовании

Никандрова:
- мотивация данной работы была такая-то. метод, предлагаемый в данном исследовании, делает то-то и то-то
- первой целью была такая-то. она была достигнута так-то. выводы такие-то
- вторая цель...
- результаты анализа литературы
- основным вопросом для исследования является такой-то. полученный алгоритм решает его так-то. но он имеет такой-то недостаток
- описание предлагаемого метода
- выводы по полученным результатам экспериментов
- где может быть использован метод и дальнейшие направления в исследовании

Калвар:
ВЫВОДЫ
- получены такие-то результаты
ОБСУЖДЕНИЕ
- вопросы исследования
- что было исследовано
- вывод по исследованию
- цель исследования
- ответы на вопросы исследования
- вывод по ответам
ДАЛЬНЕЙШЕЕ ИССЛЕДОВАНИЕ
- сложности в области проведённого исследования
- но что они могут быть хоть слегка решены с помощью исследования

Строкина:
- первая цель исследования такая-то
- о применяемых методах для её достижения. что было получено. достоинства и недостатки их применения
- вторая цель...
- из чего состояли вторые и третьи части эксперимента
- что было получено в итоге
- какие недостатки есть. что может быть улучшено, что не рассматривалось в данной работе, идеи по будущей работе

Лаакконен:
- что показали проведённые эксперименты. недостатки
- какие изменения рекомендуются
- какие требования есть к работе. как они достигаются
- что дало исследование касательно PeerHood, какие выводы они дали
- с чем связано дальнейшее исследование

Карки:
- цели исследования. какие мероприятия были проведены для его достижения. что дало
- общий вывод по проделанной работе. ответ на вопрос исследования
- недостатки и направления по дальнейшей работе

Колехмайнен:
ВЫВОДЫ
- цели работы. как они были достигнуты
- меропрятия, которые были проведены для их достижения
- анализ того, как qt подходит для peerhood
БУДУЩАЯ РАБОТА
- достоинства qt. необходимо развитие
- недостатки и то, что надо сделать
- как можно было бы улучшить peerhood с помощью qt

Яппинен:
- цель работы. как достигнута
- достоинства и недостатки применяемого подхода
- что даёт в дальнейшем
- что даёт пользователю
- надежды относительно перспективности направления исследования
\end{comment}
%
\begin{comment}
FINAL THESIS INSTRUCTIONS.

Depending on the nature and scope of the study, the report ends either with the chapter
"Conclusions", or two separate chapters, e.g. "Conclusions" and "Summary". The conclusions
analyse the observations and results drawn from the research. The conclusions examine and reflect
on e.g. the compatibility of the theory and measurements, the reasons for possible differences, and
summarise the conclusions drawn from the results. The need for further research and possible
practical applications may also be argued here.

INSTRUCTIONS FOR WRITING A MASTER'S THESIS

In most cases it is reasonable to present new findings and main results in a separate
section. This section also further analyzes observations and results. This section is also the
place to include the synthesis of the results, the future possibilities in the research topic or
the in the application area. After discussion and consideration one should present the
conclusions. This section contains the reasoning for those conclusions. In the next section,
in the summary, one just lists the conclusions.

HOW TO ORGANIZE YOUR THESIS

You generally cover three things in the Conclusions section, and each of these usually merits 
a separate subsection:

1. Conclusions 
2. Summary of Contributions 
3. Future Research

Conclusions are not a rambling summary of the thesis: they are short, concise statements 
of the inferences that you have made because of your work. It helps to organize these as short 
numbered paragraphs, ordered from most to least important. 
All conclusions should be directly related to the research question stated in Section 4.

Examples:
1. The problem stated in Section 4 has been solved: as shown in Sections ? to ??, 
an algorithm capable of handling large-scale Zylon problems in reasonable time has been developed.
2. The principal mechanism needed in the improved Zylon algorithm is the Grooty mechanism.
3. Etc.

The Summary of Contributions will be much sought and carefully read by the examiners. 
Here you list the contributions of new knowledge that your thesis makes. Of course, 
the thesis itself must substantiate any claims made here. There is often some overlap with the 
Conclusions, but that's okay. Concise numbered paragraphs are again best. 
Organize from most to least important. Examples:

1. Developed a much quicker algorithm for large-scale Zylon problems.
2. Demonstrated the first use of the Grooty mechanism for Zylon calculations.
3. Etc.

The Future Research subsection is included so that researchers picking up this work in future 
have the benefit of the ideas that you generated while you were working on the project. 
Again, concise numbered paragraphs are usually best.

WRITING THE THESIS

Results

This is a narrative presentation of your findings. This is where you present your statistics, 
tables, figures, etc. that show what the specific findings of your study are. Present them in 
detail. Remember that someone should be able to duplicate your study based solely on this 
document. This requires considerable description. 
 
It is very important not to try and combine this chapter with the next one.  You need to
carefully present your results first with no further interpretation.  Once you have presented the 
data you are ready to move on to the next section.

Discussion

This chapter should begin with a concise restatement of your study’s purpose along with any 
needed background information.  You should restate each of your hypotheses.  Now that you 
have presented the results in the previous section, discuss them in this section.  What, 
specifically, do the results mean?  How can they be interpreted?  Can they be interpreted in 
multiple ways?  What do the findings tell you about your hypothesis?  Do not claim more for 
your results than the data really shows.  Avoid speculation. 

Conclusions

This should summarize your results and discussion.  You should include a list of the most 
important findings of your study in descending order of importance.  You should also provide a 
statement about the possibility of future study.  What needs to be done and what does this 
study contribute? 

HOW TO WRITE A THESIS

The General Discussion or Conclusions integrate the whole thesis and present its main points at one
place. This should be done in the context of the unifying hypothesis of the thesis.
The Introduction and this chapter along with the Summary or Abstract are the most important parts of
the thesis.

The Conclusions record the power of your scientific thinking. You have to unite all
that has gone before with a “thread of unified perspective”. This is where you say why you
think your story is a good one and present evidence from your work to support your claim.
The fate of your hypothesis is revealed here: did it stand, fall, or require modification?
You may briefly compare your work with that of others, present whatever new knowledge
has been gained from your work, and suggest what may be done to further new knowledge.
The Conclusions should give a sense of fulfilment and finality to your thesis, and give the
reader some satisfaction that the time spent on reading it has not been in vain.

ОФОРМЛЕНИЕ ВКР

Заключение — это форма подведения итогов работы. Заключение, также как и выводы, формулируется в 
виде отдельных пунктов с порядковой нумерацией. Каждый пункт заключения обычно отражает суть работы,
описанной в одном из соответствующих разделов пояснительной записки. Однако заключение составляется 
именно по всей работе и поэтому его пункты не должны повторять выводы. Чаще всего их не боле семи, 
но знакомство с ними должно давать полное представление о проделанной работе. В заключении следует:
- отразить суть и ценность проделанной работы;
- отметить преимущества предлагаемого варианта решения или предмета рассмотрения;
- привести конкретные предложения и рекомендации для практического внедрения результатов 
исследования или указать место и время уже состоявшегося внедрения работы;
- указать семинары, конференции, публикации и.т.д., где уже была представлена данная работа или ее 
части;
- дать рекомендации по перспективе дальнейшей работы в этой области;
- отметить нерешенные задачи.
\end{comment}
