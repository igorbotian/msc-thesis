\Citation{Устаревают не только ответы, но и вопросы}{Эрнест Хемингуэй}
%
\Paragraph{Чему посвящена данная работа}
%
\Remark{Добавить собственную оценку тому, что было сделано на каждом этапе}
%
Данная работа посвящена проблеме безопасности пользовательских данных при проведении персональных коммуникаций в мобильной пиринговой среде. 
%
Одним из аспектов обеспечения их безопасности является безопасность \RussianAbbreviation{ПО}, функционирующего на мобильных устройствах. 
%
В данной работе рассматривалось \RussianAbbreviation{ПО} \IT{PeerHood}, направленное на обеспечение персональных коммуникаций в независимости от нижележащей сетевой технологии. 
%
Цель данной работы заключалась в оценке безопасности данного проекта. 

\Paragraph{Изучение проблемы безопасности ПО}
%
Первый шаг для её достижения заключался в изучении проблемы безопасности \RussianAbbreviation{ПО} в мобильном пиринговом окружении. 
%
Были рассмотрены понятие, факторы и угрозы безопасности \RussianAbbreviation{ПО}. 
%
Подробно рассмотрены известные дефекты и уязвимости безопасности \RussianAbbreviation{ПО}. 
%
Наконец, были исследованы угрозы безопасности, характерные как для мобильного, так и для пирингового окружения. 
%
Результаты изучения способствовали пониманию того, что представляет из себя проблема безопасности \RussianAbbreviation{ПО} и какие факторы на неё влияют в контексте мобильного пирингового окружения. 

\Paragraph{Изучение методов тестирования безопасности ПО}
%
Второй шаг состоял из выполнения следующих мероприятий. 
%
Было изучено понятие защищённого \RussianAbbreviation{ПО}, что дало понимание того, какими характеристиками должно обладать \RussianAbbreviation{ПО}, чтобы обеспечить безопасность пользовательских данных и быть отказоустойчивым к атакам злоумышленника. 
%
Подробно рассмотрены существующие на данный момент методы тестирования безопасности \RussianAbbreviation{ПО}. 
%
Это дало понимание того, каким образом проводится оценка безопасности \RussianAbbreviation{ПО} в целом и поиск уязвимостей безопасности в частности. 

\Paragraph{Рассмотрение проекта PeerHood}
%
Следующий шаг для достижения цели данного исследования заключался в изучении концепции \IT{PeerHood}. 
%
Были рассмотрены цели, требования и архитектура данного \RussianAbbreviation{ПО}. 
%
Данный шаг входит в практическую часть исследования и необходим для дальнейшей оценки безопасности данного проекта. 

\Paragraph{Анализ безопасности PeerHood}
%
Наконец, последний шаг содержит его практическую часть. 
%
На данном этапе был выполнен анализ безопасности \IT{PeerHood} с помощью методов тестирования, основанных на рисках и на свойствах, динамического и статического анализа данного проекта и ревизии его исходного кода. 
%
Результаты анализа показали, что \IT{PeerHood} содержит уязвимости, влияющие на целостность и конфиденциальность данных, передаваемых между устройствами, а также на доступность корректного функционирования данного \RussianAbbreviation{ПО}. 
%
Поэтому его нельзя отнести к категории защищённого и отказоустойчивого к атакам злоумышленника \RussianAbbreviation{ПО}. 

\Paragraph{Анализ полученных результатов}
%
Таким образом можно сказать, что выбранный автором подход привёл к достижению поставленной цели данного исследования. 
%
Но стоит отметить, что в процессе анализа безопасности \IT{PeerHood} происходил поиск только известных и наиболее часто встречающихся уязвимостей безопасностей \RussianAbbreviation{ПО}. 
%
Поэтому для проведения более точной оценки безопасности \IT{PeerHood} необходимо проведения более тщательного и глубокого анализа его безопасности. 
%
Но и этом случае нельзя добиться 100-\% результата из-за специфики тестирования безопасности \RussianAbbreviation{ПО}, так как уязвимость может быть не найдена, но всё же присутствовать в тестируемом \RussianAbbreviation{ПО}. 

% -------------------------------------------------------------------------------------------------------------------------

\SubSectionTitle{Направление дальнейшего исследования}{conclusions_further_work}

\Paragraph{Предложения по практическому внедрению результатов исследования}
%
Результаты исследования будут использованы в дальнейшей разработке \IT{PeerHood}. 
%
Одно из направлений в дальнейшей разработке проекта связано с реализацией в нём механизма безопасности.
%
Поэтому можно сказать, что данное исследование является первым шагом в этом направлении. 
%
Оно также может рассматриваться в качестве первого шага на пути к учёту безопасности в дальнейшей разработке \IT{PeerHood}.

\Paragraph{Причины необходимости дальнейшего исследования}
%
Несмотря на практическую пользу внедрения полученных результатов, стоит принять во внимание их недостаточность в процессе обеспечения безопасности пользовательских данных. 
%
Это объясняется тем, что помимо обеспечения безопасности \RussianAbbreviation{ПО} существуют и другие аспекты обеспечения информационной безопасности. 
%
Стоит также отметить, что обеспечение безопасности \RussianAbbreviation{ПО} осуществляется не только засчёт проведения тестирования безопасности \RussianAbbreviation{ПО}, но и засчёт других мероприятий, проводимых на всех этапах жизненного цикла разработки \RussianAbbreviation{ПО}. 

\Paragraph{Рекомендации по перспективе исследования в данной области}
%
В настоящее время существует тенденция непрерывного роста количества обнаруженных уязвимостей безопасности \RussianAbbreviation{ПО}. 
%
В связи с этим усложняется процесс обеспечения безопасности \RussianAbbreviation{ПО}. 
%
В то же время отмечается постепенный рост популярности использования \RussianAbbreviation{ПДУ} среди пользователей. 
%
Данные факты говорят о необходимости дальнейших исследований в области обеспечения безопасности \RussianAbbreviation{ПО} в мобильном пиринговом окружении. 