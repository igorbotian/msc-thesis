\Citation{Устаревают не только ответы, но и вопросы}{Эрнест Хемингуэй}
%
\Paragraph{Чему посвящена данная работа}
%
Данная работа посвящена проблеме безопасности пользовательских данных при проведении персональных коммуникаций в мобильной пиринговой среде. 
%
Одним из аспектов обеспечения их безопасности является безопасность \RussianAbbreviation{ПО}, функционирующего на мобильных устройствах. 
%
В данной работе рассматривалось \RussianAbbreviation{ПО} \IT{PeerHood}, направленное на обеспечение персональных коммуникаций в независимости от нижележащей сетевой технологии. 
%
Цель данной работы заключалась в оценке безопасности данного проекта. 
%
Для её достижения были выполнены следующие действия. 

\Paragraph{Изучение проблемы безопасности ПО}
%
\TODO{TODO}
%
\begin{comment}
понятие безопасности ПО
факторы безопасности ПО
угрозы безопасности ПО
атаки безопасности 
дефекты безопасности
уязвимости безопасности
угрозы безопасности в мобильном и пиринговом окружении
\end{comment}

\Paragraph{Изучение методов тестирования безопасности ПО}
%
\TODO{TODO}
%
\begin{comment}
понятие безопасного ПО
как проводится анализ безопасности ПО
методы белого, серого и чёрного ящика
\end{comment}

\Paragraph{Рассмотрение проекта PeerHood}
%
\TODO{TODO}
%
\begin{comment}
концепция
цели
требования
архитектура
проблема безопасности в данном проекте
\end{comment}

\Paragraph{Анализ безопасности PeerHood}
%
Результаты анализа безопасности \IT{PeerHood} показали, что данный проект содержит уязвимости, влияющие на целостность и конфиденциальность данных, передаваемых между устройствами, а также на доступность корректного функционирования данного \RussianAbbreviation{ПО}. 
%
Поэтому \IT{PeerHood} нельзя отнести к категории защищённого и отказоустойчивого к атакам злоумышленника \RussianAbbreviation{ПО}. 
%
\begin{comment}
постановка целей и задач
анализ спецификации
моделирование рисков
тестирование безопасности
\end{comment}

\Paragraph{Анализ полученных результатов}
%
\TODO{TODO}
%
\begin{comment}


===

Преимущества предлагаемого варианта решения или предмета рассмотрения
Нерешённые задачи 
Небольшая доля критики насчёт полученных выводов и результатов (Оценка соответствия теории и измерений)
Подтвердилась ли гипотеза или нет, или требует модификации
\end{comment}

% -------------------------------------------------------------------------------------------------------------------------

\SubSubSectionTitle{Направление дальнейшего исследования}{conclusions_further_work}

\Paragraph{Предложения по практическому внедрению результатов исследования}
%
Результаты исследования будут использованы в дальнейшей разработке \IT{PeerHood}. 
%
Одно из направлений в дальнейшей разработке проекта связано с реализацией в нём механизма безопасности.
%
Поэтому можно сказать, что данное исследование является первым шагом в этом направлении. 
%
Оно также рассматриваться в качестве первого шага на пути к учёту безопасности в дальнейшей разработке \IT{PeerHood}.

\Paragraph{Причины необходимости дальнейшего исследования}
%
Несмотря на практическую пользу внедрения полученных результатов, стоит принять во внимание их недостаточность в процессе обеспечения безопасности пользовательских данных. 
%
Это объясняется тем, что помимо обеспечения безопасности \RussianAbbreviation{ПО} существуют и другие аспекты обеспечения информационной безопасности. 
%
Стоит также отметить, что обеспечение безопасности \RussianAbbreviation{ПО} осуществляется не только засчёт проведения тестирования безопасности \RussianAbbreviation{ПО}, но и засчёт других мероприятий, проводимых на всех этапах жизненного цикла разработки \RussianAbbreviation{ПО}. 

\Paragraph{Рекомендации по перспективе исследования в данной области}
%
В настоящее время существует тенденция непрерывного роста количества обнаруженных уязвимостей безопасности \RussianAbbreviation{ПО}. 
%
В связи с этим усложняется процесс обеспечения безопасности \RussianAbbreviation{ПО}. 
%
В то же время отмечается постепенное увеличение популярности использования \RussianAbbreviation{ПДУ} среди пользователей. 
%
Данные факты говорят о необходимости дальнейших исследований в области обеспечения безопасности \RussianAbbreviation{ПО} в мобильном пиринговом окружении. 