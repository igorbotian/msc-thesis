\Citation{Not only answers become obsolete, but even questions}{Ernest Hemingway}
%
\Paragraph{What does the work cover}
%
This work covers the security problem of user data utilized during carrying out personal communications in mobile peer-to-peer environment. 
%
One of the aspects of it's security assurance is security of software operating on mobile devices. 
%
The PeerHood software was examined in this work. 
%
It is indended to carry out personal communications regardless of underlying network technologies. 
%
The goal of the work was to assess the security of this project. 

\Paragraph{Study of software security problem}
%
The first step to reach the goal was to study the software security problem in mobile peer-to-peer environment. 
%
The concept, factors and threats of software security were examined. 
%
Known software security defects and vulnerabilities were considered in detail. 
%
Finally, security threats both typical for mobile and peer-to-peer environments were studied. 
%
The results of this study favoured to realize that the software security problem is and by which factors it is influenced in the context of mobile peer-to-peer environment. 

\Paragraph{Study of software security testing methods}
%
The second step was to perform the following activities. 
%
The idea of secure software was studied. 
%
It helped to understand which characteristics a software software should have to ensure security of user data and to be a fault-tolerant software. 

\Paragraph{PeerHood examination}
%
The next step to reach the goal of the research was to study the PeerHood concept. 
%
The goals, requirements and architecture of the software were examined. 
%
The step was on of the practical parts of the research and was required to further PeerHood security analysis. 

\Paragraph{PeerHood security analysis}
%
Finally, the last step provided the main practical part of the research. 
%
The PeerHood analysis was made by performing of risk-based and property-based software security methods, dynamic and static analysis of the project, source code revision as well. 
%
The results of the analysis showed that PeerHood has security vulnerabilitie influenced the integrity and confidentiality of data transferred between devices and the availability proper operating of the project. 
%
Therefore PeerHood cannot be treated as a secure and fault-tolerant software. 

\Paragraph{Analysis of gained results}
%
In this way, it can be said that the chosen approach resulted in the achievement of the goal of the work. 
%
However, only well-known and most frequent software security vulnerabilities were searching in PeerHood during making the analysis. 
%
Therefore, it is necessary to make a more thorough security analysis of the project to get more accurate results. 
%
But even in this case it is impossible to achieve 100-\% result because of specific character of software security testing. 
%
A vulnerability can exist in a project but could be not found by the testing. 

% -------------------------------------------------------------------------------------------------------------------------

\SubSectionTitle{Further work}{conclusions_further_work}

\Paragraph{Suggestions of practical integration of gained results of the research}
%
The results of the research are planned to be used in further development of PeerHood. 
%
One of the further development directions is related to implementation of a security mechanism. 
%
In this way, it can be said that the research is the first step towards this direction. 
%
Also, it can be considered as the first step towards taking into account of secrurity in further development of PeerHood. 

\Paragraph{Reasons of further research}
%
In spite of practical benefits of integration of the gained results it should be taken into consideration of their insufficiency. 
%
This is because of existence of other aspects of information security assurance apart from software security assurance. 
%
Finally, software security assurance is accomplished not only by software security testing but also by other activities carried out on all stages of software development lifecycle. 

\Paragraph{Recommendations of perspective of further research in this area}
%
Currently there is a tendency of permanent growth of a number of discovered software security vulnerabilities. 
%
Hence, software security assurance process becomes more sophisticated. 
%
At the same time the gradual growth of popularity of personal trusted device usage is noted. 
%
In this way, there is the urgent need for further researches in the area of software security assurance in mobile peer-to-peer environment. 