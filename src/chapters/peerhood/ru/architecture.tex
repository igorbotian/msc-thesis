\Paragraph{Вводный параграф}
\Sentence
С архитектурной точки зрения, \PeerHood состоит из нескольких компонентов: демона, программной 
библиотеки и набора сетевых плагинов.
\Sentence
Они представлены на \ReferenceToFigure{peerhood_components}. 
\Sentence
В совокупности компоненты составляют промежуточное \RussianAbbreviation{ПО}, предоставляющее 
возможность коммуникаций в пиринговой мобильной среде в независимости от используемой сетевой 
технологии.

\ReproducedImageFigure{Компоненты PeerHood}{peerhood_components}{Kolehmainen2010}

\Paragraph{Компоненты PeerHood}{}
\Sentence
Демон \PeerHood является независимым процессом, который отвечает непосредственно за коммуникацию 
между устройствами с помощью конкретной сетевой технологии.
\Sentence
Поддержка демоном новых сетевых технологий осуществляется за счёт использования сторонних 
расширений.
\Sentence
Программная библиотека \PeerHood является компонентом, который отвечает за взаимодействие с 
пользовательскими приложениями посредством \Abbreviation{API} и выполняет роль связующего звена 
между приложениями и демоном.
\Sentence
Таким способом для пользовательских приложений инкапсулируется процесс коммуникации, и достигается 
её прозрачность.

\SubSubSectionTitle{Демон}{peerhood_architecture_daemon}
\Paragraph{Демон PeerHood}
\Sentence
Демон \PeerHood работает в качестве фонового процесса, выполняющего операции по обнаружению 
устройств и сервисов в пиринговом окружении.
\Sentence
Эти операции является довольно дорогостоящими и потребляют много ресурсов, поэтому они выполняются 
в отдельном компоненте.
\Sentence
В этом случае на момент запуска пользовательского приложения нет необходимости собирать 
информацию об окружении, поэтому снижается время, необходимое приложению для полноценной работы.
\Sentence
Данный компонент имеет такое название, так как работает в фоне и полностью соответствует 
\EnglishText{POSIX}-понятию \Emphasis{демон} \Reference{POSIX2008}.

\Paragraph{Локальные сервисы, взаимодействие с демоном}
\Sentence
В свою очередь, демон не только собирает информацию о других устройствах и сервисах в окружении.
\Sentence
Он обеспечивает работу локальных сервисов и предоставляет возможность их использования другими 
устройствами.
\Sentence
Для этого он отправляет других устройствам информацию о локальных сервисах, которые функционируют 
на данном устройстве.
\Sentence
Передача данных, а также регистрация и дерегистрация локальных сервисов пользовательскими 
приложениями осуществляется посредством программной библиотеки, которая является ещё одним 
компонентом \PeerHood.

\SubSubSectionTitle{Программная библиотека}{peerhood_architecture_library}
\Paragraph{Взаимодействие с пользовательскими приложениями}
\Sentence
Данный компонент выполняет роль связующего звена между пользовательскими приложениями с демоном.
\Sentence
Он представляет из себя динамически подключаемую библиотеку, которая напрямую используется 
пользовательскими приложениями.
\Sentence
Взаимодействие с приложениями происходит посредством специально спроектированного 
\Abbreviation{API}, представленного в \ReferenceToAppendix{appendix_peerhood_api}.
\Sentence
Для приложений библиотека является средством для получения данных о других устройствах в окружении 
\PeerHood, средством использования доступных на них удалённых сервисов, а также для возможности 
предоставления услуг локальных сервисов.
\Sentence
Взаимодействие между демоном и программной библиотекой происходит с использованием интерфейса на 
основе \Emphasis{сокета} \Reference{Stevens2004}.

\SubSubSectionTitle{Сетевые расширения}{peerhood_architecture_plugins}
\Paragraph{Описание}
\Sentence
Подсистема сетевых расширения создана для упрощения добавления поддержки новых сетевых технологий и 
последующего их использования.
\Sentence
Каждое расширение выполняет операции, которые являются специфическими для конкретной сетевой 
технологии.
\Sentence
Но в то же время расширения обязано иметь строго определённый \Abbreviation{API}, используемый 
демоном.
\Sentence
Фактически, расширения представляют из себя динамически подключаемые библиотеки, которые загружаются 
демоном в процессе работы \PeerHood.
\Sentence
Текущая реализация проекта имеет поддержку следующих сетевых технологий: \Bluetooth 
\WebSite{Bluetooth}, \Abbreviation{WLAN} \WebSite{WLAN} и \Abbreviation{GPRS} \WebSite{GPRS}.

\SubSubSectionTitle{Пользовательские приложения}{peerhood_architecture_applications}
\Paragraph{Описание}
\Sentence
В архитектуре \PeerHood пользовательские приложения находятся на самом высоком уровне.
\Sentence
Это означает, что операции, специфические для сетевых технологий, выполняются 
компонентами, которые находятся на более низких уровнях.
\Sentence
Таким образом достигается прозрачность взаимодействия на уровне приложений, и они являются 
независимыми от используемой в конкретный момент времени сетевой технологии.

\Paragraph{Модель взаимодействия}
\Sentence
Для взаимодействия между пользовательскими приложениями используется модель "клиент - сервер".
\Sentence
То есть какое-либо приложение может как использовать сервис, предоставляемый другим приложением, 
так и предоставлять свой сервис для других приложений. 
\Sentence
При этом взаимодействующие приложения физически могут быть запущены как на одном устройстве, так и 
на разных. 
\Sentence
В одно и то же время приложение может как предоставлять сервис, так и использовать свой.
\Sentence
Наконец, приложение может следить за присутствием какого-либо устройства в окружении \PeerHood.
\Sentence
Когда это устройство появляется в окружении или покидает его, приложение получает уведомление.
