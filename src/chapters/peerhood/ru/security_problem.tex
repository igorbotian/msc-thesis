\Paragraph{Описание проблемы безопасности в целом}
%
В наше время проблема безопасности информации стоит чрезвычайно остро.
%
Особенно актуально это звучит в контексте \RussianAbbreviation{ПДУ}, когда личные данные пользователя задействованы в процессе коммуникации.
%
Поэтому необходимо учитывать риски безопасности, связанные с нарушением \Emphasis{целостности}, \Emphasis{конфиденциальности} и \Emphasis{доступности} этих данных.

\Paragraph{Актуальность вопроса безопасности для PeerHood}
%
В разработке проекта \IT{PeerHood} направление, связанное с обеспечением безопасности, является одним из \Emphasis{наиболее приоритетных} \Reference{PeerHoodSpecification}. 
%
Основными причинами этого являются следующие:
\begin{enumerate}

	\item PeerHood имеет полный доступ к личной информации, хранящейся на устройстве пользователя.

	\item В процессе коммуникации PeerHood передаёт данные пользователя, в том числе личные.

	\item PeerHood является сетевым приложением, поэтому безопасность передаваемых им данных также зависит от их безопасности сетевой технологии, которую он использует в конкретный момент времени.

	\item Безопасность личных данных пользователя также зависит от безопасности самого ПО PeerHood.
\end{enumerate}

\Paragraph{Необходимость проведения анализа проекта}
%
Создание механизма безопасности для \IT{PeerHood} является одной из \Emphasis{наиболее важных} задач \Reference{PeerHoodSpecification}.
%
Первым шагом на пути к её реализации является анализ проекта на безопасность для выяснения того, какие существуют угрозы безопасности, какие места в проекте являются наиболее уязвимыми и т.д..
%
Этому вопросу посвящён следующий раздел данной диссертации.