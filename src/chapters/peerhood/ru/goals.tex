\Paragraph{Цели}
\Sentence
Причиной разработки программного проекта является необходимость решения определённого круга проблем.
\Sentence
В качестве критерия возможности их решения выступает условие выполнения всех целей, которые 
ставятся перед проектом.
\Sentence
Согласно спецификации проекта \PeerHood, его разработка направлена на достижение следующих целей 
\Reference{PeerHoodSpecification}: 
\begin{description}
	\leftskip2em%
	\setlength{\itemsep}{0pt}%
	\setlength{\parsep}{0pt}%
	
	\item[Proactivity] Упреждающее обнаружение сервисов на мобильных устройствах.
	
	\item[Connectivity] Предоставление взаимодействия между пользовательскими приложениями, 
		которое является прозрачным относительно используемой сетевой технологии.
	
	\item[Reactivity] Уведомление приложений о событиях, связанных с пользовательскими сервисами, 
		сетевыми соединениями или устройствами в окружении.
\end{description}

\Paragraph{Шаги по достижению данных целей}
\Sentence
Для достижения поставленных целей необходимо выполнение ряда задач.
\Sentence
К ним относятся следующие \Reference{PorrasEnhancing2005}:
\begin{itemize}
	\item Предоставление коммуникационной среды, в которой устройства взаимодействуют 
	с использованием пирингового подхода.
	\Sentence
	Это позволяет устройствам осуществлять коммуникационное взаимодействие без использования 
	централизованных серверов.
	
	\item Создание программной библиотеки, которая бы позволила использование какой-либо 
	сетевой технологии посредством унифицированного интерфейса.
	\Sentence
	Это приводит к упрощению разработки пользовательских сетевых приложений за 
	счёт их полной независимости от используемой ими в конкретный времени сетевой технологии.
\end{itemize}

\Paragraph{Функциональные требования}
\Sentence
Для корректной работы пользовательского приложения необходимо, чтобы \PeerHood имел такие 
функциональные характеристики, которые в совокупности обеспечивали бы поддержку 
\Emphasis{\EnglishText{proactivity}}, \Emphasis{\EnglishText{connectivity}} и 
\Emphasis{\EnglishText{reactivity}}.
\Sentence
К такими характеристикам, условиям и возможностями относятся требования к проекту 
\Reference{Krutchen2000}.
\Sentence
Из них функциональными являются следующие \Reference{PeerHoodSpecification} 
\Reference{PorrasComparison2005}:
\begin{description}
	\leftskip2em%
	\setlength{\itemsep}{0pt}%
	\setlength{\parsep}{0pt}%

	\item[Device discovery] System must be able to discovery other PeerHood capable devices 
		within range and the same device neighborhood.
		\Sentence
		Device detection can be depend used network technology.
		\item[Service discovery] System must be able to discovery services from the local device 
		and other PeerHood devices in the device neighborhood.
		\Sentence
		System must have capability to read service attributes as well with service discovery.

	\item[Service sharing] PeerHood must provide mechanism to register services and use them by 
	applications or middleware components. 
	\Sentence
	Services can locate on local or remote device. 
	\Sentence
	The PeerHood system must advertise registered services to other devices in a PeerHood 
	neighborhood.

	\item[Connection establishment] PeerHood must provide ability of connect to one or more other 
	PeerHood device in a PeerHood neighborhood. 
	\Sentence
	Connect establishment must be transparent for used underlying network technology.

	\item[Active monitoring of a device] PeerHood must provide way to set a selected device in the 
	PeerHood neighborhood under active monitoring. 
	\Sentence
	In the active monitoring state, a PeerHood client is notified when the device under monitoring 
	is out of range or when it comes back in the range. 
	\Sentence
	Proper response time and range are network technology dependent attributes.

	\item[Data transmission between devices] PeerHood must provide data transmission between 
	connected PeerHood devices. 
	\Sentence
	PeerHood should not take care of data being transferred. 
	\Sentence
	User of the PeerHood must take care of data endianess and word length of data.

	\item[Seamless connectivity] PeerHood should provide way to change used active network 
	technology automatically if established connection weakens or breaks. 
	\Sentence
	PeerHood should provide always the best possible connections for the user. 
	\Sentence
	Established connection should be possible to monitoring for detecting connection changes, which 
	might cause change of used network technology.
\end{description}

\Paragraph{Новое функциональное требование}
Позже было добавлено дополнительное требование (в текущей реализации \PeerHood оно пока 
не реализовано) \Reference{Kolehmainen2010}:
\begin{description}
	\leftskip2em%
	\setlength{\itemsep}{0pt}%
	\setlength{\parsep}{0pt}%	
	\item[User control] PeerHood could provide ability to control the following PeerHood 
		functionalities: whether PeerHood is active, a list of provided services, a list of 
		accepted services.
\end{description}

\Paragraph{Нефункциональные требования}
\Sentence
Помимо функциональных требований к проекту \PeerHood существует также ряд нефункциональных.
\Sentence
К таким относятся следующие \Reference{PeerHoodSpecification}:
\begin{description}
	\leftskip2em%
	\setlength{\itemsep}{0pt}%
	\setlength{\parsep}{0pt}%	
	\item[Network management] PeerHood should be able to manage a specific network and 
	events from the network. 
	\Sentence
	In addition, PeerHood should check availability of network and get notifications of changes 
	of the network.

	\item[Component management] PeerHood should provide events to PeerHood client of 
	changes and suspensions of discovering functionalities. 
	\Sentence
	The PeerHood operates on mobile devices where memory and power consumptions have to take 
	care. 
	\Sentence
	Due to that, used device environment is dynamic. 
	\Sentence
	As, if network interface might go power saving state or it can be closed for freeing memory 
	to other applications.

	\item[Communication concurrency base] PeerHood must support concurrent execution, 
	in that multiple connections are used and they need to get execution time evenly. 
	\Sentence
	The only exception for use of multiple simultaneous connections is if used network 
	technology limits multiple connections on the hardware level.

	\item[Event interface] PeerHood must provide event interface for be able to notify 
	dynamic changes to PeerHood client and itself.

	\item[Plugin architecture for networks] PeerHood must provide interface for its 
	functionalities to plugins. 
	\Sentence
	Network plugins implements abstractions of connectivity and device monitoring 
	functionalities. 
	\Sentence
	In addition, plugins handles device detection and service sharing.
\end{description}

\Paragraph{Связь требований с архитектурой}
\Sentence
Как уже было упомянуто выше, достижение цели разработки проекта основывается на обеспечении 
требований к нему.
\Sentence
Функциональные требования необходимы для определения функций, которые должно поддерживать 
разрабатываемое \RussianAbbreviation{ПО}, а также для определения его будущих компонентов.
\Sentence
Тогда как разработка нефункциональных требований в первую очередь определяет архитектура 
разрабатываемого \RussianAbbreviation{ПО} \Reference{Stellman2005}.
\Sentence
Архитектура проекта PeerHood рассматривается в следующем подразделе.