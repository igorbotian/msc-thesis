\Paragraph{Цели}
%
Причиной разработки программного проекта является необходимость решения определённого круга проблем.
%
В качестве критерия возможности их решения выступает условие выполнения всех целей, которые 
ставятся перед проектом.
%
Согласно спецификации проекта \PeerHood, его разработка направлена на достижение следующих целей 
\Reference{PeerHoodSpecification}: 
\begin{description}
	\leftskip2em%
	\setlength{\itemsep}{0pt}%
	\setlength{\parsep}{0pt}%
	
	\item[\EnglishText{Proactivity}] Упреждающее обнаружение сервисов на мобильных устройствах.
	
	\item[\EnglishText{Connectivity}] Предоставление взаимодействия между пользовательскими 
		приложениями, которое является прозрачным относительно используемой сетевой технологии.
	
	\item[\EnglishText{Reactivity}] Уведомление приложений о событиях, связанных 
		с пользовательскими сервисами, сетевыми соединениями или устройствами в окружении.
\end{description}

\Paragraph{Шаги по достижению данных целей}
%
Для достижения поставленных целей необходимо выполнение ряда задач.
%
К ним относятся следующие \Reference{PorrasEnhancing2005}:
\begin{itemize}
	\item Предоставление коммуникационной среды, в которой устройства взаимодействуют 
	с использованием пирингового подхода.
	%
	Это позволяет устройствам осуществлять коммуникационное взаимодействие без использования 
	централизованных серверов.
	
	\item Создание программной библиотеки, которая бы позволила использование какой-либо 
	сетевой технологии посредством унифицированного интерфейса.
	%
	Это приводит к упрощению разработки пользовательских сетевых приложений за 
	счёт их полной независимости от используемой ими в конкретный времени сетевой технологии.
\end{itemize}

\Paragraph{Функциональные требования}
%
Для корректной работы пользовательского приложения необходимо, чтобы \PeerHood имел такие 
функциональные характеристики, которые в совокупности обеспечивали бы поддержку 
\Emphasis{\EnglishText{proactivity}}, \Emphasis{\EnglishText{connectivity}} и 
\Emphasis{\EnglishText{reactivity}}.
%
К такими характеристикам, условиям и возможностями относятся требования к проекту 
\Reference{Krutchen2000}.
%
Из них функциональными являются следующие \Reference{PeerHoodSpecification} 
\Reference{PorrasComparison2005}:
\begin{description}
	\leftskip2em%
	\setlength{\itemsep}{0pt}%
	\setlength{\parsep}{0pt}%

	\item[\EnglishText{Device discovery}] \EnglishText{System must be able to discovery other 
		PeerHood capable devices within range and the same device neighborhood.
		%
		Device detection can be depend used network technology.
		\item[Service discovery] System must be able to discovery services from the local device 
		and other PeerHood devices in the device neighborhood.
		%
		System must have capability to read service attributes as well with service discovery.}

	\item[\EnglishText{Service sharing}] \EnglishText{PeerHood must provide mechanism to register 
		services and use them by applications or middleware components. 
		%
		Services can locate on local or remote device. 
		%
		The PeerHood system must advertise registered services to other devices in a PeerHood 
		neighborhood.}

	\item[\EnglishText{Connection establishment}] \EnglishText{PeerHood must provide ability of 
		connect to one or more other PeerHood device in a PeerHood neighborhood. 
		%
		Connect establishment must be transparent for used underlying network technology.}

	\item[\EnglishText{Active monitoring of a device}] \EnglishText{PeerHood must provide way to 
		set a selected device in the PeerHood neighborhood under active monitoring. 
		%
		In the active monitoring state, a PeerHood client is notified when the device under 
		monitoring is out of range or when it comes back in the range. 
		%
		Proper response time and range are network technology dependent attributes.}

	\item[\EnglishText{Data transmission between devices}] \EnglishText{PeerHood must provide data 
		transmission between connected PeerHood devices. 
		%
		PeerHood should not take care of data being transferred. 
		%
		User of the PeerHood must take care of data endianess and word length of data.}

	\item[\EnglishText{Seamless connectivity}] \EnglishText{PeerHood should provide way to change 
		used active network technology automatically if established connection weakens or breaks. 
		%
		PeerHood should provide always the best possible connections for the user. 
		%
		Established connection should be possible to monitoring for detecting connection changes, 
		which might cause change of used network technology.}
\end{description}

\Paragraph{Новое функциональное требование}
Позже было добавлено дополнительное требование (в текущей реализации \PeerHood оно пока 
не реализовано) \Reference{Kolehmainen2010}:
\begin{description}
	\leftskip2em%
	\setlength{\itemsep}{0pt}%
	\setlength{\parsep}{0pt}%	
	\item[\EnglishText{User control}] \EnglishText{PeerHood could provide ability to control 
		the following PeerHood functionalities: whether PeerHood is active, a list of provided 
		services, a list of accepted services.}
\end{description}

\Paragraph{Нефункциональные требования}
%
Помимо функциональных требований к проекту \PeerHood существует также ряд нефункциональных.
%
К таким относятся следующие \Reference{PeerHoodSpecification}:
\begin{description}
	\leftskip2em%
	\setlength{\itemsep}{0pt}%
	\setlength{\parsep}{0pt}%	
	\item[\EnglishText{Network management}] \EnglishText{PeerHood should be able to manage 
		a specific network and events from the network. 
		%
		In addition, PeerHood should check availability of network and get notifications of changes 
		of the network.}

	\item[\EnglishText{Component management}] \EnglishText{PeerHood should provide events to 
		PeerHood client of changes and suspensions of discovering functionalities. 
		%
		The PeerHood operates on mobile devices where memory and power consumptions have to take 
		care. 
		%
		Due to that, used device environment is dynamic. 
		%
		As, if network interface might go power saving state or it can be closed for freeing memory 
		to other applications.}

	\item[\EnglishText{Communication concurrency base}] \EnglishText{PeerHood must support 
		concurrent execution, in that multiple connections are used and they need to get execution 
		time evenly. 
		%
		The only exception for use of multiple simultaneous connections is if used network 
		technology limits multiple connections on the hardware level.}

	\item[\EnglishText{Event interface}] \EnglishText{PeerHood must provide event interface for 
		be able to notify dynamic changes to PeerHood client and itself.}

	\item[\EnglishText{Plugin architecture for networks}] \EnglishText{PeerHood must provide 
		interface for its functionalities to plugins. 
		%
		Network plugins implements abstractions of connectivity and device monitoring 
		functionalities. 
		%
		In addition, plugins handles device detection and service sharing.}
\end{description}

\Paragraph{Связь требований с архитектурой}
%
Как уже было упомянуто выше, достижение цели разработки проекта основывается на обеспечении 
требований к нему.
%
Функциональные требования необходимы для определения функций, которые должно поддерживать 
разрабатываемое \RussianAbbreviation{ПО}, а также для определения его будущих компонентов.
%
Тогда как разработка нефункциональных требований в первую очередь определяет архитектура 
разрабатываемого \RussianAbbreviation{ПО} \Reference{Stellman2005}.
%
Архитектура проекта PeerHood рассматривается в следующем подразделе.