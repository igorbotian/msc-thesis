\Paragraph{Общая информация}
%
\IT{PeerHood} разработан для выполнения \Emphasis{персональных коммуникаций} в независимости от технологии, используемой в сетевом окружении.
%
Так как он направлен на \RussianAbbreviation{ПДУ}, то предполагается его использование преимущественно в \Emphasis{мобильных сетях}.
%
Поэтому \IT{PeerHood} может рассматриваться как \Emphasis{окрестность мобильного пирингового окружения} 
\Reference{PorrasP2P2004}.

\Paragraph{Концепция промежуточного ПО}
%
\IT{PeerHood} представляет из себя \Emphasis{промежуточное сетевое \RussianAbbreviation{ПО}}, отвечающее за прозрачность использования пользовательских сервисов, как удалённых, так и локальных.
%
На более высоком уровне относительно \IT{PeerHood} располагаются пользовательские приложения, а на более низком - сетевые расширения, которые непосредственно выполняют операции, специфичные для конкретной сетевой технологии.
%
Схематично данная концепция изображена на \ReferenceToFigure{peerhood_concept}.

\ReproducedImageFigure{Концепция PeerHood}{peerhood_concept}{PorrasP2P2004}

\Paragraph{Функциональность}
%
\IT{PeerHood} осуществляет \Emphasis{слежение за другими устройствами} в сетевой окрестности упреждающим способом с использованием доступных на устройстве сетевых технологий.
%
При этом за \Emphasis{переключение между сетевыми технологиями} также осуществляет \IT{PeerHood}.
%
Также он имеет следующие \Emphasis{функциональные характеристики} \Reference{Kolehmainen2010}:
\begin{itemize}
	\item обнаружение и использование сервисов, доступных на других устройствах
	\item извещение других устройств о доступных локальных сервисах
	\item слежение за статусом других устройств
\end{itemize}

\Paragraph{Информация о разработке проекта}
%
Разработка проекта, основанного на описанной выше концепции, ведётся в лаборатории коммуникационного \RussianAbbreviation{ПО} Лаппеенратского Технологического Университета, Финляндия \Reference{PeerToPeerNeighborhood}, на протяжении нескольких лет.
%
Изначально проект создавался при соучастии компании \EnglishText{Nokia} \WebSite{Nokia}. 
%
Изучался вопрос о социальных коммуникациях с применением \IT{PeerHood} \Reference{Karki2008}.
%
Были разработаны пользовательские приложения для проекта \IT{UMSIC} \WebSite{UMSIC} \Reference{Laakkonen2009}.
%
На настоящий момент \IT{PeerHood} является свободным программным обеспечением и распространяется под лицензией \Term{GPL} версии 2 \WebSite{GPL2} \Reference{PeerHoodGitorious}.

\Paragraph{Текущая реализация}
%
Текущая реализация \IT{PeerHood} может быть использована как на настольных компьютерах, так и на мобильных устройствах. 
%
Это достигается за счёт того, что она основана на фреймворке \IT{Qt} \WebSite{Qt}, который направлен на создание кроссплатформенного \RussianAbbreviation{ПО}.