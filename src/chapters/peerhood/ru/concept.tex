\Paragraph{Общая информация}
\Sentence
\PeerHood разработан для выполнения персональных коммуникаций в независимости от технологии, 
используемой в сетевом окружении.
\Sentence
Так как он направлен на \RussianAbbreviation{ПДУ}, то предполагается его использование 
преимущественно в мобильных сетях.
\Sentence
Поэтому \PeerHood может рассматриваться как \Emphasis{окрестность мобильного пирингового окружения} 
\Reference{PorrasP2P2004}.

\Paragraph{Концепция промежуточного ПО}
\Sentence
\PeerHood представляет из себя промежуточное сетевое \RussianAbbreviation{ПО}, отвечающее за 
прозрачность использования пользовательских сервисов, как удалённых, так и локальных.
\Sentence
На более высоком уровне относительно \PeerHood располагаются пользовательские приложения, а на 
более низком - сетевые расширения, которые непосредственно выполняют операции, специфичные для 
конкретной сетевой технологии.
\Sentence
Схематично данная концепция изображена на \ReferenceToFigure{peerhood_concept}.

\ReproducedImageFigure{Концепция PeerHood}{peerhood_concept}{PorrasP2P2004}

\Paragraph{Функциональность}
\Sentence
\PeerHood осуществляет слежение за другими устройствами в сетевой окрестности упреждающим способом 
с использованием доступных на устройстве сетевых технологий.
\Sentence
При этом за переключение между сетевыми технологиями также осуществляет \PeerHood.
\Sentence
Также он имеет следующие функциональные характеристики \Reference{Kolehmainen2010}:
\begin{itemize}
	\item обнаружение и использование сервисов, доступных на других устройствах
	\item извещение других устройств о доступных локальных сервисах
	\item слежение за статусом других устройств
\end{itemize}

\Paragraph{Информация о разработке проекта}
\Sentence
Разработка проекта, основанного на описанной выше концепции, ведётся в лаборатории 
коммуникационного \RussianAbbreviation{ПО} Лаппеенратского Технологического Университета, 
Финляндия \Reference{PeerToPeerNeighborhood}, на протяжении нескольких лет.
\Sentence
Изначально проект создавался при соучастии компании \Nokia \WebSite{Nokia}. 
\Sentence
Изучался вопрос о социальных коммуникациях с применением \PeerHood \Reference{Karki2008}.
\Sentence
Были разработаны пользовательские приложения для проекта \EnglishText{UMSIC} \WebSite{UMSIC} 
\Reference{Laakkonen2009}.
\Sentence
На настоящий момент \PeerHood является свободным программным обеспечением и распространяется под 
лицензией GPL версии 2 \WebSite{GPL2} \Reference{PeerHoodGitorious}.

\Paragraph{Текущая реализация}
\Sentence
Текущая реализация \PeerHood может быть использована как на настольных компьютерах, так и на 
мобильных устройствах. 
\Sentence
Это достигается за счёт того, что она основана на фреймворке \Qt 
\WebSite{Qt}, который направлен на создание кроссплатформенного \RussianAbbreviation{ПО}.
