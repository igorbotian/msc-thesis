\Paragraph{Description}
%
At the present the problem related to information security is very important. 
%
It is especially actual in the context of \Abbreviation{PTD} when private user data is involved in communication process. 
%
Hence, it is necessary to take into account security risks related to vioalation of integrity, confidentiality, and availability of the data. 


\Paragraph{Urgency of the security problem concerning PeerHood}
%
There is a direction of further PeerHood development that is related to security assurance and is one of the most priority \Reference{PeerHoodSpecification}. 
%
The main reasons of this are the following: 
\begin{enumerate}
	\item PeerHood has access to private user data stored on a device
	\item PeerHood can transfer private data to other devices during communication
	\item PeerHood is a network software, so data transfer security depends on security of underlying network technology
	\item security of private user data also depends on PeerHood software security itself
\end{enumerate}

\Paragraph{Necessity to perform analysis of the project}
%
According the PeerHood specification \Reference{PeerHoodSpecification}, implementation of a security mechanism for PeerHood is one of the most important tasks. 
%
The first step to implement the mechanism is to perform software security analysis of the project to find out, which kind of security vulnerabilities PeerHood has, which places are most susceptible from security point of view, and so on. 
%
The next section of the thesis deals with this problem. 