\Paragraph{General information}
%
The aim of PeerHood is to carry out personal communications between devices in a network neighbourhood regardless of underylying network technology. 
%
Because it is intended for a \Abbreviation{PTD}, it is supposed to use the software mainly in mobile environments. 
%
Therefore the PeerHood concept can be considered as mobile \Abbreviation{P2P} neighbourhood \Reference{PorrasP2P2004}. 

\Paragraph{The concept of PeerHood middleware}
%
PeerHood is middleware responsible for providing transparency of unitilization of user services, both remote and local. 
%
User applications are located on a higher level in regard to PeerHood. 
%
Network plug-ins are located on a lower level in regard to PeerHood. 
%
They perform operations specific to a network technology directly. 
%
The concept is schematically represented on \OnReferenceToFigure{peerhood_concept}. 

\ReproducedImageFigure{The PeerHood concept}{peerhood_concept}{PorrasP2P2004}

\Paragraph{Functionality}
%
PeerHood carries out tracking of other devices in network neighborhood in proactive manner switching between network technologies supported by a given mobile device. 
%
Also, is has the following function characteristics \Reference{Kolehmainen2010}: 
\begin{itemize}
	\item detection of other devices using different network technologies
	\item discovery of services from other devices
	\item advertising own local services to other devices
	\item monitoring status of devices in network neighborhood
\end{itemize}

\Paragraph{Information about development of the project}
%
\T{The} development of a project based on the concept described above has been being conducting in Communications Software Laboratory of Lappeenranta University of Technology for several years. 
%
Initially the project was being developed with the participation of Nokia company \WebSite{Nokia}. 
%
A problem related to performing of social communications using PeerHood was studied \Reference{Karki2008}. 
%
User applications for UMSIC project \WebSite{UMSIC} were implemented \Reference{Laakkonen2009}. 
%
At this moment PeerHood is a free software, it is distributed under GPL 2 license \WebSite{GPL2} \WebSite{PeerHoodGitorious}. 

\Paragraph{Current implementation}
%
Current PeerHood implementation can be utilized both on desktops and mobile devices. 
%
It is obtained by the fact that it's components are based on Qt framework \WebSite{Qt}, which is intended to develop \A cross-platform software. 