\Paragraph{A list of PeerHood components}
%
From architectural point of view PeerHood is comprised of the following components: daemon, shared library and a set of network plug-ins. 
%
They are represented on \OnReferenceToFigure{peerhood_components}. 
%
The components together are \T{the} middleware that provides communications in mobile \Abbreviation{P2P} environment regardless of underlying network technology. 

\ReproducedImageFigure{The PeerHood components}{peerhood_components}{Kolehmainen2010}

\Paragraph{PeerHood components}
%
PeerHood daemon is an independent process which is directly responsible for providing communication between devices using a given network technology. 
%
It is accomplished by utilization of network plug-ins. 
%
PeerHood library is an component which is responsible for interaction with user applications via specified \Abbreviation{API} and have a role to play as glue between daemon and applications. 
%
That is the way a process of communication between devices is encapsulated for applications. 

% ------------------------------------------------------------------------------------------------

\SubSubSectionTitle{Демон}{peerhood_architecture_daemon}

% ------------------------------------------------------------------------------------------------

\SubSubSectionTitle{Программная библиотека}{peerhood_architecture_library}

% ------------------------------------------------------------------------------------------------

\SubSubSectionTitle{Сетевые расширения}{peerhood_architecture_plugins}

% ------------------------------------------------------------------------------------------------

\SubSubSectionTitle{Пользовательские приложения}{peerhood_architecture_applications}