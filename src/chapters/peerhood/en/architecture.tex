\Paragraph{A list of PeerHood components}
%
From architectural point of view PeerHood is comprised of the following components: daemon, shared library and a set of network plug-ins. 
%
They are represented on \OnReferenceToFigure{peerhood_components}. 
%
The components together are \T{the} middleware that provides communications in mobile \Abbreviation{P2P} environment regardless of underlying network technology. 

\ReproducedImageFigure{The PeerHood components}{peerhood_components}{Kolehmainen2010}

\Paragraph{PeerHood components}
%
PeerHood daemon is an independent process which is directly responsible for providing communication between devices using a given network technology. 
%
It is accomplished by utilization of network plug-ins. 
%
PeerHood library is an component which is responsible for interaction with user applications via specified \Abbreviation{API} and have a role to play as glue between daemon and applications. 
%
That is the way a process of communication between devices is encapsulated for applications. 

% ------------------------------------------------------------------------------------------------

\SubSubSectionTitle{Daemon}{peerhood_architecture_daemon}

\Paragraph{Description}
%
Daemon is a background process that is intended for discovering other devices and services in neighborhood. 
%
The operation is quite costly and consumes a lot of resources and hence is implemented as a separate component. 
%
In this case during initializing of an application it is not necessary to gather information about neighborhood, so time needed for an application to start to work is reduced. 
%
The component has the such name because works in background and fully corresponds to the POSIX-term daemon \Reference{POSIX2008}. 

\Paragraph{Local services and their interaction with daemon}
%
By-turn, the daemon also gather information about other devices and services in neighborhood. 
%
It provides operation of local services and gives possibility to use them by other devices. 
%
It sends information about local services to other devices in neighborhood. 
%
Data transmission, registration and deregistration of local services by applications as well are realized by shared library, which is an another component of PeerHood. 

% ------------------------------------------------------------------------------------------------

\SubSubSectionTitle{Library}{peerhood_architecture_library}

\Paragraph{Description}
%
The component has a role to play as glue between applications and daemon. 
%
It is a dynamic link library that is directly used by applications. 
%
Interaction with them is performing via specially designed \Abbreviation{API} presented in \ReferenceToAppendix{appendix_peerhood_api}. 
%
For applications the library is a medium that used to get information about other devices in neighborhood, to do data exchange with them, to utilize remote services, and to make local services available for other devices. 
%
Interaction of the library with daemon is performing via socket-based interface \Reference{Stevens2004}. 

% ------------------------------------------------------------------------------------------------

\SubSubSectionTitle{Network plug-ins}{peerhood_architecture_plugins}

\Paragraph{Description}
%
Network plug-in subsystem is intended for making support of new network technologies easier. 
%
Each plug-in performs operations specific for a given network technology. 
%
Also, it should have specially designed \Abbreviation{API} used by the daemon. 
%
Factically, plug-ins are dynamic link libraries which are loaded by daemon during PeerHood operating. 
%
Current PeerHood implementation has support of \T{the} following network technologies: Bluetooth \WebSite{Bluetooth}, \Abbreviation{WLAN} \WebSite{WLAN}, and \Abbreviation{GPRS} \WebSite{GPRS}. 

% ------------------------------------------------------------------------------------------------

\SubSubSectionTitle{User applications}{peerhood_architecture_applications}

\Paragraph{Description}
%
Applications are located on the top of PeerHood architecture. 
%
It means that operations specific for a given network technology are performed by components located on lower layers. 
%
In such \The way transparency of interaction between applications is accomplished. 

\Paragraph{Interaction model}
%
To provide such interaction ``client -- server'' model is used. 
%
That is an application can both utilize a service provided by other device and provide own service for utilizing by other applications. 
%
Interacting applications, meanwhile, can be physically situated on both one or different devices. 
%
At the same time an application can both provide a service and utilize own one. 
%
Finally, an application can perform tracking of other devices in neighborhood. 
%
When a device appeares in neighborhood or leaves it the application gets a notice about it. 