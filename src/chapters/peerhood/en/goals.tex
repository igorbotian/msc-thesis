\Paragraph{Goals}
%
The motivation to develop this project is a need to solve a variety of problems. 
%
According to PeerHood specification, it's development is intended to achieve the following goals \Reference{PeerHoodSpecification}: 
\begin{description}
	\leftskip2em%
	\setlength{\itemsep}{0pt}%
	\setlength{\parsep}{0pt}%
	
	\item[Proactivity] Proactive service discovery on mobile devices.
	
	\item[Connectivity] To provide transparent communication between application regardless of underlying network technology. 
	
	\item[Reactivity] To provide information events related to services, connections or device resources. 
\end{description}

\Paragraph{Steps to achieve the goals}
%
To achieve the goals listed above it is necessary to solve a set of tasks. 
%
They are the following \Reference{PorrasEnhancing2005}: 
%
\begin{itemize}
	
	\item To provide a communication environment where communication between devices is based on \Abbreviation{P2P} approach. 
	%
	It allows devices to communicate without utilization of central servers
	
	\item To provide a special software library to allow applications use a network technology through unified interface. 
	%
	It leads to facilitation of development of new applications because they does not depend on underlying network technologies. 
\end{itemize}

\Paragraph{Functional requirements}
%
It is necessary for PeerHood to have such functional characteristics which enable support of proactivity, connectivity, and reactivity. 
%
Supporting them by PeerHood leads to proper operating of user applications. 
%
A set of requirements to the project can be referred as these characteristics \Reference{Krutchen2000}. 
%
A set of functional requirements to the PeerHood project are listed below \Reference{PeerHoodSpecification} \Reference{PorrasComparison2005}: 
\begin{description}
	\leftskip2em%
	\setlength{\itemsep}{0pt}%
	\setlength{\parsep}{0pt}%

	\item[Device discovery] System must be able to discovery other PeerHood capable devices within range and the same device neighborhood.
	%
	Device detection can depend on used network technology.
		
	\item[Service discovery] System must be able to discover services from the local device and other PeerHood devices in the device neighborhood.
	%
	System must have capability to read service attributes as well with service discovery.

	\item[Service sharing] PeerHood must provide mechanism to register services and use them by applications or middleware components. 
	%
	Services can locate services working on a local or a remote device. 
	%
	The PeerHood system must advertise registered services to other devices in a PeerHood neighborhood.

	\item[Connection establishment] PeerHood must provide ability of connect to one or more other PeerHood devices in a PeerHood neighborhood. 
	%
	Connect establishment must be transparent for using underlying network technology.

	\item[Active monitoring of a device] PeerHood must provide way to set a selected device in a neighborhood under active monitoring. 
	%
	In the active monitoring state, a PeerHood client is notified when the device under monitoring is out of range or when it comes back in the range. 
	%
	Proper response time and range are network technology dependent attributes.

	\item[Data transmission between devices] PeerHood must provide data transmission between connected PeerHood devices. 
	%
	PeerHood should not take care of data being transferred. 
	%
	PeerHood user must take care of data endianess and word length of data.

	\item[Seamless connectivity] PeerHood should provide way to change used active network technology automatically if established connection weakens or breaks. 
	%
	PeerHood should always provide the best possible connections for a user. 
	%
	Established connection should be possible to monitoring for detecting connection changes, which might cause change of used network technology.
\end{description}

\Paragraph{A new functional requirement}
%
A new additional functional requirement was added to the specification (but it is not supported by the current implementation of PeerHood) \Reference{Kolehmainen2010}: 
\begin{description}
	\leftskip2em%
	\setlength{\itemsep}{0pt}%
	\setlength{\parsep}{0pt}%	
	
	\item[User control] PeerHood could provide ability to control the following PeerHood functionalities: whether PeerHood is active, a list of provided services, a list of accepted services.
\end{description}

\Paragraph{Non-functional requirements}
%
Also, there is a set of non-functional requirements to the PeerHood project. 
%
They are listed below \Reference{PeerHoodSpecification}: 
\begin{description}
	\leftskip2em%
	\setlength{\itemsep}{0pt}%
	\setlength{\parsep}{0pt}%	
	
	\item[Network management] PeerHood should be able to manage a specific network and events from the network. 
	%
	In addition, PeerHood should check availability of network and get notifications of changes of the network.

	\item[Component management] PeerHood should provide events to PeerHood client of changes and suspensions of discovering functionalities. 
	%
	PeerHood operates on mobile devices where memory and power consumptions have to take care. 
	%
	Due to that, used device environment is dynamic. 
	%
	As, if network interface might go power saving state or it can be closed for freeing memory to other applications.

	\item[Communication concurrency base] PeerHood must support concurrent execution, in that multiple connections are used and they need to get execution time evenly. 
	%
	The only exception for use of multiple simultaneous connections is if used network technology limits multiple connections on the hardware level.

	\item[Event interface] PeerHood must provide event interface for be able to notify dynamic changes to PeerHood client and itself.

	\item[Plugin architecture for networks] PeerHood must provide interface for its functionalities to plugins. 
	%
	Network plugins implements abstractions of connectivity and device monitoring functionalities. 
	%
	In addition, plugins handles device detection and service sharing.
\end{description}

\Paragraph{Connection between requirements and architecture}
%
As it was noted above, all these requirements should be met to achieve the goals of PeerHood. 
%
Functional requirements are required to define a list of features which should be supported by \T{a} developing software, and a list of it's future components as well. 
%
Whereas development of non-functional requirements in the first place defines \T{an} architecture of \T{a} developing software \Reference{Stellman2005}. 
%
The architecture of PeerHood is considered in the next subsection. 