\Paragraph{Goals}
\TODO{Goals}


\Paragraph{Key requirements}
\TODO{Key requiremenets}
\begin{description}
	\item[Device discovery]. System must be able to discovery other PeerHood capable devices 
		within range and the same device neighborhood.
		%
		Device detection can be depend used network technology.

		\item[Service discovery]. System must be able to discovery services from the local device 
		and other PeerHood devices in the device neighborhood.
		%
		System must have capability to read service attributes as well with service discovery.

		\item[Service sharing]. PeerHood must provide mechanism to register services and use them 
		by applications or middleware components. 
		%
		Services can locate on local or remote device. 
		%
		The PeerHood system must advertise registered services to other devices in a PeerHood 
		neighborhood.

		\item[Connection establishment]. PeerHood must provide ability of connect to one or more 
		other PeerHood device in a PeerHood neighborhood. 
		%
		Connect establishment must be transparent for used underlying network technology.

		\item[Active monitoring of a device]. PeerHood must provide way to set a selected device 
		in the PeerHood neighborhood under active monitoring. 
		%
		In the active monitoring state, a PeerHood client is notified when the device under 
		monitoring is out of range or when it comes back in the range. 
		%
		Proper response time and range are network technology dependent attributes.

		\item[Data transmission between devices]. PeerHood must provide data transmission between 
		connected PeerHood devices. 
		%
		PeerHood should not take care of data being transferred. 
		%
		User of the PeerHood must take care of data endianess and word length of data.

		\item[Seamless connectivity]. PeerHood should provide way to change used active network 
		technology automatically if established connection weakens or breaks. 
		%
		PeerHood should provide always the best possible connections for the user. 
		%
		Established connection should be possible to monitoring for detecting connection changes, 
		which might cause change of used network technology.

		\item[Network management]. PeerHood should be able to manage a specific network and events 
		from the network. 
		%
		In addition, PeerHood should check availability of network and get notifications of changes 
		of the network.

		\item[Component management]. PeerHood should provide events to PeerHood client of changes 
		and suspensions of discovering functionalities. 
		%
		The PeerHood operates on mobile devices where memory and power consumptions have to take 
		care. 
		%
		Due to that, used device environment is dynamic. 
		%
		As, if network interface might go power saving state or it can be closed for freeing memory 
		to other applications.

		\item[Communication concurrency base]. PeerHood must support concurrent execution, in that 
		multiple connections are used and they need to get execution time evenly. 
		%
		The only exception for use of multiple simultaneous connections is if used network 
		technology limits multiple connections on the hardware level.

		\item[Event interface]. PeerHood must provide event interface for be able to notify 
		dynamic changes to PeerHood client and itself.

		\item[Plugin architecture for networks]. PeerHood must provide interface for its 
		functionalities to plugins. 
		%
		Network plugins implements abstractions of connectivity and device monitoring 
		functionalities. 
		%
		In addition, plugins handles device detection and service sharing.

		\item[User control]. PeerHood could provide ability to control PeerHood functionalities
		\begin{enumerate}
			\item Is PeerHood active
			\item What services are provided
			\item What services are accepted
		\end{enumerate}

		This is a new requirement and that is not yet implemented in the existing PeerHood 
		implementation.
\end{description}