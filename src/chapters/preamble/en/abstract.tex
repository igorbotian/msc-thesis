\Paragraph{Description of the problem}
%
Increase of computational power and emergence of new computer technologies led to popularity of local communications between personal trusted devices. 
%
By-turn, it led to emergence of security problems related to user data utilized in such communications. 
%
One of the main aspects of \T{the} security assurance is security of software operating on mobile devices. 

\Paragraph{What is studied in this work}
%
The aim of this work was to analyze security threats of PeerHood, software intended for performing personal communications between mobile devices regardless of underlying network technologies. 
%
To reach this goal risk-based software security testing was carried out. 
%
The results of the testing showed that the project has security vulnerabilities. 
%
Therefore PeerHood cannot be considered as a secure software. 
%
The analysis made in the work is the first step towards the further implementation of PeerHood security mechanism, as well as taking into account the security in the development process of this project. 