\Paragraph{Description of the problem}
%
\The increase of computational power and emergence of new computer technologies have been favoured a popularity of carrying out local communications between personal trusted devices. 
%
This has led to security problems of user data utilized in a process of such communications. 
%
One of the main aspects of the security assurance is security of software operating on mobile devices. 

\Paragraph{What is studied in this work}
%
The aim of this work was to analyze security threats of PeerHood, software intended to perform personal communications between mobile devices regardless of underlying network technologies. 
%
To reach this goal risk-based software security testing was carried out. 
%
The results of the testing showed that the project has security vulnerabilities. 
%
Therefore PeerHood cannot be considered as a secure software. 
%
The analysis made in the work is the first step towards the further implementation of security mechanism of PeerHood, taking into account the security in the development process of this project as well. 