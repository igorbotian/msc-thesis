\PreambleSectionTitle{\AbstractTitle}
%
\ThesisSchool
%
%\vspace*{1em}
%
\begin{doublespace}
\ThesisAuthor

\textbf{\ThesisTitle}
\\
\ThesisSubject
\\
\CurrentYear
\\
\TotalNumberOfPages \TotalPagesWord, 6 \FiguresWord, 3 \TablesWord~\AndWord~6 \AppendicesWord
%
\begin{tabbing}
\ExaminersWord:\quad\= \ThesisFirstExaminer\\
\> \ThesisSupervisor
\end{tabbing}
%
\KeywordsWord: \ThesisKeywords
\end{doublespace}
\IncludePreambleSectionContents{abstract}
%
\begin{comment}
FINAL THESIS INSTRUCTIONS.

The abstract is a concise (one A4 sheet), objective, independent summary
of the Master’s thesis. It should be intelligible as such, without the original document. 

It explains the contents of the thesis: the objective, methodologies, results and conclusions. 
A good abstract is written in complete and concise sentences. 
The author does not express his or her opinions, but describes the thesis as would an outside reporter. 
No direct references are made to the original text. 

The abstract is a public document, and therefore all confidential information must be excluded
from it. The abstract is prepared in Finnish and English. 
Both the Finnish and English abstracts are included in the thesis. 

The abstracts are also submitted to the faculty study affairs services as an annex to the assessment 
application of the thesis.
Foreign nationals do not need to prepare an abstract in Finnish.

The author sends electronic copies of the abstracts or the entire thesis to the LUT library. 
More details are available from the library and its web site.

=== 

An abstract is prepared on all Master's theses. You should favour the passive voice or the 3rd
person active in case the abstract is published separately. Unestablished abbreviations, symbols or
technical terms should be explained. Tables, equations etc. are used only if they are necessary for
the sake of clarity. No direct references are made to the original text.

The abstract is done in both Finnish and English (equivalent contents). In the Finnish abstract, the
title is in Finnish and in the English one in English. Foreign students do not need to prepare an
abstract in Finnish.

The complete identification information should be included in the beginning of both the Finnish and
the English abstract.

Author’s name
Title of thesis
Faculty
Degree programme and/or major subject
Year of completion
Master’s Thesis University
Number of pages, figures, tables and appendices
Examiners (1st and 2nd)
Keywords in Finnish
Keywords in English

The keywords must be informative and describe the contents of the thesis accurately. 
Concrete concepts (e.g. equipment) are in plural, abstract ones (e.g. methods) in singular. 
A good title should include at least some of the most important keywords.
The number of keywordsshould be three to five.

INSTRUCTIONS FOR WRITING A MASTER'S THESIS

Present the core items of your thesis, background, goal(s), results, and the main conclusions.
Favour short sentences. The text in the thesis may have sentences and structures of various styles. 
There is one abstract page and the abstract text is one paragraph, not two or many. 
Present the goal(s) of your work, present also the main constraints and delimitations, 
if some profound constraints are present. 
Main results and main conclusions are given such that the reader gets highly interested in the work
done. Use passive form instead of active in the presentation, such as “this study concentrates on
digital communications systems”, not as “I studied communications”. 
Alignment is justified (both left and right). Page numbering starts from the cover page but the page
number in Roman is not printed.
Page numbering with Arabic numbers start from the page with the table of contents.

WRITING THE THESIS

This is the synopsis of your thesis.  It should state your hypothesis, your methods (how you 
went about testing the hypothesis), a brief summary of your findings, and a brief conclusion.  
This is the LAST thing that you write.  Wait until everything else is written before attempting the 
abstract. 

HOW TO WRITE A THESIS

Abstract — of approximately 300 words. (It should not exceed 700 words.) The Abstract or summary
should summarize the appropriate headings, aims, scope and conclusion of the thesis.

Examiners will look here to find out whether it is new knowledge; and if so what.

Try condensing your thesis in:
- one word
- one line
- one sentence
- one paragraph
- one page
- one chapter

ОФОРМЛЕНИЕ ВКР

Аннотация должна отражать суть выполненной работы, содержать перечень используемых методов 
исследования и полученных результатов; включать сведения об особенностях разработки, ее 
возможностях и предполагаемых областях применения.

===

Рябченко:
- описание существующей проблемы
- что предлагается в качестве решения данной проблемы
- на чём основан предлагаемый метод
- что получается в результате

Никандрова:
- описание существующего решения проблемы
- оно содержит недостаток

- цель диссертации - решение данного недостатка (предложение более лучшего решения)
- краткое описание в чём оно заключается
- что показали эксперименты
- одно предложение, что он проанализирован и найден ряд недостатков

Строкина:
- описание существующей проблемы. есть возможность что-то там делать
- цель данной диссертации - анализ и проверка того, а можно ли это сделать с помощью чего-то там
- после проведения экспериментов с использованием данного метода что-то там получено и изучено

Лаакконен:
- что такое проекты UMSIC и PeerHood, зачем они нужны
- в работе рассматривается что-то там, связанное с ними и направленное на что-то
- что было проведено, получены такие-то результаты, какие выводы получены
- выводы, сделанные по всей диссертации (что реальна дала данная работа)

Колехмайнен:
- что такое концепция PeerHood
- есть одна реализация, цель реализовать с помощью Qt, цель переписки, что это даст
- оценка качества полученной реализации - что оценили и что в итоге: лучше или хуже

Карки:
- что такое концепция social networking, что даст её связка с мобильным окружением
- что содержит работа - описание sn, мобильное окружение и PeerHood
- представлена реализация sn-приложения на основе peerhood

Калвар:
- в работе что-то там концептуализировано. следующие вопросы - причина этого. они и выбранная модель дают оценку
- что составляет эмпирическую часть работы. из чего она состоит. какие методы оценки используются. что каждый из них даёт
- какие результаты получены и как они повлияют на концептуализацию. их анализ показал, что они должны быть изучены ещё более глубоко

Яппинен:
- описание существующей проблемы. нужны новые подходы к её решению
- что рассматривается в данном дипломе: методы. как называется предлагаемый подход. его основные характеристики и преимущества

- работа рассматривает два разных, но взаимодополняющих аспекта. что даёт каждый из них

- как они могут быть применены в разных окружениях. рассмотрен простой примерный сервис
\end{comment}