\Paragraph{Описание проблемы}
%
Рост вычислительных мощностей и появление новых технологий способствовали популярности проведения локальных взаимодействий между персональными мобильными устройствами. 
%
Это привело к появлению проблемы безопасности пользовательских данных, используемых в процессе таких взаимодействий. 
%
Одним из основных аспектов её обеспечения является безопасность программного обеспечения, используемого на мобильных устройствах. 

\Paragraph{Что исследуется в данной работе}
%
\Remark{Middleware}
%
Цель данной работы заключается в анализе угроз безопасности \EnglishText{PeerHood}, программного обеспечения, направленного на выполнение персональных коммуникаций между мобильными устройствами в независимости от нижележащей сетевой технологии. 
%
Для её достижения проводится тестирование безопасности программного обеспечения, основанное на рисках. 
%
Его результаты показывают, что проект содержит уязвимости безопасности.
%
Поэтому \EnglishText{PeerHood} нельзя считать безопасным программным обеспечением. 
%
Исследование, проведённое в данной работе, является первым шагом к дальнейшей реализации в нём механизма безопасности, а также к введению учёта безопасности в процессе разработки данного проекта.

\begin{comment}
- описание проблемы (предпосылки)
- что такое peerhood
- (недостаток заключается в безопасности)
- цель работы (гипотеза)
- используемый метод для достижения результата (задачи, необходимые для решения)
- (ограничения)
- что получилось в результате
- вывод по полученным результатам
- дальнейшие направления в разработке

предпосылки - рост популярности мобильных взаимодействий с пиринговой концепцией, концепция PTD, использование пользовательских данных в процессе коммуникации, их безопасность
цель работы (вопрос исследования) - выяснить, является ли peerhood защищённым ПО
основная задача - анализ безопасности peerhood
задачи - изучить проблему безопасности ПО, рассмотреть уязвимости безопасности ПО, оценить влияние мобильности окружения и концепции пирингового взаимодействия, изучить существующие методы анализа безопасности ПО, произвести поиск уязвимостей peerhood с помощью тестирования рисков безопасности ПО
дальнейшие направления - первый шаг на пути к реализации механизмов безопасности peerhood
ограничения - практическая часть лежит в области тестирования ПО, так как peerhood почти завершён, только поиск уязвимостей (ничего с ними не делается), упор сделан на язык с++ и технологию qt
полученные результаты - уязвимости найдены
выводы - peerhood не является незащищённым ПО, поэтому его использование за пределами лаборатории пока преждевременно. требуется их устранение
гипотеза - является ли peerhood защищённым ПО
используемые методы тестирования - тестирование безопасности ПО, основанное на рисках

===

суть выполненной работы, перечень используемых методов исследования, особенности разработки, её возможности, предполагаемые области применения
содержит ключевые характеристи, предпосылки, вопрос и цели, основные ограничения, полученные результаты и основные выводы
формулирует гипотезу, используемые методы проверки (тестирования), кратко резюме по тому, что найдено, и краткий вывод
краткий обзор работы
краткий итог по всем заголовкам, цели, область и выводы по работе
обычно 300 слов (но не более 700)
\end{comment}