\newcommand{\PeerHoodAnalysisFolderName}{peerhood_analysis}
%
\SectionTitle{\PeerHoodAnalysisTitle}{peerhood_analysis}
\TODO{PeerHood analysis}
%
\begin{comment}
HOW TO ORGANIZE YOUR THESIS

This part of the thesis is much more free-form. It may have one or several sections and subsections.
But it all has only one purpose: to convince the examiners that you answered the question 
or solved the problem that you set for yourself in Section 4. 
So show what you did that is relevant to answering the question or solving the problem: 
if there were blind alleys and dead ends, do not include these, unless specifically relevant 
to the demonstration that you answered the thesis question.

HOW TO WRITE A THESIS

Each of these should preferably be self-contained and clearly focused. Think of the story
you want to tell. Choose and present only those results that are relevant to your hypothesis.
A morass of experimental results unilluminated by a hypothesis and unembellished by a
discussion is insulting and confusing to your reader.

The sections in your chapter should follow the experimental schema set out in Figure 2.
State your hypothesis clearly. Indicate all assumptions. Include enough information
about materials and methods to enable another suitably qualified person to repeat your
experiments. Relegate tedious but necessary details to an Appendix, so that there are no
breaks in the flow of ideas in your presentation.

(Assumptions, Hypothesis, Methods, Materials) => Experiment -> Results -> Analysis => (Discussion,
Conclusions)

Do not present results chronologically; present them logically.

The Discussion section of your experimental chapter is where you add value to your
work. This is where you comment on your results. Why are they what they are? What
meaning can you wrest from them? Are they in accord with accepted theory? What do
they mean with respect to your hypothesis? Do your results uphold your assumptions?
How do you treat unexpected or inconsistent results? Can you account for them? Do
your results suggest that you need to revise your experiments or repeat them? Do they
indicate a revised hypothesis? What are the limitations in your methodology? How do
your results fit in with the work of others in the field? What additional work can you
suggest?

Throughout your thesis, and especially in your experimental chapters, there should be
no gaps in the flow of logic. Keep the links of a chain in mind. Each link is connected to
two other links: one before and one after. Absence of any one link is a weakness. Absence
of both means there is no chain!
\end{comment}