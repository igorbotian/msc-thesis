\Paragraph{Какие уязвимости безопасности содержит PeerHood}
%
Анализ безопасности \EnglishText{PeerHood} показал, что проект содержит уязвимости безопасности. 
%
К ним относятся следующие:
\begin{itemize}
	%\leftskip2em%
	\setlength{\itemsep}{0pt}%
	%\setlength{\parsep}{0pt}%

	\item отсутствие установки безопасных прав доступа к файлу конфигурации
	\item отсутствие механизмов безопасности
	\item передача данных между устройствами в незашифрованном виде
	\item передача пары значений ''название хоста -- идентификатор устройства'' в процессе коммуникации
	\item нарушение инкапсуляции классов \EnglishText{PeerHood} \Abbreviation{API}
	\item отсутствие проверки данных в точках входа приложения
\end{itemize}

\Paragraph{Нарушаются ли функциональные характеристики PeerHood}
%
Присутствие в \EnglishText{PeerHood} перечисленных выше уязвимостей безопасности влияет на его функционирование. 
%
Это может привести к нарушению следующих функциональных характеристик \EnglishText{PeerHood}:
\begin{itemize}
	%\leftskip2em%
	\setlength{\itemsep}{0pt}%
	%\setlength{\parsep}{0pt}%

	\item возможность передачи данных по сети
	\item конфиденциальность и целостность данных, передаваемых по сети

	\item доступность, конфиденциальность и целостность актуальной информации об устройствах в сетевом окружении
	\item возможность обмена пользовательских данных и данных об устройствах в сетевом окружении между устройствами

	\item доступность, конфиденциальность и целостность актуальной информации о доступных сервисах в сетевом окружении
	\item возможность обмена информации о доступных сервисах в сетевом окружении между устройствами

	\item возможность передачи пользовательских данных между устройствами
	\item конфиденциальность и целостность пользовательских данных, передаваемых между устройствами
\end{itemize}

\Paragraph{Нарушение принципов безопасности и требований к PeerHood}
%
Нарушение перечисленных выше функциональных характеристик приводит к нарушению всех составляющих безопасности \EnglishText{PeerHood}: целостности, доступности и конфиденциальности. В свою очередь, это приводит к угрозе нарушения функциональных требований к EnglishText{PeerHood}.

\Paragraph{Нарушение целей PeerHood}
%
Таким образом, угроза нарушения функциональных требований \EnglishText{PeerHood} приводит к угрозе необеспечения целей данного проекта. 
%
А это, в свою очередь, негативным образом влияет на его отказоустойчивость. 
%
Поэтому можно сделать вывод о том, что \EnglishText{PeerHood} не является незащищённым \RussianAbbreviation{ПО}. 