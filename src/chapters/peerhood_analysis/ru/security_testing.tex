\Paragraph{Доступность исходного кода PeerHood}
%
Набор используемых методов тестирования безопасности \RussianAbbreviation{ПО} зависит от доступности исходного кода тестируемого \RussianAbbreviation{ПО} и его сторонних компонентов.
%
\PeerHood является открытым \RussianAbbreviation{ПО}, поэтому его исходный код доступен \Reference{PeerHoodGitorious} (для тестирования используется версия \Italic{81eeb64} от \FourDigitsDate{2011}{08}{11}). 
%
Поэтому необходимо провести выбор необходимых методов тестирования из числа тех, которые рассматриваются в \InReferenceToSection{software_security_analysis}.
%
По причине доступности исходного кода в первую очередь встаёт вопрос о необходимости тестирования по методу ''чёрного'' ящика. 

% -------------------------------------------------------------------------------------------------------------------------

\SubSubSectionTitle{Выбор методов тестирования}{peerhood_analysis_security_testing_method_choosing}

\Paragraph{Fuzz-тестирование}
%
\Fuzz-тестирование по своей природе довольно затратно, а результаты его проведения можно получить более лёгким способом, а именно при помощи ревизии участков кода, представляющих собой точки входа приложения.
%
Для \PeerHood это ''уязвимые'' места, перечисленные в \InReferenceToSection{peerhood_analysis_risk_modeling}. 

\Paragraph{Реверс-инжиниринг}
%
Проведение реверс-инжиниринга является излишним по причине доступности исходного кода, а закрытые сторонние компоненты не используются. 
%
\PeerHood в качестве платформы использует фреймворк \Qt \Reference{Qt}, поэтому можно сказать, что его безопасность зависит от безопасности данной платформы. 

\Paragraph{Отладка по методу чёрного ящика}
%
Проведение отладки по методу чёрного ящика также излишне по тем же причинам, что и проведение реверс-инжиниринга. 
%
Более всесторонний анализ того, как приложение взаимодействует с программной средой, можно провести с помощью тестирования по методу ''белого'' ящика. 
%

\Paragraph{Сканеры уязвимости}
%
Использование сканеров уязвимостей излишне по той же причине, что и проведение \fuzz-тестирования. 
%
Более качественные результаты можно получить путём ревизии участков кода, представляющих собой ''уязвимые'' места \PeerHood. 

\Paragraph{Внедрение неисправностей в исходный код}
%
Внедрение неисправностей в исходный код также излишне по причине доступности исходного кода \PeerHood. 
%
Вместо него можно провести ревизию исходного кода с целью анализа механизма обработки ошибок и проверки принципа ''безопасность в глубине''. 
%
\PeerHood не является большим проектом. 

\Paragraph{Внедрение неисправностей}
%
Напротив, внедрение неисправностей в ресурсы и компоненты приложения может быть эффективным. 
%
С помощью него можно оценить, насколько \PeerHood является отказоустойчивым к неисправностям среды (ресурсов, сторонних компонентов, программного окружения). 

\Paragraph{Тестирование внедрения}
%
Проведение тестирования внедрения направлено на оценку отказоустойчивости приложения. 
%
Оценка рисков, анализ ''уязвимых'' мест и поиск уязвимостей составляют множество мероприятий, которые используются при тестировании внедрения. 
%
Поэтому можно сказать, что процесс анализа безопасности \PeerHood подразумевает проведение тестирования внедрения. 

\Paragraph{Тестирование, основанное на свойствах}
%
То же самое можно сказать про проведение тестирования, основанного на свойствах. 
%
Тестирование безопасности \PeerHood направлено на оценку таких характеристик, как составляющие \CIATriad применительно к \PeerHood. Ниже приведён развёрнутый список данных характеристик:
\begin{itemize}
	%\leftskip2em%
	\setlength{\itemsep}{0pt}%
	%\setlength{\parsep}{0pt}%

	\item возможность передачи данных по сети
	\item конфиденциальность и целостность данных, передаваемых по сети

	\item доступность, конфиденциальность и целостность актуальной информации об устройствах в сетевом окружении
	\item возможность обмена пользовательских данных и данных об устройствах в сетевом окружении между устройствами
	\item возможность одновременной поддержки нескольких клиентов

	\item доступность, конфиденциальность и целостность актуальной информации о доступных сервисах в сетевом окружении
	\item возможность обмена информации о доступных сервисах в сетевом окружении между устройствами

	\item возможность передачи пользовательских данных между устройствами
	\item конфиденциальность и целостность пользовательских данных, передаваемых между устройствами
	\item возможность контроля работы системы
\end{itemize}

\Paragraph{Статический и динамический анализ кода}
%
Проведение статического и динамического анализа направлено на поиск уязвимостей соответственно на уровне исходного кода и времени выполнения. 
%
Использование разнообразных инструментальных средств способствует более детальному анализу \PeerHood. 
%
Но применение данных методов тестирования безопасности не позволяет провести всесторонний анализ приложения. 
%
Поэтому можно сказать, что их проведение обосновано и не является излишним.

\Paragraph{Ревизия кода}
%
С помощью автоматизированных методов тестирования безопасности можно найти лишь ограниченное количество уязвимостей. 
%
Поэтому для выполнения всестороннего анализа безопасности \PeerHood необходимо проведение ревизии исходного кода. 
%
Причём, как уже было упомянуто выше, особенно это актуально для ''уязвимых'' мест приложения. 

% -------------------------------------------------------------------------------------------------------------------------

\SubSubSectionTitle{Проведение статического анализа}{peerhood_analysis_security_testing_static_analysis}

\Paragraph{Цель проведения статического анализа}
%
Целью проведения статического анализа является поиск уязвимостей безопасности уровня исходного кода (см. \ReferenceToSection{software_security_analysis_blackbox_testing_techniques_static_code_analysis}). 
%
Оно заключается в использовании автоматических инструментальных средств, направленных на поиск ошибок программирования, влияющих на безопасность приложения. 
%
Для проведения статического анализа \PeerHood используются следующие инструменты: \Term{Cppcheck}, \Term{RATS}, средства проверки компилятора \Term{GCC}. 

\Paragraph{GCC}
%
Компилятор \Term{GCC} позволяет осуществлять поиск уязвимостей форматных строк, проверку границ массивов, обнаружение неинициализированных данных и других ошибок на этапе компиляции \WebSite{GCC}. 
%
Для этого при сборке проекта вызов компилятора происходит с передачей специального аргумента \SourceCode{-Wall}, включающего выполнение указанных выше действий. 
%
Компиляция \PeerHood происходит успешно и с отсутствием каких-либо предупреждений. 
%
Это говорит о том, что с использованием данного инструментального средства уязвимости безопасности не найдены. 

\Paragraph{Cppcheck}
%
\Term{Cppcheck} направлен на поиск ошибок следующего вида: выход за пределы границ массивов, безопасность исключений, утечка памяти, неинициализированные значения переменных и т.п. \WebSite{Cppcheck}. 
%
Использование данного инструментального средства ошибок в \PeerHood не выявило (см. \ReferenceToAppendix{appendix_cppcheck_output}). 
%
Приводится лишь несколько рекомендаций, но с безопасностью они не связаны. 

\Paragraph{RATS}
%
\Term{RATS} направлен на поиск узявимостей переполнения буфера, состояния гонки и других видов \WebSifte{RATS}. 
%
Данное инструментальное средство приводит несколько предостережений безопасности, но ревизия соответствующих участков кода ошибок не выявила (см. \ReferenceToAppendix{appendix_rats_output}). 
%
Первое из них связано с возможностью переполнения буфера, содержащего данные передаваемого сетевым расширением \EnglishText{ICMP}-пакета. 
%
Его использование выполняется в рамках безопасности. 
%
Второе предостережение связано с возможностью состояния гонки при регистрации нескольких сигналов с помощью функции \SourceCode{signal()}. 
%
Но в приложении регистрируется только один. 
%
Последнее предупреждение связано с опасностью использования функции \SourceCode{gethostbyname()}, результаты которой могут быть подменены извне. 
%
Но за её безопасность ответственно программное окружение \PeerHood (платформа или \RussianAbbreviation{ОС}), а не само приложение.  

\Paragraph{Выводы}
%
Таким образом, проведение статического анализа \PeerHood ошибок безопасности на уровне кода не выявило. 
%
Отчасти это можно объяснить тем, что приложение активно использует средства фреймворка \Qt. 

% -------------------------------------------------------------------------------------------------------------------------

\SubSubSectionTitle{Проведение динамического анализа кода}{peerhood_analysis_security_testing_dynamic_analysis}

\Paragraph{Цель проведения динамического анализа}
%
Целью проведения динамического анализа является поиск уязвимостей безопасности времени выполнения (см. \ReferenceToSection{software_security_analysis_gray_box_techniques_dynamic_code_analysis}). 
%
Оно заключается в использовании автоматических средств, направленных на обнаружение аномальных действий, поиск ошибок взаимодействия с программным окружением и других дефектов, влияющих на безопасность. 
%
С помощью его могут быть следующие ошибки времени выполнения: переполнение буфера, уязвимость форматных строк, утечка памяти и другие. 
%
Для проведения динамического анализа \PeerHood используются инструментальных средства \Term{Valgrind} и \Term{Helgrind}. 

\Paragraph{Valgrind}
%
\Term{Valgrind} направлен на поиск ошибок времени выполнения, перечисленных выше, и ошибок, связанных с распределением памяти \WebSite{Valgrind}. 
%
Использование данного инструментального средства каких-либо аномальных действий или ошибок не выявило (см. \ReferenceToAppendix{appendix_valgrind_output}). 
%
Стоит отметить, что им обнаружена потенциальная утечка памяти, но ревизия исходного кода \PeerHood показала, что утечка связана с внутренней реализацией средств \Qt. 


\Paragraph{Helgrind}
%
\Term{Helgrind} направлен на обнаружение ошибок синхронизации, в том числе состояния гонок \WebSite{Helgrind}. 
%
Его использование не выявило ошибок, связанных с неправильным использованием примитивов синхронизации в \PeerHood (см. \ReferenceToAppendix{appendix_helgrind_output}). 
%
В свою очередь им обнаружена потенциальное появление состояния гонки во многих местах приложения. 
%
Но как показал дальнейший анализ, оно связано с внутренней реализацией используемых средств \Qt. 

\Paragraph{Выводы}
%
Таким образом, проведение динамического анализа \PeerHood ошибок времени выполнения не выявило. 
%
Однако обнаружены потенциальные ошибки в \Qt, средства которого используются \PeerHood. 
%
Поэтому можно сделать вывод о том, что безопасность \PeerHood зависит от безопасности данной платформы.  

% -------------------------------------------------------------------------------------------------------------------------

\SubSubSectionTitle{Внедрение неисправностей}{peerhood_analysis_security_testing_fault_injection}

\Paragraph{Цель внедрения неисправностей}
%
Целью внедрения неисправностей является оценка устойчивости приложения в условии некорректного функционирования используемых им компонентов или ресурсов (см. \ReferenceToSection{software_security_analysis_black_box_techniques_fault_injection}). 
%
Оно заключается во внесении неисправности в какой-либо компонент тестируемого приложения, после чего проводится проверка корректности взаимодействия других компонентов с данным. 

\Paragraph{Внесение неисправностей в компоненты PeerHood}
%
\PeerHood состоит из следующих взаимодействующих элементов: сетевые расширения, демон, библиотека общих компонентов и пользовательская библиотека, пользовательское приложение, файл конфигурации (см. \ReferenceToSection{peerhood_analysis_risk_modeling}). 
%
Взаимодействие происходит следующими способами:
\begin{itemize}
	%\leftskip2em%
	\setlength{\itemsep}{0pt}%
	%\setlength{\parsep}{0pt}%

	\item динамическое связывание разделяемых объектов (библиотек) и дальнейшее взаимодействие посредством \Abbreviation{API} (демон -- сетевые расширения, демон -- библиотека общих компонентов, пользовательская библиотека -- библиотека общих компонентов, пользовательское приложение -- пользовательская библиотека)
	\item взаимодействие посредством сетевого сокета (демон -- демон, пользовательская библиотека -- демон)
	\item взаимодействие с использованием средств файловой системы (библиотека общих компонентов -- файл конфигурации)
\end{itemize}

\Paragraph{Анализ взаимодействия посредством API}
%
Внедрение бинарной неисправности в разделяемый объект (библиотеку) может быть двух видов. 
%
Первый из них затрагивает интерфейсную часть данного объекта. 
%
В этом случае происходит ошибка в процессе его динамического связывания, что приводит к отказу его в работе. 
%
Второй же затрагивает лишь изменение внутреннего функционирования данного объекта, что приводит к его некорректному функционированию. 
%
Это может оказать влияение на тестируемый компонент, например, возврат некорректных или даже зловредных данных. 
%
Для оценки отказоустойчивости тестируемого компонента необходимо провести анализ \Abbreviation{API}, который используется для данного взаимодействия. 
%
Так как исходный код \PeerHood доступен, лучшим способом выполнения данного действия является ревизия соответствующих участков кода. 

\Paragraph{Анализ взаимодействия посредством сетевого сокета}
%
Такой же вывод можно сделать относительно компонентов, взаимодействующих посредством сетевого сокета (демон и пользовательская библиотека). 
%
Полученные таким способом данные могут быть некорректными, поэтому необходимо выполнение оценки отказоустойчивости к ним данных компонентов. 
%
При доступности исходного кода \PeerHood наиболее доступным способом выполнения данного действия является ревизия соответствующих участков кода. 

\Paragraph{Анализ взаимодействия с использованием файловой системы}
%
Библиотека общих компонентов содержит средства конфигурирования приложения. 
%
\PeerHood хранит свою конфигурацию в заданном файле специально заданного формата. 
%
Отсутствие данного файла или нарушение его формата приводит к отказу от работы \PeerHood. 
%
В конфигурационном файле содержится путь к файлам сетевых расширений, которые загружаются демоном. 
%
В текущей реализации демона не предусмотрено значение данного ключа по умолчанию, поэтому его отсутствие влечёт за собой невозможность загрузки сетевых расширений и функционирования самого демона. 

\Paragraph{Анализ контроля доступа к файлам PeerHood}
%
Текущая реализация \PeerHood не содержит специального скрипта установки. 
%
В данный момент развёртывание приложения заключается в копировании библиотек, исполняемых модулей и ресурсов в заданный каталог. 
%
Для них не выставляются необходимые с точки зрения безопасности права доступа. 

\Paragraph{Возможность компрометации модулей}
%
Это может привести к компрометации модулей \PeerHood. 
%
Следствием этого является возможность отказа в работе приложения или возможность проведения атаки на передаемые между ними данные. 

\Paragraph{Анализ контроля доступа конфигурационного файла}
%
\PeerHood также использует текстовый файл для хранения в нём собственной конфигурации. 
%
Данный файл также доступен для чтения и записи для других приложений. 
%
Поэтому существует риск внесения в него несанкционированных изменений. 
%
Целью данного действия может быть, например, изменение пути к файлам сетевых расширений с последующей загрузкой демоном зловредных сетевых расширений.

\Paragraph{Выводы}
%
Используя метод внесения неисправностей в ресурсы \PeerHood, была выявлена уязвимость, связанная с его небезопасным развёртыванием. 
%
Её эксплойтирование со стороны злоумышленника приводит к нарушению всех составляющих безопасности \PeerHood (см. \ReferenceToSection{peerhood_analysis_security_testing_method_choosing}).
% -------------------------------------------------------------------------------------------------------------------------

\SubSubSectionTitle{Ревизия кода}{peerhood_analysis_security_testing_code_inspection}

\Paragraph{Цель ревизии кода}
%
TODO
%
\begin{comment}
На что направлена ревизия
Как она будет происходит
Какие результаты необходимо получить
Что они дадут (связь с задачами анализа)

Анализ кода с точки зрения безопасности
Методы тестирования — лишь дополнительный инструмент
Более глубокий поиск уязвимостей
\end{comment}

\Paragraph{Что можно проверить в ходе ревизии кода}
%
TODO
%
\begin{comment}
Язык разработки С++: переполнение буфера, целочисленные данные, форматные строки, указатели
Apple Secure Coding Guide: помимо перечисленных выше ещё проверка входных данных, состояние гонки, IPC, файловая система, проблемы контроля доступа
C Secure Coding Standard: препроцессор, инициализация, целочисленные данные, массивы, строки, распределение памяти, ввод/вывод, временные файлы, окружение, сигналы
Seven Pernicious Kingdoms: проверка входных данных, неправильное использование API, механизм безопасности, проблема синхронизации, неправильная обработка исключительных ситуаций, нарушение инкапсуляции (раскрытие информации), программное окружение
Подведение итогов: выжимка из перечисленных выше списков
\end{comment}

\Paragraph{Что будет проверено в ходе ревизии кода}
%
TODO
%
\begin{comment}
Уязвимости, характерные для С++, будут проверены с помощью ревизии кода
Проверка входных данных, в каких местах: с помощью ревизии кода
Необходима проверка API: ревизия кода
Состояния гонки с помощью инструментальных средств, а примитивы синхронизации с помощью ревизии кода
Чек-лист из C Secure Coding Standard по списку с помощью ревизии кода
Программное окружение: с помощью ревизии кода + анализ Qt
Раскрытие информации: с помощью ревизии кода
Неправильная обработка исключений: ревизия кода

Проверка точек входа приложения
\end{comment}

\Paragraph{Общие проблемы С/C++}
%
TODO
%
\begin{comment}
Переполнение буфера - поиск массивов в коде
Целочисленные данные - поиск целочисленных переменных в коде и их отслеживание
Форматные строки - поиск форматных строк
Указатели - поиск указателей и их отслеживание
\end{comment}

\Paragraph{Входные данные}
%
TODO
%
\begin{comment}
Анализ кода, отвечающего за приём и отправку данных через сокетный интерфейс демона
Анализ API для приложений и API для сетевых интерфейсов
\end{comment}

\Paragraph{Синхронизация}
%
TODO
%
\begin{comment}
Анализ и поиск мест, где нужно синхронизация. Анализ кода
\end{comment}

\Paragraph{Использование API}
%
TODO
%
\begin{comment}
Анализ вызываемых системных API функций и Qt API
\end{comment}

\Paragraph{Механизм безопасности}
%
TODO
%
\begin{comment}
Отсутствует
\end{comment}

\Paragraph{Механизм обработки ошибок}
%
TODO
%
\begin{comment}
Поиск и анализ исключительных ситуаций в коде
\end{comment}

\Paragraph{Раскрытие информации}
%
TODO
%
\begin{comment}
Анализ сообщений об ошибках, ресурсов приложений, характер передачи пользовательских данных
\end{comment}