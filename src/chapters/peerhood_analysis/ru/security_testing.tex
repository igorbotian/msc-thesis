\Paragraph{Анализ факта доступности исходного кода}
%
TODO
%
\begin{comment}
Для тестирования используются методы белого, серого и чёрного ящика
Доступен исходный код всех частей приложения, поэтому некоторые методы серого и чёрного ящика необходимо проанализировать, нужно ли их использовать
Исходный код взят из репозитория Gitorious (срез от 11.08.2011, номер коммита 81eeb64)
\end{comment}

\Paragraph{Неиспользуемые методы тестирования}
%
TODO
%
\begin{comment}
Fuzz-тестирование слишком затратно, так как проще проанализировать участки кода точек входа приложения
Реверс-инжиниринг не нуждается в проведении, так как сторонние компоненты не используются (платформа не считается), а весь исходный код доступен
Отладка по методу чёрного ящика не нуждается в применении, так как нет сторонних составных частей без исходного кода. Анализ того, как приложение взаимодействует с внешней средой, лучше провести с помощью методов белого ящика
Использование сканеров уязвимостей также излишне, так как это можно сделать с помощью анализа точек входа в приложение
Внедрение неисправностей в исходный код не выполняется, так как доступен приложение не большое и менее затратно проанализировать то, как проводится обработка ошибок, а проверку принципа «безопасность на глубине» можно выполнить ревизией кода на уровне интерфейсов классов
\end{comment}

\Paragraph{Что даст каждый метод тестирования, который будет использован}
%
TODO
%
\begin{comment}
Внедрение неисправностей в ресурсы приложения может быть эффективным, так как выступает в качестве симуляции неисправности компонентов приложения, среды, ресурсов (как приложение взаимодействует с программным окружением)
Динамический и статический анализ кода не может найти все уязвимости, а лишь часть, поэтому является лишь дополнительным
Основными методами тестирования являются тестирование внедрение и тестирование, основанное на свойствах
\end{comment}

\Paragraph{Про ревизию кода}
%
TODO
%
\begin{comment}
С помощью методов тестирования можно найти лишь ограниченное количество ошибок
Поэтому также необходимо провести ревизию кода
Приложение небольшое и легко провести его ревизию
К тому же анализ безопасности, представленный в этой главе, с учётом ревизии согласовывается с тестированием, основанным на свойствах, и тестированием внедрения (можно получить одни и те же результаты)

По сути, вся эта глава посвящена тестированию внедрения
Риски безопасности уже проанализированы
Их модель тоже построена (CIA Triad)
В качестве тестирования выбраны другие методы тестирования + ревизия кода
Как такового тестирования, основанного на свойствах, проводится не будет, так как уже проведён анализ спецификации и оно вполне вписывается в план тестирования в общем
\end{comment}

% -------------------------------------------------------------------------------------------------------------------------

\SubSubSectionTitle{Проведение статического анализа}{peerhood_analysis_security_testing_static_analysis}

\Paragraph{Вводные слова про цели тестирования по методу белого ящика}
%
TODO
%
\begin{comment}
На что направлено тестирование: на анализ исходного кода, поиск в нём ошибок безопасности
Как оно будет происходить: анализ исходного кода с помощью инструментальных средств
Что будет использовано: только инструментальные средства
Какие результаты будут получены: информация об ошибках программирования
Что они дадут (связь с задачами тестирования): как найденные ошибки влияют на безопасность, с каким риском связаны, можно ли провести атаку
\end{comment}

\Paragraph{Статический анализ}
%
TODO
%
\begin{comment}
Используется во время разработки приложения
Установить цели ревизии
Использовать инструментальные средства
По отчётам провести ревизию слабых мест (анализ отчёта)
Недостатки: является лишь вспомогательным инструментом, для выявления ошибок времени выполнение применяются другие методы тестироавания
Инструменты: Splint, RATS, ITS4, PVS Studio, GCC (флаги компилятора)

Поиск ошибок программирования с помощью инструментальных средств
Найденные ошибки проанализировать на степень их влияния на безопасность
Уязвимости связать с возможностью атак и с рисками безопасности => влияние на защищённость
Используемые инструменты: RATS, ITS4, PVS Studio, GCC
\end{comment}

% -------------------------------------------------------------------------------------------------------------------------

\SubSubSectionTitle{Проведение динамического анализа кода}{peerhood_analysis_security_testing_dynamic_analysis}

\Paragraph{Вводные слова про цели тестирования по методу серого ящика}
%
TODO
%
\begin{comment}
На что направлено тестирование
Как оно будет происходит
Что будет использовано
Какие результаты будут получены
Что они дадут (связь с задачами тестирования)
\end{comment}

\Paragraph{Динамический анализ кода}
%
TODO
%
\begin{comment}
Обнаружение аномальных действий, 
Направлены на поиск ошибок взаимодействия с пользователем, программным окружением или сторонними компонентами: переполнение буфера, уязвимость форматных строк, указателей
Инструменты: stackguard, libsafe, formatguard, valgrind, helgrind

Запуск приложения под тестами и поиск ошибок выполнения с помощью инструментальных средств
Запуск valgrind
Запуск helgrind
Остальное в рамках ревизии кода, так как используется библиотека Qt (многие опасные средства С/С++  инкапсулированы в функциях Qt)
\end{comment}

% -------------------------------------------------------------------------------------------------------------------------

\SubSubSectionTitle{Внедрение неисправностей}{peerhood_analysis_security_testing_fault_injection}

\Paragraph{Вводные слова про цели тестирования по методу чёрного ящика}
%
TODO
%
\begin{comment}
На что направлено тестирование
Как оно будет происходит
Что будет использовано
Какие результаты будут получены
Что они дадут (связь с задачами тестирования)
\end{comment}

\Paragraph{Внедрение неисправностей в бинарный код}
%
TODO
%
\begin{comment}
Внесение изменения в ресурсы
Оценка состояния приложения и его свойств безопасности

Как один компонент ведёт себя при некорректной работе другого компонента
Рассмотрение взаимодействий между компонентами:
демон — сетевое расширение, демон — общие компоненты, демон — демон, демон — библиотека, библиотека — приложение, библиотека — общие компоненты, общие компоненты — файл конфигурации 
Какие ошибки найдены
К каким рискам они относятся
Какие атаки могут быть проведены
Как они влияют на защищённость
\end{comment}

% -------------------------------------------------------------------------------------------------------------------------

\SubSubSectionTitle{Ревизия кода}{peerhood_analysis_security_testing_code_inspection}

\Paragraph{Зачем она дожна проводиться}
%
TODO
%
\begin{comment}
На что направлена ревизия
Как она будет происходит
Какие результаты необходимо получить
Что они дадут (связь с задачами анализа)

Анализ кода с точки зрения безопасности
Методы тестирования — лишь дополнительный инструмент
Более глубокий поиск уязвимостей
\end{comment}

\Paragraph{Что можно проверить в ходе ревизии кода}
%
TODO
%
\begin{comment}
Язык разработки С++: переполнение буфера, целочисленные данные, форматные строки, указатели
Apple Secure Coding Guide: помимо перечисленных выше ещё проверка входных данных, состояние гонки, IPC, файловая система, проблемы контроля доступа
C Secure Coding Standard: препроцессор, инициализация, целочисленные данные, массивы, строки, распределение памяти, ввод/вывод, временные файлы, окружение, сигналы
Seven Pernicious Kingdoms: проверка входных данных, неправильное использование API, механизм безопасности, проблема синхронизации, неправильная обработка исключительных ситуаций, нарушение инкапсуляции (раскрытие информации), программное окружение
Подведение итогов: выжимка из перечисленных выше списков
\end{comment}

\Paragraph{Что будет проверено в ходе ревизии кода}
%
TODO
%
\begin{comment}
Уязвимости, характерные для С++, будут проверены с помощью ревизии кода
Проверка входных данных, в каких местах: с помощью ревизии кода
Необходима проверка API: ревизия кода
Состояния гонки с помощью инструментальных средств, а примитивы синхронизации с помощью ревизии кода
Чек-лист из C Secure Coding Standard по списку с помощью ревизии кода
Программное окружение: с помощью ревизии кода + анализ Qt
Раскрытие информации: с помощью ревизии кода
Неправильная обработка исключений: ревизия кода
\end{comment}

\Paragraph{Общие проблемы С/C++}
%
TODO
%
\begin{comment}
Переполнение буфера - поиск массивов в коде
Целочисленные данные - поиск целочисленных переменных в коде и их отслеживание
Форматные строки - поиск форматных строк
Указатели - поиск указателей и их отслеживание
\end{comment}

\Paragraph{Входные данные}
%
TODO
%
\begin{comment}
Анализ кода, отвечающего за приём и отправку данных через сокетный интерфейс демона
Анализ API для приложений и API для сетевых интерфейсов
\end{comment}

\Paragraph{Синхронизация}
%
TODO
%
\begin{comment}
Анализ и поиск мест, где нужно синхронизация. Анализ кода
\end{comment}

\Paragraph{Использование API}
%
TODO
%
\begin{comment}
Анализ вызываемых системных API функций и Qt API
\end{comment}

\Paragraph{Механизм безопасности}
%
TODO
%
\begin{comment}
Отсутствует
\end{comment}

\Paragraph{Механизм обработки ошибок}
%
TODO
%
\begin{comment}
Поиск и анализ исключительных ситуаций в коде
\end{comment}

\Paragraph{Раскрытие информации}
%
TODO
%
\begin{comment}
Анализ сообщений об ошибках, ресурсов приложений, характер передачи пользовательских данных
\end{comment}