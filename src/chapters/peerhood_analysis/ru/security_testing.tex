\Paragraph{Доступность исходного кода PeerHood}
%
Набор используемых методов тестирования безопасности \RussianAbbreviation{ПО} зависит от доступности исходного кода тестируемого \RussianAbbreviation{ПО} и его сторонних компонентов.
%
\PeerHood является открытым \RussianAbbreviation{ПО}, поэтому его исходный код доступен \Reference{PeerHoodGitorious} (для тестирования используется версия \Italic{81eeb64} от \FourDigitsDate{2011}{08}{11}). 
%
Поэтому необходимо провести выбор необходимых методов тестирования из числа тех, которые рассматриваются в \InReferenceToSection{software_security_analysis}.
%
По причине доступности исходного кода в первую очередь встаёт вопрос о необходимости тестирования по методу "чёрного" ящика. 

% -------------------------------------------------------------------------------------------------------------------------

\SubSubSectionTitle{Выбор методов тестирования}{peerhood_analysis_security_testing_method_choosing}

\Paragraph{Fuzz-тестирование}
%
\Fuzz-тестирование по своей природе довольно затратно, а результаты его проведения можно получить более лёгким способом, а именно при помощи ревизии участков кода, представляющих собой точки входа приложения.
%
Для \PeerHood это "слабые" места, перечисленные в \InReferenceToSection{peerhood_analysis_risk_modeling}. 

\Paragraph{Реверс-инжиниринг}
%
Проведение реверс-инжиниринга является излишним по причине доступности исходного кода, а закрытые сторонние компоненты не используются. 
%
\PeerHood в качестве платформы использует фреймворк \Qt \Reference{Qt}, поэтому можно сказать, что его безопасность зависит от безопасности данной платформы. 

\Paragraph{Отладка по методу чёрного ящика}
%
Проведение отладки по методу чёрного ящика также излишне по тем же причинам, что и проведение реверс-инжиниринга. 
%
Более всесторонний анализ того, как приложение взаимодействует с программной средой, можно провести с помощью тестирования по методу "белого" ящика. 
%

\Paragraph{Сканеры уязвимости}
%
Использование сканеров уязвимостей излишне по той же причине, что и проведение \fuzz-тестирования. 
%
Более качественные результаты можно получить путём ревизии участков кода, представляющих собой "слабые" места \PeerHood. 

\Paragraph{Внедрение неисправностей в исходный код}
%
Внедрение неисправностей в исходный код также излишне по причине доступности исходного кода \PeerHood. 
%
Вместо него можно провести ревизию исходного кода с целью анализа механизма обработки ошибок и проверки принципа "безопасность на глубине". 
%
\PeerHood не является большим проектом. 

\Paragraph{Внедрение неисправностей}
%
Напротив, внедрение неисправностей в ресурсы приложения может быть эффективным. 
%
С помощью него можно оценить, насколько \PeerHood является отказоустойчивым к неисправностям среды (ресурсов, сторонних компонентов, программного окружения). 

\Paragraph{Тестирование внедрения}
%
Проведение тестирования внедрения направлено на оценку отказоустойчивости приложения. 
%
Оценка рисков, анализ "слабых" мест и поиск уязвимостей составляют множество мероприятий, которые используются при тестировании внедрения. 
%
Поэтому можно сказать, что процесс анализа безопасности \PeerHood подразумевает проведение тестирования внедрения. 

\Paragraph{Тестирование, основанное на свойствах}
%
То же самое можно сказать про проведение тестирования, основанного на свойствах. 
%
Тестирование безопасности \PeerHood направлено на оценку таких характеристик, как составляющие \CIATriad применительно к \PeerHood. Ниже приведён развёрнутый список данных характеристик:
\begin{itemize}
	%\leftskip2em%
	\setlength{\itemsep}{0pt}%
	%\setlength{\parsep}{0pt}%

	\item возможность передачи данных по сети
	\item конфиденциальность и целостность данных, передаваемых по сети

	\item доступность, конфиденциальность и целостность актуальной информации об устройствах в сетевом окружении
	\item возможность обмена пользовательских данных и данных об устройствах в сетевом окружении между устройствами
	\item возможность одновременной поддержки нескольких клиентов

	\item доступность, конфиденциальность и целостность актуальной информации о доступных сервисах в сетевом окружении
	\item возможность обмена информации о доступных сервисах в сетевом окружении между устройствами

	\item возможность передачи пользовательских данных между устройствами
	\item конфиденциальность и целостность пользовательских данных, передаваемых между устройствами
	\item возможность контроля работы системы
\end{itemize}

\Paragraph{Статический и динамический анализ кода}
%
Проведение статического и динамического анализа направлено на поиск уязвимостей уровня кода и выполнения соответственно. 
%
Использование разнообразных инструментальных средств способствует более детальному анализу \PeerHood. 
%
Но применение данных методов тестирования безопасности не позволяет провести всесторонний анализ приложения. 
%
Поэтому можно сказать, что их проведение обоснованно и не является излишним.

\Paragraph{Ревизия кода}
%
С помощью автоматизированных методов тестирования безопасности можно найти лишь ограниченное количество уязвимостей. 
%
Поэтому для выполнения всестороннего анализа безопасности \PeerHood необходимо проведение ревизии исходного кода. 
%
Причём, как уже было упомянуто выше, особенно это актуально для "слабых" мест приложения. 

% -------------------------------------------------------------------------------------------------------------------------

\SubSubSectionTitle{Проведение статического анализа}{peerhood_analysis_security_testing_static_analysis}

\Paragraph{Вводные слова про цели тестирования по методу белого ящика}
%
TODO
%
\begin{comment}
На что направлено тестирование: на анализ исходного кода, поиск в нём ошибок безопасности
Как оно будет происходить: анализ исходного кода с помощью инструментальных средств
Что будет использовано: только инструментальные средства
Какие результаты будут получены: информация об ошибках программирования
Что они дадут (связь с задачами тестирования): как найденные ошибки влияют на безопасность, с каким риском связаны, можно ли провести атаку
\end{comment}

\Paragraph{Статический анализ}
%
TODO
%
\begin{comment}
Используется во время разработки приложения
Установить цели ревизии
Использовать инструментальные средства
По отчётам провести ревизию слабых мест (анализ отчёта)
Недостатки: является лишь вспомогательным инструментом, для выявления ошибок времени выполнение применяются другие методы тестироавания
Инструменты: Splint, RATS, ITS4, PVS Studio, GCC (флаги компилятора)

Поиск ошибок программирования с помощью инструментальных средств
Найденные ошибки проанализировать на степень их влияния на безопасность
Уязвимости связать с возможностью атак и с рисками безопасности => влияние на защищённость
Используемые инструменты: RATS, ITS4, PVS Studio, GCC
\end{comment}

% -------------------------------------------------------------------------------------------------------------------------

\SubSubSectionTitle{Проведение динамического анализа кода}{peerhood_analysis_security_testing_dynamic_analysis}

\Paragraph{Вводные слова про цели тестирования по методу серого ящика}
%
TODO
%
\begin{comment}
На что направлено тестирование
Как оно будет происходит
Что будет использовано
Какие результаты будут получены
Что они дадут (связь с задачами тестирования)
\end{comment}

\Paragraph{Динамический анализ кода}
%
TODO
%
\begin{comment}
Обнаружение аномальных действий, 
Направлены на поиск ошибок взаимодействия с пользователем, программным окружением или сторонними компонентами: переполнение буфера, уязвимость форматных строк, указателей
Инструменты: stackguard, libsafe, formatguard, valgrind, helgrind

Запуск приложения под тестами и поиск ошибок выполнения с помощью инструментальных средств
Запуск valgrind
Запуск helgrind
Остальное в рамках ревизии кода, так как используется библиотека Qt (многие опасные средства С/С++  инкапсулированы в функциях Qt)
\end{comment}

% -------------------------------------------------------------------------------------------------------------------------

\SubSubSectionTitle{Внедрение неисправностей}{peerhood_analysis_security_testing_fault_injection}

\Paragraph{Вводные слова про цели тестирования по методу чёрного ящика}
%
TODO
%
\begin{comment}
На что направлено тестирование
Как оно будет происходит
Что будет использовано
Какие результаты будут получены
Что они дадут (связь с задачами тестирования)
\end{comment}

\Paragraph{Внедрение неисправностей в бинарный код}
%
TODO
%
\begin{comment}
Внесение изменения в ресурсы
Оценка состояния приложения и его свойств безопасности

Как один компонент ведёт себя при некорректной работе другого компонента
Рассмотрение взаимодействий между компонентами:
демон — сетевое расширение, демон — общие компоненты, демон — демон, демон — библиотека, библиотека — приложение, библиотека — общие компоненты, общие компоненты — файл конфигурации 
Какие ошибки найдены
К каким рискам они относятся
Какие атаки могут быть проведены
Как они влияют на защищённость
\end{comment}

% -------------------------------------------------------------------------------------------------------------------------

\SubSubSectionTitle{Ревизия кода}{peerhood_analysis_security_testing_code_inspection}

\Paragraph{Зачем она дожна проводиться}
%
TODO
%
\begin{comment}
На что направлена ревизия
Как она будет происходит
Какие результаты необходимо получить
Что они дадут (связь с задачами анализа)

Анализ кода с точки зрения безопасности
Методы тестирования — лишь дополнительный инструмент
Более глубокий поиск уязвимостей
\end{comment}

\Paragraph{Что можно проверить в ходе ревизии кода}
%
TODO
%
\begin{comment}
Язык разработки С++: переполнение буфера, целочисленные данные, форматные строки, указатели
Apple Secure Coding Guide: помимо перечисленных выше ещё проверка входных данных, состояние гонки, IPC, файловая система, проблемы контроля доступа
C Secure Coding Standard: препроцессор, инициализация, целочисленные данные, массивы, строки, распределение памяти, ввод/вывод, временные файлы, окружение, сигналы
Seven Pernicious Kingdoms: проверка входных данных, неправильное использование API, механизм безопасности, проблема синхронизации, неправильная обработка исключительных ситуаций, нарушение инкапсуляции (раскрытие информации), программное окружение
Подведение итогов: выжимка из перечисленных выше списков
\end{comment}

\Paragraph{Что будет проверено в ходе ревизии кода}
%
TODO
%
\begin{comment}
Уязвимости, характерные для С++, будут проверены с помощью ревизии кода
Проверка входных данных, в каких местах: с помощью ревизии кода
Необходима проверка API: ревизия кода
Состояния гонки с помощью инструментальных средств, а примитивы синхронизации с помощью ревизии кода
Чек-лист из C Secure Coding Standard по списку с помощью ревизии кода
Программное окружение: с помощью ревизии кода + анализ Qt
Раскрытие информации: с помощью ревизии кода
Неправильная обработка исключений: ревизия кода
\end{comment}

\Paragraph{Общие проблемы С/C++}
%
TODO
%
\begin{comment}
Переполнение буфера - поиск массивов в коде
Целочисленные данные - поиск целочисленных переменных в коде и их отслеживание
Форматные строки - поиск форматных строк
Указатели - поиск указателей и их отслеживание
\end{comment}

\Paragraph{Входные данные}
%
TODO
%
\begin{comment}
Анализ кода, отвечающего за приём и отправку данных через сокетный интерфейс демона
Анализ API для приложений и API для сетевых интерфейсов
\end{comment}

\Paragraph{Синхронизация}
%
TODO
%
\begin{comment}
Анализ и поиск мест, где нужно синхронизация. Анализ кода
\end{comment}

\Paragraph{Использование API}
%
TODO
%
\begin{comment}
Анализ вызываемых системных API функций и Qt API
\end{comment}

\Paragraph{Механизм безопасности}
%
TODO
%
\begin{comment}
Отсутствует
\end{comment}

\Paragraph{Механизм обработки ошибок}
%
TODO
%
\begin{comment}
Поиск и анализ исключительных ситуаций в коде
\end{comment}

\Paragraph{Раскрытие информации}
%
TODO
%
\begin{comment}
Анализ сообщений об ошибках, ресурсов приложений, характер передачи пользовательских данных
\end{comment}