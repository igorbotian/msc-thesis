\Paragraph{Анализ вопроса и поставленных целей исследования}
%
TODO
%
\begin{comment}
Анализ вопроса исследования (какую роль играет практическая часть исследования и в чём она заключается)
Анализ целей исследования (что нужно получить в результате проведения практической части исследования)
Цели и задачи проведения тестирования безопасности PeerHood
Цель проведения тестирования безопасности: является ли PeerHood защищённым.
\end{comment}

\Paragraph{Задачи, которые необходимо решить для достижения поставленной цели}
%
TODO
%
\begin{comment}
анализ спецификации PeerHood + факторы, влияющие на его безопасность
анализ рисков PeerHood
результаты анализа спецификации и рисков помогут в том, чтобы провести мероприятия, связанные с тестированием безопасности ПО
результаты тестирования безопасности дадут информацию об уязвимостях в PeerHood
анализ результатов тестирования даст ответ на вопрос, является ли PeerHood защищённым (ненарушение принципов CIA Triad + 2 других)
\end{comment}

\Paragraph{Что даст анализ PeerHood}
%
TODO
%
\begin{comment}
Анализ спецификации и рисков PeerHood даёт информацию о том, какие факторы влияют на его безопасность и что формируют принципы CIA Triad применимо к нему
Анализ PeerHood заключается в высокоуровневом анализе спецификации
Это даст информацию о том, что влияет на его отказоустойчивость, а значит на свойства CIA Triad 
Значит, последующие задачи направлены на проверку данных свойств
Результаты анализа непосредственно формируют вектор тестирования и множество используемых методов 
Какие мероприятия необходимо провести, чтобы протестировать PeerHood на безопасность
Согласно Майклу [Michael2005], мерояприятия по тестированию безопасности ПО являются следующими: понимание работы ПО и допущений разработчиков (уже получено как результат анализа PeerHood), анализ рисков с помощью техники их моделирования (уже получено ранее),  анализ возможных уязвимостей ПО (будет выполнено в рамках ревизии кода), использование необходимых методов тестирования для получения результатов
\end{comment}

\Paragraph{Что даст анализ результатов тестирования безопасности PeerHood}
%
TODO
%
\begin{comment}
Результаты тестирования безопасности ПО дадут возможность понять, нарушаются ли принципы CIA Triad
Таким образом можно понять, может ли PeerHood корректно функционировать при проведении атак
А это определяет степень его отказоустойчивости и ответить на вопрос, является ли он защищённым
\end{comment}