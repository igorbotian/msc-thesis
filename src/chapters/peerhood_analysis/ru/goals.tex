\Paragraph{Анализ вопроса исследования}
%
Практическая часть данной рыботы заключается в анализе безопасности сетевого окружения \IT{PeerHood}. 
%
Она играет наиболее важную роль, так как позволяет дать ответ на вопрос данного исследования: содержит ли \IT{PeerHood} уязвимости \RussianAbbreviation{ПО}. 
%
Таким образом, в результате проведения анализа безопасности \IT{PeerHood} необходимо \Important{выяснить, является ли он защищённым \RussianAbbreviation{ПО}} или нет (см. \ReferenceToSection{software_security_analysis_secure_software_concept}). 

\Paragraph{Задачи, необходимые для достижения поставленной цели}
%
Следуя Майклу \Reference{Michael2005}, для анализа безопасности \RussianAbbreviation{ПО} необходимо провести \Important{ряд мероприятий}. 
%
Касательно \IT{PeerHood}, к ним относятся:
\begin{enumerate}
	\leftskip2em%
	\setlength{\itemsep}{0pt}%
	\setlength{\parsep}{0pt}%

	\item анализ спецификации
	\item оценка рисков безопасности
	\item проведение мероприятий, связанных с тестированием безопасности
	\item анализ полученных результатов
\end{enumerate} 

%
%Это позволяет тщательно провести оценку рисков безопасности проекта. 
%
%На основе результатов анализа рисков можно сформировать набор необходимых для проведения методов тестирования безопасности. 
%
%В свою очередь, анализ полученных результатов тестирования даст ответ на вопрос исследования данной диссертации. 

\Paragraph{Что даёт анализ спецификации PeerHood}
%
Анализ спецификации \IT{PeerHood} даёт информацию о том, какие факторы могут влиять на его безопасность, а также что формирует принципы \Term{CIA Triad} применительно к нему (см. \ReferenceToSection{software_security_concept}). 
%
Он позволяет получить в формализованном виде \Emphasis{набор характеристик}, определяющих отказоустойчивость проекта, а значит и его защищённость. 
%
Такие характеристики формируют требования безопасности к проекту. 
%
Поэтому все другие задачи направлены на анализ данных характеристик. 

\Paragraph{Что даёт оценка рисков PeerHood}
%
Результаты анализа спецификации позволяют тщательно провести оценку рисков безопасности \IT{PeerHood}. 
%
А на основе результатов оценки рисков можно сформировать набор методов тестирования безопасности \RussianAbbreviation{ПО}, необходимых для проведения. 

\Paragraph{Что даёт тестирование безопасности ПО и анализ его результатов}
%
Тестирование безопасности \IT{PeerHood} направлено на проверку соответствия требований безопасности тому, как они выполняются в текущей реализаци проекта. 
%
Его результаты дают возможность понять, содержит ли \IT{PeerHood} уязвимости безопасности. 
%
А значит выяснить, нарушаются ли принципы \Term{CIA Triad} касательно \IT{PeerHood}, то есть может ли он корректно функционировать при проведении атак.