\Paragraph{Цель проведения анализа рисков}
%
Анализ спецификации \IT{PeerHood} позволяет понять, что формирует его функциональность. 
%
Но необходимо исследовать, что влияет на его \Emphasis{отказоустойчивость}. 
%
Это позволяет сделать проведение анализа рисков безопасности \IT{PeerHood}. 

\Paragraph{Анализ архитектуры и диаграммы развёртывания PeerHood}
%
Зная функциональные характеристики \IT{PeerHood}, необходимо понять, как они реализуются в проекте. 
%
Согласно архитектуре, \IT{PeerHood} состоит из нескольких \Emphasis{компонентов}: демон, пользовательская библиотека и набор сетевых расширений (см. \ReferenceToSection{peerhood_architecture}).
%
То, как они реализованы и взаимодействуют между собой, показано на диаграмме развыртывания \IT{PeerHood} (\ReferenceToFigure{peerhood_deployment_diagram}). 

\ImageFigure{Диаграмма развёртывания PeerHood}{peerhood_deployment_diagram}

\Paragraph{За что отвечает каждый компонент}
%
Пользовательские приложения в виде сервиса взаимодействует с библиотекой \IT{PeerHood} посредством специально спроектированного \Abbreviation{API}.
%
Сама библиотека с одной стороны использует общие компоненты для регистрации сервиса, контроля работы \IT{PeerHood} и осуществления других действий, а с другой взаимодействует с локальным демоном \IT{PeerHood}.
%
Демон \IT{PeerHood} использует доступные сетевые расширения для взаимодействия с другими устройствами и общие компоненты для чтения конфигурации. 

\Paragraph{Пары компонентов, взаимодействущих между собой}
%
Таким образом можно составить \Important{пары} взаимодействующих между собой компонентов.
%
Среди них следующие:
\begin{itemize}
	%\leftskip2em%
	\setlength{\itemsep}{0pt}%
	%\setlength{\parsep}{0pt}%

	\item Сетевое расширение -- Сетевое расширение
	\item Демон -- Сетевое расширение
	\item Демон -- Библиотека общих компонентов
	\item Демон -- Демон
	\item Библиотека общих компонентов -- Файловая система
	\item Пользовательская библиотека -- Библиотека общих компонентов
	\item Пользовательская библиотека -- Демон
	\item Пользовательское приложение -- Пользовательская библиотека
\end{itemize}

\Paragraph{Переход к анализе каждой пары}
Каждая из пар отвечает за обеспечение некоторой функционильной характеристики, а следовательно, связана с какой-либо составляющей безопасности \IT{PeerHood}. 
%
Ниже приводится анализ взаимодействия каждой из них по отдельности, а также риски безопасности, связанные с каждой из них. 

\Paragraph{Сетевое расширение - Сетевое расширение}
%
Сетевые расширения используются для поддержки заданной сетевой технологии и отвечают за поддержку связи между устройствами и передачу данных между ними. 
%
С точки зрения безопасности, с этим связаны следующие риски безопасности. 
%
Передаваемые данные могут быть ценными, поэтому на них может быть проведена атака с целью их перехвата, порчи, подмены и выполнения других действий. 
%
Они должны быть защищены на уровне используемой сетевой технологии или протокола или путём шифрования. 
%
С другой стороны, возможно проведение атаки и на саму сетевую технологию или протокол. 
%
Таким образом, если реализация данного взаимодействия не является отказоустойчивой, то это может привести к нарушению всех принципов \Term{CIA Triad} по отношению \IT{PeerHood}. 

\Paragraph{Демон - Сетевое расширение}
%
Демон осуществляет взаимодействие с сетевым расширением с помощью \Abbreviation{API} для передачи данных. 
%
Они могут представлять из себя как пользовательские данные, так и информацию о доступных сервисах и устройствах в сети. 
%
Поэтому каждое сетевое расширение должно быть доверенным для демона.
%
С точки зрения безопасности существует риск выполнения зловредных действий над ними со стороны расширения. 
%
Это может привести к нарушению принципов целостности и конфиденциальности по отношению к \IT{PeerHood}. 

\Paragraph{Демон, Программная библиотека - Библиотека общих компонентов}
%
\IT{PeerHood} является конфигурируемым приложением. 
%
Программные средства его конфигурирования предоставляются библиотекой общих компонентов. 
%
С точки зрения безопасности, полученные от библиотеки значения могут содержать некорректные данные. 
%
Это может повлиять на корректность функционирования приложения и привести к нарушению принципа доступности по отношению к \IT{PeerHood}. 

\Paragraph{Демон, пользовательская библиотека - Демон}
%
Взаимодействие между демонами и между пользовательской библиотекой и демоном осуществляется посредством коммуникации через сокет. 
%
Передаваемые данные могут содержить как пользовательскые данные, так и информацию о доступных сервисах и устройствах в сетевом окружении. 
%
С точки зрения безопасности, взаимодействие с демоном, функционирующим на другом устройстве, может привести к выполнению зловредных действий над этими данными. 
%
Также возможен вариант, когда работа приложения, маскирующегося под удалённого демона или пользовательскую библиотеку, может быть предоставлять из себя атаку на самого демона с целью его выведения из строя. 
%
Таким образом, могут быть нарушены все принципы \Term{CIA Triad}, связанные с данными, по отношению к \IT{PeerHood}. 

\Paragraph{Библиотека общих компонентов - Файловая система}
%
Как уже было упомянуто ранее, программные средства конфигурирования \IT{PeerHood} предоставляются библиотекой общих компонентов. 
%
Сам процесс конфигурирования заключается в чтении заданного файла конфигурации и предоставлении значений, хранящихся в нём. 
%
В частности, с помощью данного средства осуществляется получение пути, по которому находятся файлы сетевых расширений. 
%
С точки зрения безопасности, возможность несанкционированного изменения содержимого файла конфигурации может повлиять на отказоустойчивать и безопасность \IT{PeerHood}. 
%
К примеру, существует риск загрузки недоверенного сетевого расширения, которое может осуществлять зловредные действия над передаваемыми данным.
%
Таким образом, это приведёт к нарушению принципов конфиденциальности и целостности данных по отношению к \IT{PeerHood}. 

\Paragraph{Итоги анализа взаимодействия компонентов}
%
Подводя итоги анализа взаимодействия компонентов \IT{PeerHood}, можно сделать вывод о том, что проект содержит ''уязвимые'' места. Наибольшая вероятность проведения атаки связана с ними. На диаграмме развёртывания \IT{PeerHood} (\ReferenceToFigure{peerhood_deployment_diagram}) они изображены в виде связей между компонентами. ''Уязвимые'' места \IT{PeerHood} являются следующими:
\begin{itemize}
	%\leftskip2em%
	\setlength{\itemsep}{0pt}%
	%\setlength{\parsep}{0pt}%

	\item файл конфигурации
	\item сокет, используемый демоном для поддержки клиентов
	\item \Abbreviation{API} для сетевых расширений
	\item \Abbreviation{API} для средств конфигурации
	\item \Abbreviation{API} для пользовательских приложений
\end{itemize}

\Paragraph{Учёт мобильности окружения}
%
Так как использование \IT{PeerHood} направлено на функционирование в мобильной среде, необходимо учитывать связанные с ней угрозы безопасности. 
%
Мобильная среда характеризуется угрозой выполнения зловредных действий над передаваемыми данными, утечки конфиденциальной пользовательской информации и отказом \RussianAbbreviation{ПО} в работе. 

\Paragraph{Учёт пирингового взаимодействия}
%
В свою очередь, сетевое окружение \IT{PeerHood} основывается на пиринговой концепции взаимодействия. 
%
С точки зрения безопасности, доступность сервиса для всех устройств в окружении характеризуется угрозой проведения атаки как на передаваемые данные, функционирующие сервисы, так и на само устройство в целом. 

\Paragraph{Выводы по учётам}
%
Можно сделать вывод о том, что учёт пирингового взаимодействия и мобильности сетевого окружения лишь подчёркивает значимость рисков безопасности, рассмотренных в данном подразделе. 
%
Такиим образом, по отношению к \IT{PeerHood} данный факт способствует росту вероятности нарушения конфиденциальности и целостности передаваемых данных, а также доступности устройств в сетевом окружении. 

\Paragraph{Переход к тестировании безопасности}
%
Для оценки рисков безопасности \IT{PeerHood} необходимо провести тестирование его безопасности. 
%
Это позволит оценить отказоустойчивость компонентов данного \RussianAbbreviation{ПО}. 
%
Мероприятия, связанные с тестированием безопасности, рассматриваются в следующем подразделе. 