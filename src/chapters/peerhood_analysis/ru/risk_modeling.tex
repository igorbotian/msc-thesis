\Paragraph{Анализ архитектуры и диаграммы развёртывания PeerHood}
%
TODO
%
\begin{comment}
PeerHood состоит из нескольких компонентов
Рассмотрение диаграммы развёртывания
Библиотека общих компонентов работает с файловой системой (защита конфигурации)
Сетевой плагин осуществляет взаимодействие с сетью (какая сетевая технология используется: безопасность передачи данных)
Демон тоже это делает посредством сокета (безопасность передачи данных)
Демон использует библиотеку общих компонентов для чтения конфигурации
Демон использует сетевые плагины (их защита, зловредный плагин, безопасность API)
Библиотека взаимодействует с общими компонентами для чтения конфигурации
Библиотека взаимодействует с демоном с помощью сокетного интерфейса (зловредная библиотека, зловредный удалённый демон)
Библиотека взаимодействует с пользовательскими приложениями (безопасность API, защита пользовательских данных, зловредная библиотека)
PeerHood направлен на передачу пользовательских данных (как достигается их конфиденциальность)
PeerHood имеет доступ к файловой системе (с эксплойта в нём можно получить к ним доступ, уровень доступа)
Демон запускается в фоновом процессе (для постоянной работы может запускаться под суперпользоавателем)
PeerHood использует Qt (уязвимости Qt == уязвимости PeerHood)
Анализ конфигурации (что может подразумеваться разработчиками, но влияет на безопасность)
\end{comment}

\Paragraph{Учёт мобильности окружения}
%
TODO
%
\begin{comment}
Мобильное окружение более уязвимое
Пассивный перехват или порча данных — надо обеспечить защиту передаваемых данных
Возможность утечки конфиденциальной информации — максимальная защита файловой системы, передачи пользовательских данных, API
Отказ в работе — если нет ни одного сетевого плагина (сделать localhost встроенным?), конфигурация (?), атака через пользовательское приложение, атака через зловредный плагин, атака через демона
\end{comment}

\Paragraph{Учёт пиринговости окружения}
%
TODO
%
\begin{comment}
Пиринговость усложняет обеспечение безопасности
Анализ трафика
Участник сети, сервис, данные — жертвы атаки
Нарушение доступности — отказ в работе сервиса или низкий уровень его функционирования
Нарушение целостности — повреждение данных
Нарушение конфиденциальности — все сервисы доступны всем участникам сети
Атаки: человек посередине, саморепликация
Защита: реактивный режим, система репутации пользователей
\end{comment}

\Paragraph{Какого вида атаки могут быть проведены на PeerHood (по классификации)?}
%
TODO
%
\begin{comment}
Расммотреть какие уязвимости к проведению каких атак способствуют:
Перехват, анализ трафика, модификация, фальсификация, 
Reconnaisance-атаки для того, чтобы узнать больше о системе 
Enabling-атаки для возможности проведения атаки других видов
Disclosure-атаки для получения конфиденциальных данных
Subversion-атаки для изменения хода работы системы
Sabotage-атаки для отказа в работе системы и доступе к ней для других пользователей
\end{comment}

\Paragraph{Риски безопасности PeerHood}
%
TODO
%
Полная выжимка из всех угроз безопасности, перечисленных выше

\Paragraph{Влияние рисков на CIA Triad}
%
TODO
%
\begin{comment}
Каждый принцип — с чем связан касательно PeerHood — какие угрозы могут его нарушить — с помощью какой атаки можно это совершить
\end{comment}

\ImageFigure{Диаграмма развёртывания PeerHood}{peerhood_deployment_diagram}