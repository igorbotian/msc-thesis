\Paragraph{Из чего состоит спецификация PeerHood}
%
Спецификация \IT{PeerHood} содержит следующую информацию.
%
\Emphasis{Цели} проекта определяют то, какой круг проблем должен решаться с помощью данного проекта. 
%
\Emphasis{Функциональные требования} определяют то, какие функциональные характеристики должны поддерживаться \IT{PeerHood} для достижения поставленных перед ним целей. 
%
\Emphasis{Нефункциональные требования} определяют структуру проекта. 

% ---------------------------------------------------------------------------------------------------------------

\SubSubSectionTitle{Анализ целей PeerHood}{peerhood_analysis_specification_analysis_goals_analysis}

\Paragraph{Цели PeerHood}
%
Цель \IT{PeerHood} заключается в обеспечении \Emphasis{персональных коммуникаций} в мобильной пиринговой среде в независимости от используемых сетевых технологий. 
%
Это достигается засчёт обеспечения проактивности, способности к взаимодействию и реактивности (см. \ReferenceToSection{peerhood_concept}).

\Paragraph{Анализ способности к взаимодействию}
%
Обеспечение \Emphasis{способности к взаимодействию} формирует логический уровень приложения, отвечающий за прозрачность взаимодействия между мобильными устройствами.
%
Согласно концепции \IT{PeerHood}, он служит для связи между элементами типа ''Сетевые технологии'' (см. \ReferenceToFigure{peerhood_concept}). 

\Paragraph{Анализ реактивности}
%
Обеспечение \Emphasis{реактивности} формирует логический уровень приложения, отвечающий за коммуникацию между пользовательскими сервисами. 
%
Согласно концепции \IT{PeerHood}, он служит для связи между элементами типа ''Пользовательские приложения'' (см. \ReferenceToFigure{peerhood_concept}). 

\Paragraph{Анализ проактивности}
%
Обеспечение \Emphasis{проактивности} формирует логический уровень приложения, отвечающий за связь между двумя другими логическими уровнями. 
%
Согласно концепции \IT{PeerHood}, он служит для связи между элементами ''Сетевые технологии'' и ''Модули промежуточного уровня'', а также элементами ''Пользовательские приложения'' и ''Модули промежуточного уровня'' (см. \ReferenceToFigure{peerhood_concept}). 

\Paragraph{Резюме}
%
Таким образом, корректность работы компонентов \IT{PeerHood} на каждом логическом уровне определяет возможность успешного решения задач, на которые он направлен. 
%
А их отказоустойчивость определяет возможность корректного функционирования \IT{PeerHood} при проведении атаки со стороны злоумышленника.

% ---------------------------------------------------------------------------------------------------------------

\SubSubSectionTitle{Анализ требований к PeerHood}{peerhood_analysis_specification_analysis_requirement_analysis}

\Paragraph{Анализ требований к PeerHood}
%
Корректность работы каждого компонента \PeerHood определяется выполнением связанных с ним требований, функциональных и нефункциональных.
%
А все требования в совокупности направлены на достижение целей этого проекта. 

\Paragraph{Способность к взаимодействию}
%
\Important{Способность к взаимодействию} достигается выполнением следующих требований: прозрачное взаимодействие, установление соединения, передача данных между устройствами, архитектура сетевых расширений и управление сетью. 
%
Его обеспечение заключается в прозрачном обмене данных между устройствами с возможностью переключения между доступными сетевыми технологиями.
%
Это достигается с помощью \Emphasis{расширений}, каждое из которых направлено на поддержку определённой сетевой технологии, имеет специальный интерфейс для выполнения операций сетевого уровня (передача данных, установление соединения и другие). 

\Paragraph{Проактивность}
%
\Important{Проактивность} достигается выполнением следующих требований: обнаружение устройств, слежение за устройствами, управление компонентами и основа для одновременной работы нескольких клиентов.
%
Её обеспечение заключается в предоставлении актуальной информации о сетевом окружении одновременно одному или более клиентам. 
%
Это достигается с помощью \Emphasis{модулей промежуточного уровня}, связывающих компоненты других уровней. 
%
Компонентам более высокого уровня эти модули предоставляют информацию о сетевом окружению и уведомляют об изменениях в нём. 
%
В свою очередь, компоненты более низкого уровня используются ими для непосредственной передачи данных и обработки приходящих от них уведомлений о сетевых событиях. 

\Paragraph{Реактивность}
%
\Important{Реактивность} достигается выполнением следующих требований: обнаружение сервисов, совместное использование сервисов, контроль со стороны пользователя, предоставление клиентского интерфейса для получения событий.
%
Её обеспечение заключается в совместном использовании локальных и удалённых сервисов и контроле системы со стороны пользователя. 
%
Это достигается с помощью \Emphasis{модулей верхнего уровня}, которые направлены на выполнение пользовательский операций в рамках концепции сервисов. 

\Paragraph{Функциональные характеристики PeerHood}
%
Понимание того, засчёт чего осуществляется функционирование всех компонентов \IT{PeerHood}, позволяет определить \Emphasis{множество функциональных характеристик} данного проекта. 
%
В \InReferenceToTable{peerhood_assets} приведены характеристики, от которых зависит успешное выполнение каждой цели проекта: проактивности, способности к взаимодействию и реактивности. 
%
Их анализ позволяет определить составляющие информационной безопасности по отношению к данному проекту. 

\TableFigure{Функциональные характеристики PeerHood}{peerhood_assets} {
	\begin{tabular}{ | p{6cm} | p{8cm} | }
	  \hline                       
	  \Bold{Цель} & \Bold{Характеристики} \\ \hline
	  Способность к взаимодействию & передача данных по сети, \linebreak получение актуальной информации об устройствах в сетевом окружении (посредством разных сетевых технологий) \\ \hline
	  Проактивность & получение актуальной информации об устройствах в сетевом окружении, \linebreak обмен данных между устройствами, \linebreak одновременная поддержка нескольких клиентов (локальных или удалённых) \\ \hline
	  Реактивность & получение информации о доступных сервисах в сетевом окружении (локальных или удалённых), \linebreak передача пользовательских данных, \linebreak контроль работы системы \\ \hline
	\end{tabular}
}

\Paragraph{Составляющие безопасности PeerHood}
%
С точки зрения безопасности, влияние на каждую функциональную характеристику происходит по-разному в зависимости от её \Emphasis{природы}. 
%
Это зависит от того, как она связана с каждым принципом \Term{CIA Triad}: доступностью, конфиденциальностью или целостностью. 

\Paragraph{Доступность касательно PeerHood}
%
\Important{Доступность} связана с такими функциональными характеристиками, как получение актуальной информации об устройствах и сервисах в сетевом окружении, одновременная работа нескольких клиентов, а также с контроль работы системы и . 
%
Это объясняется тем, что отказоустойчивость данных характеристик связана с фактом возможности как получения информации, контроля работы системы, так и поддержки одновременной работы нескольких клиентов. 

\Paragraph{Конфиденциальность касательно PeerHood}
%
\Important{Конфиденциальность} связана с такими функциональными характеристиками, как передача данных по сети и между компонентами приложения. 
%
Согласно концепции \IT{PeerHood}, отправляемые пользовательские данные передаются по всем уровням приложения (модули промежуточного уровня, сетевые расширения, сеть), поэтому существует вероятность их перехвата.

\Paragraph{Целостность касательно PeerHood}
%
\Important{Целостность} связана со всеми функциональными характеристиками, за исключением контроля работы системы и одновременной работы нескольких клиентов. 
%
Они связаны с передачей данных, пользователя или информацией о сервисах и устройствах, поэтому существует вероятность их изменения, подмены и других видов атак.

\Paragraph{Доступ к системным и пользовательским данным}
%
Стоит отметить, что каждый компонент \IT{PeerHood} потенциально может иметь доступ к системным файлам, в том числе пользовательским. 
%
Данный факт в свою очередь также влияет на безопасность проекта, так как они могут быть испорчены или украдены. 
%
Поэтому потенциальный доступ к системным данным связан с проблемой их целостности и конфиденциальности. 

\Paragraph{Переход к анализу рисков PeerHood}
%
Нарушение каждого принципа \Term{CIA Triad}, рассмотренного выше, влечёт за собой необеспечение безопасности \IT{PeerHood}. 
%
Существует множество рисков безопасности, влияющих на нарушение каждого из них.