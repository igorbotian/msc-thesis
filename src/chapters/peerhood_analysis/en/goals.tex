\Paragraph{Research question analysis}
%
Practical part of the work deals with security analysis of PeerHood network environment. 
%
It plays an important role because it allows give an answer to the question of the research: whether PeerHood contains software security vulnerabilities or not. 
%
Thus analysing PeerHood security it is necessary determine whether it is secure software or not (see \ReferenceToSection{software_security_analysis_secure_software_concept}). 

\Paragraph{Tasks that has be stated to obtain the result}
%
Following Michael \Reference{Michael2005}, it is necessary to conduct a number of activities to analyse software security. 
%
Concerning PeerHood they are the following: 
\begin{enumerate}
	\leftskip2em%
	\setlength{\itemsep}{0pt}%
	\setlength{\parsep}{0pt}%

	\item specification analysis
	\item estimation of security risks
	\item carrying out the activities related to security testing
	\item obtained results analysis
\end{enumerate}

\Paragraph{What PeerHood specification analysis gives}
%
PeerHood specification analysis gives information about which factors may affect it's security, as well as composes CIA Triad principles concerning the project (see \ReferenceToSection{software_security_concept}). 
%
It enables to get a set of formalized characteristics determining fault-tolerance of the project, and that means it's safety. 
%
Such characteristics compose requirements to the project. 
%
Thus all the following tasks aim at the characteristics analysis. 

\Paragraph{What PeerHood risk estimation gives}
%
The results of the specification analysis enable to make PeerHood security risks estimation thoroughly. 
%
And on the basis of the results it is possible to compose a set of software security testing methods. 

\Paragraph{What software security testing gives; analysis of the given results}
%
PeerHood security testing aim at check of a compliance of security requirements to with their realization in the current implementation of the project. 
%
The testing results give a possibility to realize whether PeerHood contains security vulnerabilities or not. 
%
And it means to determine are PeerHood CIA Triad principles violated or not, can it operates properly under attack or not. 