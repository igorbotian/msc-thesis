\Paragraph{The fact of PeerHood source code availability}
%
A set of utilized software security testing methods depends on source code availability of a tested application and it's third-party components. 
%
PeerHood is an open source software, the source code is available on \Reference{PeerHoodGitorious} (the \Italic{81eeb64} version of \FourDigitsDate{2011}{08}{11} is used for the testing). 
%
Thus it is necessary to make a selection of appropriate testing methods of the ones considered in \InReferenceToSection{software_security_analysis}. 
%
Because PeerHood source code is available, in the first place it is necessary to examine applicability of black box testing methods. 

% -------------------------------------------------------------------------------------------------------------------------

\SubSubSectionTitle{Testing methods selection}{peerhood_analysis_security_testing_method_choosing}

\Paragraph{Fuzz testing}
%
Fuzz testing by its nature is quite cost-based, and the results can be obtained using easier methods, namely using entry points source code revision. 
%
For PeerHood the entry points are the ``vulnerable'' places listed in \InReferenceToSection{peerhood_analysis_risk_modeling}. 

\Paragraph{Reverse engineering}
%
PeerHood reverse engineering conducting is unnecessary because of the source code availability, and also closed third-party components are not utilized by the project. 
%
PeerHood utilizes Qt framework \Reference{Qt} as a platform, thus it can be said that PeerHood security depends of security of this platform. 

\Paragraph{Black box debugging}
%
PeerHood black box debugging is also unnecessary for the same reasons. 
%
More comprehensive analysis of a way how an application interacts with software environment can be peformed using white box testing methods. 

\Paragraph{Vulnerability scanners}
%
Utilization of vulnerability scanners is also unnecessary for the same reasons. 
%
More qualitative results can be obtained using source code revision of PeerHood ``vulnerable'' places. 
%
Source code revision aimed at error handling mechanisms analysis and checking the ``security in depth'' principle can be performed instead of this method. 
%
PeerHood is not so big project. 

\Paragraph{Fault injection}
%
On the contrary, resource and component fault injection may be effective. 
%
Using this method it is possible to estimate how much PeerHood is fault-tolerant to it's environment faults (resources, third-party component, software environment). 

\Paragraph{Penetration testing}
%
Penetration testing method is aimed at estimation of an application fault-tolerance. 
%
Risk estimation, ``vulnerable'' places analysis, and search of vulnerabilities put together a set of activities that are used for penetration testing performance. 
%
Thus it can be said that PeerHood security analysis implies penetration testing. 

\Paragraph{Property based testing}
%
The same can be said about property based testing. 
%
PeerHood security testing is aimed at estimation of such characteristics as the CIA Triad principles concerning PeerHood. 
%
A detailed list of such characteristics is gived below:
\begin{itemize}
	%\leftskip2em%
	\setlength{\itemsep}{0pt}%
	%\setlength{\parsep}{0pt}%

	\item availability of data transfer
	\item confidentiality and integrity of transferred data

	\item availability, confidentiality, and integrity of up-to-date information about devices in the network environment
	\item availability of exchange of user data and information about available devices between devices in the network environment
	\item possibility of simultaneous several clients serving

	\item availability, confidentiality, and integrity of up-to-date information about available services in the network environment
	\item availability of exhange of information about available services between devices in the network environment

	\item possibility of user data exchange between devices
	\item confidentiality and integrity of user data transferred between devices
	\item possibility of the system control
\end{itemize}

\Paragraph{Static and dynamic analysis}
%
Static and dynamic analysis conducting is aimed at search of vulnerabilities on source code level and at runtime respectively. 
%
Utilization of various tools promotes to more detail analysis of PeerHood. 
%
But utilization of these security testing methods does not enable to perform comprehensive analysis of the application. 
%
Thus it can be said that conducting of these methods is reasonable and necessary. 

\Paragraph{Source code revision}
%
Using automatic security testing methods one can find only a limited number of vulnerabilities. 
%
Thus it is necessary to perform source code revision to carry out more comprehensive PeerHood analysis. 
%
As it was already noted above, it is especially urgent for the ``vulnerable'' places of PeerHood. 

% -------------------------------------------------------------------------------------------------------------------------

\SubSubSectionTitle{PeerHood code static analysis}{peerhood_analysis_security_testing_static_analysis}

\Paragraph{Goals}
%
The goal of static analysis is a search of source code level security vulnerabilities (see \ReferenceToSection{software_security_analysis_blackbox_testing_techniques_static_code_analysis}). 
%
It conists in utilization of automatic tools intended for search of programming errors affecting the application security. 
%
To carry out PeerHood static analysis the following tools are used: Cppcheck, RATS, GCC compiler features related to security. 

\Paragraph{GCC}
%
GCC compiler allows to perform search of format string vulnerabilities, array bounds checking, detection of uninitialized data, and others types of errors at compile time \WebSite{GCC}. 
%
To perform this operations, the tool is executed with the special \SourceCode{-Wall} argument which switches execution of such operations on. 
%
PeerHood compilation passes without any errors and warnings. 
%
This says that vulnerabilities are not found using this tool. 

\Paragraph{Cppcheck}
%
Cppcheck is intended for searching the following types of errors: array bounds crossing, exception safety, memory leakage, uninitalized variable values, and others \WebSite{Cppcheck}. 
%
Utilization of this tools concerning PeerHood has not found any errors (see \ReferenceToAppendix{appendix_cppcheck_output}). 
%
Only some reccomendations are given but they are not related to security. 

\Paragraph{RATS}
%
RATS is intended for search of buffer overflow vulnerabilities, race conditions, and other errors \WebSite{RATS}. 
%
This tools gives only several security warnings but source code revision of the appropriate blocks of the code does not detect any errors (see \ReferenceToAppendix{appendix_rats_output}). 
%
The first warning is related to a possibility of overflow of the buffer that stores ICMP message data transferred by a network plug-in. 
%
Utilization of the buffer is carried out securely. 
%
The second warning is related to a possibility of a race condition state when several signals are registering via \SourceCode{signal()} function at the same time. 
%
But only signal is registered in the application. 
%
The last warning is related to non-safety of the \SourceCode{gethostbyname()} function because a returned by the function value can be fabricated from without. 
%
But software environment of the project (platform or \Abbreviation{OS}) is responsible for the safety of this function, not PeerHood itself. 

\Paragraph{Conclusions}
%
Thus PeerHood static analysis has not found any security errors on source code level. 
%
It can be explained by the fact that the application actively utilizes facilities of the Qt framework. 

% -------------------------------------------------------------------------------------------------------------------------

\SubSubSectionTitle{PeerHood code dynamic analysis}{peerhood_analysis_security_testing_dynamic_analysis}

\Paragraph{Goals}
%
The goal of dynamic code analysis is a search of security vulnerabilities at runtime (see \ReferenceToSection{software_security_analysis_gray_box_techniques_dynamic_code_analysis}). 
%
It consists in utilization of automatic tools intended for abnormal activity detection, search of errors related to interaction between a tested application and it's software environment, and detection of other defects affecting security of the application. 
%
Using these tools the following runtime errors can be detected: buffer overflow, format string vulnerabilies, memory leakage, and others. 
%
To perform PeerHood dynamic analysis Valgrind and Helgrind tools are utilized. 

\Paragraph{Valgrind}
%
Valgrind tool is intended to detect runtime errors listed above and errors related to memory allocation \WebSite{Valgrind}. 
%
Utilization of the tools has not detected any abnormal activities or runtime errors (see \ReferenceToAppendix{appendix_valgrind_output}). 
%
It should be noted that the tools has detected a potentional memory leakage. 
%
But the appropriate PeerHood source code revision has shown that the leakage is related to inner mechanism of Qt implementation. 

\Paragraph{Helgrind}
%
Helgring tools is intended for synchronization error detection, including race conditions \WebSite{Helgrind}. 
%
Utilization of the tools has not detected any errors related to incorrect use of synchronization primitives in the application (\ReferenceToAppendix{appendix_helgrind_output}). 
%
The tool, in turn, has detected possible race conditions in many places. 
%
But they are related to inner mechanisms of Qt implementation. 

\Paragraph{Conclusions}
%
Thus PeerHood dynamic analysis has not detected any runtime errors. 
%
However, potentional errors are detected in Qt framework that is used by PeerHood. 
%
Hence one can conclude that PeerHood security depends on security of this platform. 

% -------------------------------------------------------------------------------------------------------------------------

\SubSubSectionTitle{Fault injection}{peerhood_analysis_security_testing_fault_injection}

\Paragraph{Goal}
%
The goal of fault injection is a tolerance estimation of a tested application when it's component are functioning incorrectly or it's resources are not corrupted (see \ReferenceToSection{software_security_analysis_black_box_techniques_fault_injection}). 
%
It conists in fault injection into every component of a tested application, after that the appropriate interaction checking is carried out. 

\Paragraph{Fault injection concerning PeerHood}
%
PeerHoos is comprised of the following interacting components: a set of network plug-ins, the daemon, common components, the user library, user applications, the configuration file (see \ReferenceToSection{peerhood_analysis_risk_modeling}). 
%
The interaction is peformed in these ways: 
\begin{itemize}
	%\leftskip2em%
	\setlength{\itemsep}{0pt}%
	%\setlength{\parsep}{0pt}%

	\item dynamic loading of shared objects (libraries) and further interaction via \Abbreviation{API} (the daemon -- network plug-ins, the daemon -- common components, the user library -- common components, a user application - the user library)
	\item interaction via socket (the daemon -- the daemon, the user library -- the daemon)
	\item interaction with file system (the user library -- the configuration file)
\end{itemize}

\Paragraph{Interaction via API analysis}
%
Binary fault injection into a shared object (library) can be performed in two ways. 
%
The first one affects an interface part of the object. 
%
In this case an error occurs during dynamic binding of the object, this leads to it's failure. 
%
The second one affects only inner functioning change of this object, and this leads to it's incorrect functioning. 
%
It can affect a tested component, for instance, return of incorrect or even malicious data. 
%
To estimate tolerance of a tested component it is necessary to perform analysis of the \Abbreviation{API} utilized to perform this interaction. 
%
Because PeerHood source code is available, the best way to perform this operation is to revise appropriate blocks of the source code. 

\Paragraph{Interaction via socket analysis}
%
The same conclusion can be made concerning the components interacting via network socket (the daemon and the user library). 
%
Received in \A such way data may be incorrect, thus it is necessary to estimate tolerance of these components. 
%
Because PeerHood source code is available, the easiest way to perform this operation is to revise the appropriate blocks of the source code. 

\Paragraph{Interacting with file system analysis}
%
Common components comprise configuration facilities. 
%
PeerHood stores it's own configuration in a specified file with a given format. 
%
Lack of this file or violation of it's format leads to PeerHood failure. 
%
The configuration file contains a file system path to network plug-ins loaded by the daemon. 
%
In the current daemon implementation there is no default value of this key, thus lack of the key leads to inability for network plug-ins loading and the daemon proper functioning. 

\Paragraph{Analysis of access rights to files used by PeerHood}
%
The current PeerHood implementation does not comprise a special install script. 
%
At the present, deployment of the application conists in a copy of it's libraries, executing modules, and resources to a specified directory. 
%
The necessary from the security point of view access rights to these files are not assigned. 

\Paragraph{Possibility to compromise PeerHood modules}
%
This can lead to PeerHood modules compromise. 
%
The consequence of this fact is a possibility of the application failure or a possibility of an attack performance on data transferred between the modules. 

\Paragraph{Configuration file access control analysis}
%
PeerHood also utilizes a text file to store it's own configuration there. 
%
The file is also readable and writable for other applications. 
%
Thus there is a risk related to an unathorized alteration of the file. 
%
The goal of this alteration can be, for example, a change of file system path to network plug-ins location with further malicious plug-ins loading by the daemon. 

\Paragraph{Conclusions}
%
Using this method concerning PeerHood, a vulnerability related to unsafe deployment of the project has been detected. 
%
Exploitation of the vulnerability by an attacker leads to violation of all PeerHood security components (see \ReferenceToSection{peerhood_analysis_security_testing_method_choosing}). 

% -------------------------------------------------------------------------------------------------------------------------

\SubSubSectionTitle{Source code revision}{peerhood_analysis_security_testing_code_inspection}

The goal of PeerHood source code revision is to search of vulnerabilities affecting it's security. 
%
According to \Reference{Michael2005} \Reference{NISTSP500268}, such process is comprised of the following steps: revision goals establishment, selection of potential vulnerability classes, and further their detecion in source code. 
%
The PeerHood revision goal consists in estimation of it's security components listed in \InReferenceToSection{peerhood_analysis_security_testing_method_choosing}. 
%
And particular attention is given to these blocks of code that corresponds to ``vulnerable'' places at that (see \ReferenceToSection{peerhood_analysis_risk_modeling}). 
%
Seven Pernicious Kingdoms vulnerability categories can be used as the such vulnerability classes (see \ReferenceToSection{software_security_vulnerabilities_seven_pernicious_kingdoms}). 
%
These are as follows: input data checking and representation, improper \Abbreviation{API} use, security mechanisms, time and state, errors, encapsulation, software environment \Reference{Tsipenyuk2005}. 

\Paragraph{Input data checking}
%
Input data checking category consists in the following vulnerabilities: buffer overflow, code injection, format string vulnerabilities, integer overflow, invalid pointer value \Reference{Tsipenyuk2005}. 
%
C/C++ arrays are not used in PeerHood source code with the exception considered in \InReferenceToSection{peerhood_analysis_security_testing_static_analysis}. 
%
More safe Qt alternatives are used instead of them: \SourceCode{QString}, \SourceCode{QList}, \SourceCode{QStringList}, \SourceCode{QByteArray} \WebSite{QtDocs}. 
%
Format strings are also not used in the project. 
%
Source code revision has shown that the project does not contain integer overflow errors, as well as improper pointer usage. 

\Paragraph{Security mechanisms}
%
The category related to security mechanisms contains vulnerabilities related to a full or particular lack of security mechanisms in a software (access control, authorization, authentification, protection of transferred data, and others) \Reference{Tsipenyuk2005}. 
%
According to the specification (see \ReferenceToSection{peerhood}), the mechanism are not yet implemented in PeerHood. 
%
But their implementation is planned and it is related to one of the further development directions of the project. 
%
Source code revision has shown that data are transferred in unencrypted form between mobile devices in the network environment. 
%
Thus it can be said that confidentiality and integrity of the data as PeerHood security components can be violated. 

\Paragraph{Errors}
%
The category related to errors contains the following vulnerabilities: double free of memory blocks, memory leakage, dereferencing of a NULL pointer \Reference{Tsipenyuk2005}. 
%
The current PeerHood implementation is delivered with a set of tests, including those that are intended for memory leakage detection. 
%
They pass successfully, thus it can be said that PeerHood does not contain memory leaks. 
%
This is proved by dynamic analysis results (see \ReferenceToSection{peerhood_analysis_security_testing_dynamic_analysis}). 
%
PeerHood source code revision has shown that errors related to NULL pointer dereferencing and double free of memory block are not found in the project. 

\Paragraph{Software environment}
%
The category related to software environment contains the following vulnerabilities: unsafe assignment of access rights to files used by a software, storing confidential information in unencrypted form \Reference{Tsipenyuk2005}. 
%
The PeerHood vulnerability related to access rights set to it's configuration file has been considered in \InReferenceToSection{peerhood_analysis_security_testing_fault_injection}. 
%
But there is another one that is related to the fact that the configuration file stores information about name of the mobile device and it's identifier. 
%
These values are used to differentiate devices in a network environment. 
%
Unauthorized modification of these parameters can lead to incorrect device identification. 
%
In this case an attacker can pose as a normal user and perform appropriate malicious operations. 
%
The consequence of this fact is a possible violation of confidentiality related to transferred data. 

\Paragraph{Encapsulation}
%
The category related to encapsulation contains the following vulnerabilities: information disclosure, encapsulation violation related to classes of an software \Reference{Tsipenyuk2005}. 
%
In the current PeerHood implementation, the daemon utilizes the hostname and the device identifier and transfers them to another devices in a network environment. 
%
Such behaviour from the security point of view can be considered as a fact of information disclosure because it helps an attacker to pose himself as another user in the network. 
%
The consequences of this fact has been already considered above in this subsection. 
%
Search of \Abbreviation{API} classes encapsulation violation is performed further in this subsection. 

\Paragraph{Improper API use}
%
The category related to improper API use deals with \Abbreviation{API} contract elements violation \Reference{Tsipenyuk2005}. 
%
Such operation can promote to incorrect functioning of an application, including unsafe behaviour. 
%
PeerHood \Abbreviation{API} security analysis enables to estimate fault-tolerance of it's components. 
%
Additionally it promotes to a search of vulnerabilities related to information disclosure. 

\Paragraph{Encapsulation violation}
%
As it was noted above, PeerHood utilizes the more safe alternatives to C/C++ arrays: \SourceCode{QList}, \SourceCode{QString} classes and others \WebSite{QtDocs}. 
%
According to Qt documentation \Reference{QtDocs}, these classes are not immutable, so their state can be changed at runtime. 
%
Including in a unauthorized manner, it may lead to some security problems. 
%
In \Abbreviation{OOP} languages it is expressed as violation of class encapsulation. 
%
The violation is related to modification of a field of a class from outside of the class, not in a method or a constructor of the class. 
%
From the security point of view, such vulnerability can lead to confidentiality violation, if a field is storing some confidential information. 
%
Also this can lead to violation of integrity, when a value of the field is modified in a unauthorized manner. 
%
Finally, this can lead to violation of availability, when a block of memory holding the field is freed in an unauthorized manner. 

\Paragraph{Encapsulation violation in PeerHood}
%
PeerHood \Abbreviation{API} contains a number of vulnerabilities related to it's class encapsulation violation. 
%
They are listed in \InReferenceToTable{peerhood_analysis_security_testing_api_encapsulation}. 
%
These vulnerabilities affect all the functional PeerHood characteristics except availability of simultenous several client serving and a capabiltiy of user control (see \ReferenceToSection{peerhood_analysis_security_testing_method_choosing}). 

\TableFigure{PeerHood API classes with violated encapsulation}{peerhood_analysis_security_testing_api_encapsulation}{
	\begin{tabular}{ | p{4cm} | p{10cm} | }
		\hline
		\Bold{Class} & \Bold{Methods/constructors} \\ \hline
		\SourceCode{Service} & \SourceCode{const QString\& name() \linebreak const QStringList\& attributes() \linebreak const QStringList\& attributes()} \\ \hline
		\SourceCode{AbstractConnection} & \SourceCode{const QString remoteAddress()} \\ \hline
		\SourceCode{AbstractCreator} & \SourceCode{const QString\& connectionBase()} \\ \hline
		\SourceCode{AbstractMonitor} & \SourceCode{const QString\& connectionBase() \linebreak const QString\& address()} \\ \hline
		\SourceCode{AbstractPinger} & \SourceCode{const QString\& connectionBase() \linebreak const QString\& address()} \\ \hline
		\SourceCode{Device} & \SourceCode{const QString\& name() \linebreak const QString\& prototype() \linebreak const QString\& address() \linebreak QList<Service\*>\& serviceList()} \\ \hline
		\SourceCode{AbstractAdverter} & \SourceCode{const QString\& connectionBase()} \\ \hline
	\end{tabular}
}

\Paragraph{Method arguments checking}
%
To provide fault-tolerance assurance for an application, it's components should perform input data checking. 
%
In case of \Abbreviation{API} function arguments act as such input data. 
%
Lack of input data checking by a component may lead to violation of it's functioning. 
%
From the security point of view, this may cause availability violation when input data has unexpected or incorrect value and they cause the application failure. 

\Paragraph{Vulnerable methods of PeerHood API classes}
%
Concerning PeerHood such vulnerabilities affect it's functioning. 
%
They affect it's functional characteristics related to availability and capability (see \ReferenceToSection{peerhood_analysis_security_testing_method_choosing}). 
%
The methods related in PeerHood \Abbreviation{API} classes are listed in \InReferenceToTable{peerhood_analysis_security_testing_unsafe_api_methods}. 
%
They do not verify values of their arguments. 
%
Source code revision has shown that PeerHood do not check input data, including transferred data. 
%
In particularly, deserialization of \SourceCode{Service} and \SourceCode{Device} classes is unsafe. 

\TableFigure{PeerHood API classes that do not check input data}{peerhood_analysis_security_testing_unsafe_api_methods}{
	\begin{tabular}{ | p{4cm} | p{10cm} | }
		\hline
		\Bold{Class} & \Bold{Methods/constructors} \\ \hline
		\SourceCode{Service} & \SourceCode{void setPort(unsigned int) \linebreak Service(const QString\& name, const QStringList\& attributes, unsigned int port, quint64 pid = 0, QObject\* parent=0)} \\ \hline

		\SourceCode{AbstractConnection} & \SourceCode{AbstractConnection(const QString\& connectionBase, const QHostAddress\& address = QHostAddress::Any, QObject\* parent = 0) \linebreak bool listen(quint16 port)} \\ \hline
		\SourceCode{AbstractCreator} & \SourceCode{AbstractMonitor\* createMonitor(const QString\& address) \linebreak AbstractPinger\* createPinger(const QString\& address)} \\ \hline
		\SourceCode{AbstractMonitor} & \SourceCode{AbstractMonitor(const QString\& address, const QString\& connectionBase)} \\ \hline
		\SourceCode{AbstractPinger} & \SourceCode{AbstractPinger(const QString\& address, const QString\& connectionBase)} \\ \hline
		\SourceCode{Device} & \SourceCode{Device(QDataStream\& stream, QObject \*parent = 0)} \\ \hline
		\SourceCode{PeerHood} & \SourceCode{int registerService(const QString\& name, const QStringList\& attributes = QStringList(), unsigned int port = 0) \linebreak bool unregisterService(const QString\& name, unsigned int port = 0) \linebreak void connectNotify(const char\* signal) \linebreak void disconnectNotify(const char\* signal)} \\ \hline
	\end{tabular}
}

\Paragraph{Capability of user control and simultaneous several clients serving}
%
PeerHood user control consists in a capability of changing it's state, for example, transition of low power consumption state. 
%
From the security point of view, \Abbreviation{API} components responsible for this functionality are implemented correctly. 
%
The same can be said about simultaneous serving of several clients. 
%
The feature is implemented correctly by means of synchronization primitives (\SourceCode{QMutex} \WebSite{QtDocs}). 
%
Revision of the appropriate blocks of PeerHood source code responsible for utilization of resources has not detected any race condition. 