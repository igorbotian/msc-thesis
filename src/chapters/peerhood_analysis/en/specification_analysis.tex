\Paragraph{What PeerHood specification contains}
%
PeerHood specification contains the following information. 
%
The project goals define that should the software must to do. 
%
The functional requirements define which functional characteristics should be have PeerHood to achieve it's goals. 
%
Non-functional requirements define a structure of the project. 

% ---------------------------------------------------------------------------------------------------------------

\SubSubSectionTitle{PeerHood goals analysis}{peerhood_analysis_specification_analysis_goals_analysis}

\Paragraph{Goals}
%
The PeerHood goal consists in personal communication support in a mobile \Abbreviation{P2P} environment regardless of underlying network technologies. 
%
It is achieved at the expense of proactivity, connectivity and reactivity support (see \ReferenceToSection{peerhood_concept}). 

\Paragraph{Connectivity analysis}
%
Connectivity support composes a logical level of the applicatio that is responsible to seamless communication between mobile devices. 
%
According PeerHood concept, it is needed to link ``Network technologies'' elements with each other (see \ReferenceToFigure{peerhood_concept}). 

\Paragraph{Reactivity analysis}
%
Reactivity support composes a logical level of the application that is responsible to communication between user services. 
%
According to PeerHood concept, it is needed to link ``User applications'' elements with each other (see \ReferenceToFigure{peerhood_concept}). 

\Paragraph{Proactivity analysis}
%
Proactivity support composes a logical level of the application that is responsible to connect the levels described above. 
%
According to PeerHood concept, it is needed to link ``Network technologies'' elements with ``Middleware modules'' elements and ``User applications'' elements with ``Middleware modules'' (see \ReferenceToFigure{peerhood_concept}). 

\Paragraph{Summary}
%
Thus correctness of all PeerHood components functioning on each logical level defines a possibility of successful solution of all tasks targeted by the project. 
%
And their fault-tolerance defines a possibility of correct PeerHood functioning under attack. 

% ---------------------------------------------------------------------------------------------------------------

\SubSubSectionTitle{PeerHood requirements analysis}{peerhood_analysis_specification_analysis_requirement_analysis}

\Paragraph{PeerHood requirements analysis}
%
Correctnes of each PeerHood component functioning is defined by meeting of all requirements related to the component, functional and non-functional. 
%
And all requirements in the aggregate aim at the project goals achievement. 

\Paragraph{Connectivity}
%
Connectivity is achieved by meeting the following requirements: seamless communication, connection establishment, data transfer between devices, architecture of network plug-ins, and network management.
%
It's achievement conists in seamless data exchange between devices with a possibility of switching between available network technologies. 
%
It is achieved via network plug-ins, each of them is aimed at support of a specified network technology, it has special interface for performing network-specific operations (data transfer, connection establishment, and others). 

\Paragraph{Proactivity}
%
Proactivity is achieved by meeting of the following requirements: device discovery, device detection, component management, and provision of the basis for simultaneous client serving. 
%
It's achievement consists in provision of up-to-date information about the network environment to one or more client. 
%
It is achieved via middleware modules that connect components of other levels. 
%
The modules provide information about the network environment to high level components and notify them about changes. 
%
Low level component in their turn are utilized by the modules to perform data transfer directly and to process received from they notifications about network events. 

\Paragraph{Reactivity}
%
Reactivity is achieved by meeting the following requirements: service discovery, shared service utilization, user control, provision of client interface to receive events. 
%
It's achievement conists in shared both local and remote services utilization and the system control by user applications. 
%
It is achieved via high level modules that are aimed at execution of user operations within the concept of the services.

\Paragraph{PeerHood functional characteristics}
%
Realization of the fact how all PeerHood components are functioning enables to define a set of functional characteristics of the project. 
%
The charactestics affecting successul each PeerHood goal achievement are given in \InReferenceToTable{peerhood_assets}: proactivity, connectivity, and reactivity. 
%
Their analysis enables to define components of information security concering PeerHood. 

\TableFigure{PeerHood functional characteristics}{peerhood_assets} {
	\begin{tabular}{ | p{4cm} | p{10cm} | }
	  \hline                       
	  \Bold{Goal} & \Bold{Characteristics} \\ \hline
	  Connectivity & data transfer, retrieving of up-to-date information about devices in the network environment (using different network technologies) \\ \hline
	  Proactivity & retrieving of up-to-date information about available devices in the network environment data exchange between devices,  simultaneous serving of several clients (local or remote) \\ \hline
	  Reactivity & retrieving of up-to-date information about available devices in the network environment (local or remote), user data transfer, user control \\ \hline
	\end{tabular}
}

\Paragraph{PeerHood security components}
%
From the security point of view, influence on each functional characteristic is \T{arisen} depending on it's nature. 
%
It depends on the fact how the characteristic is related to each CIA Triad principle: availability, confidentiality, and integrity. 

\Paragraph{Availability concering PeerHood}
%
Availability is related to the following functional characteristics: retrieving of up-to-date information about devices and services in the network environment, simultaneous serving of several clients, as well as user control. 
%
It can be explained by the fact that fault-tolerance of these characteristics is related to the fact of a possibility of information retrieving, user control, simultaneous serving of several clients. 

\Paragraph{Confidentiality concerning PeerHood}
%
Confidentiality is related to the following functional characteristics: data transfer between both the system components and devices. 
%
According to PeerHood concept, sent user data are transferred on all levels of the application (middleware modules, network plug-ins, network). 
%
Thus there is a possibility to intercept the data. 

\Paragraph{Integrity concerning PeerHood}
%
Integrity is related to the all functional characteristics except user control and simultaneous serving of several clients. 
%
They are related to user data transfer or information about services and devices. 
%
Thus there is a possibility of their modification, fabrication, and other types of attacks. 

\Paragraph{Access to system and user data}
%
It should be noted that each PeerHood component can potentionally have an access to system files, including user data. 
%
This facut in its turn also affects the project security because the data can be corrupted or hijacked. 
%
Thus a potentional access to system files is related to the problem of their integrity and confidentiality. 

\Paragraph{Transition to PeerHood risk modeling}
%
Violation of any CIA Triad principle considered above leads to PeerHood security non-assurance. 
%
There is a lot of security risks that affects the each principle. 