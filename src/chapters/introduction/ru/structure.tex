\Paragraph{Теоретическая часть работы}
%
Первые два раздела составляют теоретическую часть данной работы.
%
Раздел 2 посвящён проблеме безопасности ПО и содержит вводную информацию о предметной области.
\begin{comment}
- понятие безопасности ПО
- истоки проблемы безопасности ПО
- дефекты безопасности ПО
- угрозы безопасности ПО
- атаки на ПО
- уязвимости ПО
- угрозы безопасности ПО в мобильном окружении
- угрозы безопасности ПО в пиринговом окружении
\end{comment}
%
%
В разделе 3 рассматриваютя способы проведения анализа безопасности \RussianAbbreviation{ПО}, необходимые для проведения практической части исследования.
\begin{comment}
- обеспечение безопасности ПО
- понятие безопасного ПО
- тестирование ПО на безопасность
- тестирование по методу белого ящика
- тестирование по методу серого ящика
- тестирование по методу чёрного ящика
\end{comment}
%
%\Paragraph{Практическая часть работы}
%
Следующие два раздела относятся к практической части данной работы.
%
В разделе 4 кратко описывается сетевое окружение \IT{PeerHood}, исследуемое в данной работе. 
%
Рассмотрены концепция, цели, ключевые требования и архитектура данного проекта.
%
Раздел 5 содержит анализ проекта \IT{PeerHood} не безопасность.
%
%\Paragraph{Заключение}
%
В заключительном разделе содержатся общие выводы по данной работе, список нерешённых задач и рекомендации по перспективе дальнейшей работы.
%
\begin{comment}
Глаголы, которые можно использовать для того, чтобы описать, что содержится в разделах диссертации:
- рассматривает - considers
- отражает компоненты - represents
- рассматривается - to be considered, studies
- описывает - describes
- включает - includes
- предназначен для - to be intended for 
- даётся - deals with
- знакомит - introduces
- на чём будет сфокусирован - will focus on
- начинается и заканчивается - begins and concludes
- обрисовывает - outlines
- содержит - to be comprised of
- объясняет - explains
- состоит из подразделов - is consist of
- посвящён - dedicates to
- подводит итог - summarizes
- исследуется - to be studied
\end{comment}
