\begin{remarks}
	Тезисы:
	\begin{itemize}
		\item Рост мощностей вычислительных устройств за последние 10-15 лет, в том числе мобильных
		\item Появления новых видов мобильных устройств пользовательского уровня
		\item Рост появления новых технологий (передачи данных, связи и др.)
		\item Одновременное появление новых парадигм коммуникации (P2P и др.)		
		\item Современная коммуникационная среда отличается высоким уровнем гетерогенности её 
			составляющих элементов
		\item Картинка, иллюстрирующая ситуацию

		\item В прошлом мобильные устройства имели строго определённый функционал и делились на два 
			типа: хранение и обработка пользовательских данных и обеспечение связи
		\item На данный момент существует тенденция универсальности мобильных устройств
		\item Стремление производителей ПО к универсализации их платформ и ОС
		
		\item Универсализация мобильных устройств, технологий и ПО приводит к тому, что, 
			образно говоря, устройство является представителем пользователя в виртуальной среде
		\item Концепция доверенного персонального устройства
			
		\item За универсализацией как устройств, так и технологий последовала тенденция к 
			универсализации производителям программного обеспечения (Microsoft, Google, Canonical, 
			Apple), используемого на данных устройствах
		\item Следствием такой универсализации является повышенное внимание к безопасности личных
			пользовательских данных
		
		\item Лёгкость, спонтанность и кратковременность установления связи между двумя и более
			пользователями с помощью мобильных устройств
		\item Независимость концепции персонального доверенного устройства от коммуникационного 
			окружения и используемых технологий
		\item Появление новых интернет-сервисов, основывающихся на принципиально новой модели 
			взаимодействия с пользователем (Google Services, Skype, DropBox)
		\item Следствием универсализации коммуникационной платформы является то, что персонализация
			становится одним из наиболее перспективных направлений в области предоставления сервисов
			(социальные сети, Google Plus, Google AdSense, и пр.)
		\item Рост популярности локальных взаимодействий с помощью пирингового подхода
		\item PeerHood предоставляет свой подход к локальному взаимодействию в мобильном окружении
		\item Необеспечение безопасности при использовании мобильного пользовательского устройства
			может привести к несанкционированному доступу к конфиденциальным данным пользователя,
			к внедрению вредоносного ПО, к отказу устройства в работе, к удалённому контролю над
			работой устройства
		\item В данной работе рассматривается вопрос безопасности ПО в мобильном пиринговом окружении
	\end{itemize}
\end{remarks}