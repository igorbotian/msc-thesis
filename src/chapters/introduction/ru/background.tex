\Paragraph{Появление новых видов мобильных устройств}
\Sentence
Рост мощностей вычислительных устройств, в том числе и мобильных, за последние 10-15 лет привёл 
к появлению новых видов мобильных устройств.
\Sentence
Раньше каждое мобильное устройство относилось к какой-либо категории, исходя из его функциональных 
возможностей. Например, оно могло либо обеспечивать коммуникацию, либо хранить и обрабатывать 
пользовательские данные, либо выполнять действия другого вида.
\Sentence
Тогда как современные устройства отнести к какой-либо категории можно лишь \Important{условно}.

\Paragraph{Универсализация современных мобильных устройств}
\Sentence
Причиной этому является тот факт, что современное мобильное устройство по своим возможностям 
предоставляет функционал, который ранее могли предоставить только мобильные устройства, относящиеся 
к различным категориям.
\Sentence
Примером такого устройства может являться современный \Emphasis{смартфон}, используемый 
пользователем как в качестве средства коммуникации, так и в качестве средства хранения его личных 
данных.
\Sentence
Тогда как раньше для выполнения данных операций пользователь был вынужден использовать несколько 
устройств: \Emphasis{мобильный телефон} и \Emphasis{\RussianAbbreviationShort{КПК}}. 
\Sentence
Это является основой современной тенденции к \Important{универсализации} мобильных устройств.

\Paragraph{Рост пирингового подхода в области мобильных коммуникаций}
\Sentence
Одновременно с ростом вычислительной мощности мобильных устройств широкое распространение 
в области коммуникаций получили и новые сетевые технологии.
\Sentence
К примеру, поддержка технологии Bluetooth \WebSite{Bluetooth} привела к росту популярности 
локальных взаимодействий между пользователями.
\Sentence
Такие локальные взаимодействия характеризуются лёгкостью, спонтанностью и кратковременностью 
установления связи, что привело к большой популярности использованися 
\Emphasis{пиринговой сетевой парадигмы} \Reference{Schollmeier2001}.
\Sentence
Пиринговый подход способствует совместному использованию ресурсов мобильного окружения и является
\Important{независимым} от технологии взаимодействия, так как он находится на его верхнем уровне 
взаимодействия \Reference{PorrasP2P2004}, \Reference{PorrasEnhancing2005}.

\begin{comment} % параграф не несёт важной смысловой нагрузки
\Paragraph{Гетерогенность мобильного пирингового окружения}
\Sentence
Стоит отметить, что следствием процесса коммуникации между различными видами мобильных устройств 
является повышение уровня гетерогенности мобильной коммуникационной среды.
\Sentence
Это может привести к появлению потенциальных проблем, например, к невозможности коммуникации
между мобильными устройствами, имеющими поддержку разных сетевых технологий.
\Sentence
Но в настоящий момент проблемы такого рода решены путём универсальности мобильных устройств в целом, 
в частности, поддержкой устройствами только стандартизированных сетевых технологий 
\WebSite{Bluetooth}, \WebSite{WiFi}.
\end{comment}

\Paragraph{Мобильное устройство как представитель пользователя в виртуальной среде}
\Sentence
Следствием универсализации современных мобильных устройств и сетевых технологий является 
\Important{универсализация} программного обеспечения, используемого на данных устройствах.
\Sentence
Яркими примерами являются использование одной и той же программной платформы на различных мобильных 
устройствах (\IT{Apple iOS} \WebSite{iOS}, \IT{Google Android} \WebSite{Android}), а также 
направленность производителей программного обеспечения на унификацию интерфейса пользователя на 
разных устройствах (\IT{Microsoft Metro} \WebSite{MetroUI}, \IT{Ubuntu Unity} \WebSite{Unity}).
\Sentence
Универсализация мобильных устройств в целом с одной стороны приводит к уменьшению числа компьютерных 
устройств, используемых пользователем, а с другой стороны к тому, что используемое им мобильное 
устройств начинает выступать в качестве \Emphasis{персонального представителя пользователя} в 
виртуальной среде.

\Paragraph{Концепция персонального доверенного устройства}
\Sentence
Использование мобильного устройства пользователем в качестве такого представителя приводит к 
появлению понятия \Emphasis{персонального доверенного устройства (\RussianAbbreviationShort{ПДУ})} 
\Reference{PorrasPTD2004}.
\Sentence
Концепция \RussianAbbreviation{ПДУ} основывается на том, что пользовательские данные 
активно используются в рамках коммуникации, производимой данным пользователем.
\Sentence
Причём стоит отметить, что процесс коммуникации является \Important{независимым} от 
коммуникационного окружения и используемых технологий.
\Sentence
В свою очередь, недостаточное обеспечение безопасности личных данных пользователя может привести 
к различного рода проблемам \Reference{Mavridis1997}:
\begin{itemize}
	\item несанкционированный доступ к конфиденциальным данным пользователя;
	\item внедрение вредоносного \RussianAbbreviation{ПО};
	\item отказ устройства в работе;
	\item удалённый контроль над работой устройства.
\end{itemize}

\Paragraph{Важность безопасности личных данных пользователя - основная предпосылка}
\Sentence
В условии универсализации коммуникационной платформы \Emphasis{персонализация} является одним из 
наиболее перспективных направлений в области предоставления сервисов \Reference{PorrasP2P2004}.
\Sentence
Поэтому вопрос безопасности личных данных пользователя в условиях является одним из 
\Important{ключевых}.
\Sentence
Широкое использование мобильных пиринговых сетей и важность безопасности личных данных пользователя 
являются \Important{основными предпосылками} для проведения исследования, рассматриваемого 
в данной работе.