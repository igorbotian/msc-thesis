\Paragraph{Безопасность ПО}
\Sentence
Учёт безопасности происходит на каждом этапе жизненного цикла разработки 
\RussianAbbreviation{ПО} \Reference{Goertzel2007}. 
\Sentence
Для этого на каждом из них проводятся соответствующие мероприятия 
\Sentence
Этап разработки исследуемого в данной работе проекта \PeerHood почти завершён 
\Reference{Kolehmainen2010}. 
\Sentence
Поэтому практическая часть работы преимущественно лежит области тестирования данного 
\RussianAbbreviation{ПО} на безопасность.

\Paragraph{Анализ PeerHood}
\Sentence
Анализ проекта \PeerHood состоит только в поиске уязвимостей его безопасности.
\Sentence
Никаких дополнительных действий, связанных с найденными уязвимостями, в данной работе 
не производится.

\Paragraph{Инструментальные средства}
\Sentence
Практическая часть работы проводится в среде GNU/Linux \WebSite{Linux}, следствием чего является 
сужение набора используемых для анализа инструментальных средств.
\Sentence
В процессе анализа используются только те средства, которые доступны на данной платформе.

\Paragraph{Анализ PeerHood}
\Sentence
Проект \PeerHood написан на языке программирования С++ \Reference{CppStandard2003} и взаимодействие 
с \RussianAbbreviation{ОС} осуществляет через фреймворк \Qt \WebSite{Qt}.
\Sentence
Поэтому при изучении уязвимостей безопасности \RussianAbbreviation{ПО} основное внимание направлено 
на те уязвимости, которые связаны с данными технологиями.