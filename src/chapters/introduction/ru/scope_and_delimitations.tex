\Paragraph{Безопасность ПО}
%
Учёт безопасности происходит на каждом этапе жизненного цикла разработки 
\RussianAbbreviation{ПО} \Reference{Goertzel2007}. 
%
Для этого на каждом из них проводятся соответствующие мероприятия 
%
Этап разработки исследуемого в данной работе проекта \EnglishText{PeerHood} почти завершён 
\Reference{Kolehmainen2010}. 
%
Поэтому практическая часть работы преимущественно лежит области тестирования данного 
\RussianAbbreviation{ПО} на безопасность.

\Paragraph{Анализ PeerHood}
%
Анализ проекта \EnglishText{PeerHood} состоит только в поиске известных уязвимостей безопасности.
%
Никаких дополнительных действий, связанных с найденными уязвимостями, в данной работе не производится.

\Paragraph{Инструментальные средства}
%
Практическая часть работы проводится в среде GNU/Linux \WebSite{Linux}, следствием чего является сужение набора используемых для анализа инструментальных средств.
%
В процессе анализа используются только те средства, которые доступны на данной платформе.

\Paragraph{Анализ PeerHood}
%
Проект \EnglishText{PeerHood} написан на языке программирования С++ \Reference{CppStandard2003} и взаимодействие 
с \RussianAbbreviation{ОС} осуществляет через фреймворк \EnglishText{Qt} \WebSite{Qt}.
%
Поэтому при изучении уязвимостей безопасности \RussianAbbreviation{ПО} основное внимание направлено на те уязвимости, которые связаны с данными технологиями.