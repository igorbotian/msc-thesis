\Paragraph{Факторы обеспечения безопасности данных пользователя}
\Sentence
Для обеспечения безопасности личных данных пользователя необходимо учитывать ряд факторов.
\Sentence
К человеческому фактору относится то, как пользователь использует свои личные данные в процессе
коммуникации
\Sentence
К техническим факторам относятся следующие:
\begin{itemize}
	\item безопасность технологий передачи данных, поддерживаемых мобильным устройством;
	\item безопасность \RussianAbbreviation{ПО}, используемого при работе над личными данными 
		пользователя, а также в процессе коммуникации;
	\item 
	\item и другие
\end{itemize}

\Paragraph{Субъект, объект и вопрос исследования}
\Sentence

\Paragraph{PeerHood - практическая часть исследования}
\Sentence
Помимо теоретического части данная работа также содержит практическую часть исследования, которая 
включает в себя анализ \TODO{мобильной пиринговой среды} PeerHood.

\Paragraph{Основные цели}
\Sentence
Для того чтобы получить ответ на поставленный вопрос, то есть достигнуть основной цели проведения 
данного исследования, необходимо решить следующие задачи:
\begin{enumerate}
	\item Рассмотреть область безопасности программного обеспечения: общие понятия, факторы, 
		основные действия по обеспечению, последствия.
	\item \TODO{Угрозы безопасности ПО в целом}
	\item \TODO{Угрозы безопасности в мобильной среде}
	\item \TODO{Угрозы безопасности в пиринговой среде}
	\item \TODO{Обобщить угрозы безопасности для мобильной пиринговой среды}
	\item \TODO{Рассмотреть проект PeerHood}
	\item \TODO{Провести анализ проекта PeerHood (результат - неучтённые угрозы безопасности 
		для данного проекта)}
\end{enumerate}

\Paragraph{Переход к ограничениям}
\Sentence
Что не делается в данной работе
