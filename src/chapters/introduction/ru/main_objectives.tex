\Paragraph{Факторы обеспечения безопасности данных пользователя}
%
Безопасность личных данных пользователя зависит от ряда как человеческих, так и технических 
факторов \Reference{Winograd2008}.
%
К техническим можно отнести следующие: безопасность технологий передачи данных, 
поддерживаемых мобильным устройством пользователя; безопасность \RussianAbbreviation{ПО}, которое 
выполняет операции коммуникации и имеет доступ к личным данным пользователя; ряд других.

\Paragraph{Субъект, объект и вопрос исследования}
%
Объект исследования данной работы связан с уязвимостями безопасности \RussianAbbreviation{ПО}, 
которое используется пользователем на его \RussianAbbreviation{ПДУ} для проведения коммуникации. 
%
Предметом исследования является сетевое окружение \PeerHood, основанное на концепции пирингового 
взаимодействия \RussianAbbreviation{ПДУ} в мобильной среде \Reference{PeerToPeerNeighborhood}.
%
Поэтому вопрос исследования заключается в поиске уязвимостей \RussianAbbreviation{ПО} в данном 
окружении.

\Paragraph{Основные цели}
%
Для того чтобы получить ответ на поставленный вопрос, необходимо решить следующие задачи:
\begin{enumerate}
	\item рассмотреть общие понятия безопасности программного обеспечения;
	\item исследовать проблему уязвимостей безопасности \RussianAbbreviation{ПО};
	\item изучить существующие подходы, методы и способы поиска уязвимостей безопасности 
		\RussianAbbreviation{ПО};
	\item тщательно изучить исследуемое \RussianAbbreviation{ПО} \PeerHood;
	\item провести анализ проекта \PeerHood c целью поиска в нём уязвимостей безопасности 
		\RussianAbbreviation{ПДУ}.
\end{enumerate}

\Paragraph{Переход к ограничениям}
%
Тема, связанная с безопасностью \RussianAbbreviation{ПО}, является довольно объёмной.
%
Существует целый ряд аспектов для её рассмотрения. 
\Sentnce
Данный факт существенно влияет на результаты проводимого в данной работе исследования.
%
По этой причине автору необходимо очертить границы проводимого исследования. 
%
Они рассматриваются в следующем подразделе. 

\begin{comment}
Автор должен сформулировать цели и задачи работы, описать объект и предмет исследования, показать 
новизну работы и ее практическую значимость. Формулировка целей заключает в себе ответ на вопрос 
«Что нужно сделать?», а формулирование задач отвечает на вопрос «Как нужно действовать, чтобы 
достичь этой цели?». Задачи обычно даются в форме перечисления (изучить..., описать..., 
установить..., выявить..., вывести формулу..., разработать,...представить... и т. п.).

Объект исследования – это процесс или явление, порождающее проблемную ситуацию и избранное для 
изучения, например. «Оценка профессиональной подготовки студента-выпускника», «Информационные
потоки в ОАО Ленэнерго». В объекте выделяется та его часть, которая служит предметом исследования. 
На предмет исследования направлено основное внимание исследователя, именно предмет исследования 
определяет тему работы, которая выносится на титульный лист как ее заглавие, например. 
«Моделирование рейтинговой системы оценки знаний студентов», «Создание модели клиент-серверного 
приложения для автоматизации документооборота в СПб ГЭТУ», «Разработка интерфейса html-docmyna к
SQL-серверу на платформе ОС UNIX», «Создание программного пакета лексического анализатора (ЛА) 
под Windows».

Новизна работы может быть научной (концепция, гипотеза, закономерность и т.д.) и практической 
(правило, предложение, рекомендации, средство, требование, методическая система и т. д.).
Обычно она заключается в следующем:
- новом объекте исследования;
- использовании новых подходов, методик, методов исследования;
- решении вопросов, отражающих специфику и особенности региона;
- получении нового знания, являющегося результатом обобщения и критического анализа источников 
литературы.

Практическая значимость работы – это возможность внедрения ее результатов с указанием конкретных 
областей применения.

INSTRUCTIONS FOR WRITING A MASTER'S THESIS

Express the goals for the work, include also the delimitations. This way the reader knows
when the results are valid and she can place the work in a proper framework and scope. 
It is also important to say, what is not done during the work for the thesis. Then the
thesis will show how the goals are met. In the thesis this subsection occupies from 1 to 2
pages.
\end{comment}