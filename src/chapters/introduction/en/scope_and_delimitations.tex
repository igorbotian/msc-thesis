\Paragraph{Software security}
%
Security assurance is performed on every software development stage \Reference{Goertzel2007}. 
%
Appropriate activities are conducted to provide it. 
%
The current development stage of PeerHood project is implementation of it's functionality characteristics. 
%
And it is almost completed. 
%
So \The practical part of the work mainly related to software security testing area. 

\Paragraph{PeerHood analysis}
%
PeerHood project analysis consists only in search of known security vulnerabilities. 
%
No additional activities related to \The found vulnerabilities do not conduct in the work. 

\Paragraph{Tools}
%
The practical part of the work is performed in GNU/Linux \WebSite{Linux} environment. 
%
It influences on a number of applied tools used for performing of the software security analysis. 
%
Only those tools are utilized during the analysis that are available on this platform. 

\Paragraph{PeerHood characteristics}
%
PeerHood project is implemented in C++ \Reference{CppStandard2003} programming languange and interacts with \Abbreviation{OS} through Qt framework \WebSite{Qt}. 
%
Therefore, those vulnerabilities that related to the technologies are focused during studying of software security concept. 