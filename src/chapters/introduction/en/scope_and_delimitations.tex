\Paragraph{Software security}
%
Security assurance is performed on each stage of software development lifecycle \Reference{Goertzel2007}. 
%
Appropriate activities are conducted to provide it. 
%
The current development stage of PeerHood project is implementation of it's functionality characteristics, and it is almost completed. 
%
So a practical part of the work mainly related to software security testing area. 

\Paragraph{PeerHood analysis}
%
PeerHood project analysis consists only in search of known security vulnerabilities. 
%
No additional activities related to the found vulnerabilities is not conducted in the investigation. 

\Paragraph{Tools}
%
The practical part of the work is performed in GNU/Linux \WebSite{Linux} environment. 
%
It influences on a number of applied tools used to perform software security analysis. 
%
Only those tools are utilized during the analysis that are available on this platform. 

\Paragraph{PeerHood characteristics}
%
PeerHood is implemented in C++ programming languange \Reference{CppStandard2003} and interacts with \Abbreviation{OS} through Qt framework \WebSite{Qt}. 
%
Therefore, those vulnerabilities that related to these technologies are focused during studying of the software security concept. 