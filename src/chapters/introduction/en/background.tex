\Paragraph{Emergence of new types of mobile devices}
%
Increase of power of computation devices including mobile devices for the last 10-15 years led to \The emergence of new types of mobile devices. 
%
Previously each mobile device was related to one or more categories in terms of it's functionality. 
%
For example, it could provide communication functionality, store and process user data, or perform any other type of operations. 
%
Whereas it is hard to refer a contemporary mobile device to any category. 

\Paragraph{Universalization of contemporary mobile devices}
%
The main reason of this fact is that a contemporary mobile device provide the same set of functionalities that different categories of mobile devices could provide in the past. 
%
For instance, contemporary smartphone can be used by \A user to communicate or to store it's personal data. 
%
Whereas \A user had to use several devices, mobile phone and \Abbreviation{PDA}, to perform both of these operations. 
%
This fact is the basis of \The contemporary trend towards the universalization of contemporary movile devices. 

\Paragraph{Popularity of peer-to-peer approach in mobile communications}
%
At the same new communication technologies are widely used. 
%
For example, support of Bluetooth \WebSite{Bluetooth} technology by mobile devices increased popularity of local interactions between users. 
%
Such kind of interactions are charecterized by ease, spontaneity, and short time of communication setup. 
%
These factors led to popularity of peer-to-peer network paradigm \Reference{Schollmeier2001}. 
%
Peer-to-peer approach favours shared utilization of mobile environment resources and does not depend on underlying communication technology \Reference{PorrasP2P2004} \Reference{PorrasEnhancing2005}. 

\Paragraph{Mobile device as a user representative in virtual environment}
%
The main consequence of the universalization of contemporary mobile devices and technologies is the universalization of software utilized on these mobile devices. 
%
Clear examples are utilization of the same software platform on different mobile devices (Apple iOS \WebSite{iOS}, Google Android \WebSite{Android}) as well as orientation of software platform manufacturers to support one unified user interface on different devices (Microsoft Metro \WebSite{MetroUI}, Ubuntu Unity \WebSite{Unity}). 
%
Mobile device universalization on the one hand leads to decrease of a number of computer devices used by a user and on the other hand it leads to the fact that a mobile device becomes to be as a user personal representative in virtual environment. 

\Paragraph{Personal trusted device concept}
%
Mobile device utilization as a such representative leads to the emergence of \Abbreviation{PTD} idea \Reference{PorrasPTD2004}. 
%
The \Abbreviation{PTD} concept is based on the fact that user data is actively utilized within communication process performed by the user. 
%
At the same, communication does not depend on it's environment and used technologies. 
%
Insufficient personal user data security assurance could lead to various problems \Reference{Mavridis1997}:
\begin{itemize}
	\item unathorized access to personal user data
	\item malicious software injection
	\item denial of service
	\item remote control of a device
\end{itemize}

\Paragraph{Significance of private user data security}
%
In the conditions of the communication platform universalization\T{,} personalization is one of the most promising directions in service providing area. 
%
So it can be said that the personal user data security problem can be considered as one of the most critical problems in the area. 
%
Two factors, broad use of mobile peer-to-peer networks and importance of personal user data security assurance, are the main preconditions for the research conducted in this work. 