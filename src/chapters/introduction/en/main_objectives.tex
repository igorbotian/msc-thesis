\Paragraph{User data security assurance factors}
%
Personal user data security depends on a number of factors, both human and technical \Reference{Winograd2008}. 
%
Technical factors are the following: security of data transfer technologies supported by user mobile device; security of software that performes communication operations and has access to private user data; and a number of others. 

\Paragraph{Subject, object and question of the research}
%
The \T{object} of the research is related to security vulnerabilities of software utilized by a user on his \Abbreviation{PTD} to perform communication operations. 
%
The \T{subject} of the research is a network environment, based on the concept of peer-to-peer communication between \Abbreviation{PTD}'s in a mobile environment \Reference{PeerToPeerNeighborhood}. 
%
So the question of the research is the following: does PeerHood contain security vulnerabilities or not. 

\Paragraph{Main goals}
%
It is necessary to solve the following tasks to answer the question of the research: 
\begin{enumerate}
	\item to study the software security problem
	\item to consider existing methods of software security analysis
	\item to examine PeerHood project
	\item to perform PeerHood analysis to find software security vulnerabilities related to \Abbreviation{PTD}
\end{enumerate}

\Paragraph{Transition to delimitations}
%
Software security area is quite extensive. 
%
There is a number of aspects for examination of the area. 
%
This fact considerably influences on \The results of the research conducted in the work. 
%
For this reason, it is necessary to define a scope of the research. 
%
The scope is described below. 
