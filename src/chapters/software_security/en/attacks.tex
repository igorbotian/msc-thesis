\Paragraph{Definition}
%
Attack on software is an an intentional act by which an entity attempts to evade security services and violate the security policy of a system \Reference{RFC4949}. 
%
A person conducting such kind of attacks is called malicious person, or threat agent \Reference{Stallings2008}. 
%
Purposes of the attack carrying out can be the following: gathering or destruction of information system resources (assets) or information itself and denial of service \Reference{NISTGlossary2011}. 
%
At this moment there are various types of the attacks, and also classifications. 

\Paragraph{Classification of the attacks by their nature}
%
Attacks are classified by their nature and have the following types: interception, modification, falsification, and interference \Reference{Talukder2007}. 
%
Interception is an unathorized party gaining access to an asset. 
%
Modification is an unauthorized party gaining control of an asset and tampering with it. 
%
Falsification is an unauthorized party inserts counterfeited objects into the system. 
%
Interference is occured when an asset is destroyed or made unusable. 

\Paragraph{Classification of attacks by their principle of operation}
%
Attacks are also classified by their principle of operation and have the following types \Reference{Goertzel2007}. 
%
Reconnaissance-attacks help a malicious person to gather more important information about an attacked system and it's environment. 
%
Enabling-attacks are referred to actions enabled a malicious person to carry out another types of attacks. 
%
Disclosure-attacks are aimed at obtain confidential data. 
%
Subversion-attacks are related to change of attacked system workflow. 
%
Sabotage-attacks have one of the following goals: denial of service or denial of access for normal users. 

\Paragraph{Classification of attacks by moment of a vulnerability appearance}
%
Attacks are also classified by development lifecycle stages: when an appropriate error led to possibility of an attack performance has been made \Reference{Graff2003}. 
%
If it has been made on design stage then the following types of attacks may be performed: man-in-the-middle, race conditions, replay-attacks, sniffer-attacks, and others \Reference{Stallings2008}. 
%
If an error has been made during code writing, then attacks related to buffer overflow or input data check, as well as use of a backdoor, may be performed \Reference{Langweg2004}. 
%
Finally, an attack may be performed owing to unsafe software deployment. 
%
The most common type of such attacks is DoS-attack (Denial of Service) \Reference{Stallings2008}. 

\Paragraph{What is an exploit}
%
Attacks on software are performed owing to exploits. 
%
Exploit is a method which enables a malicious person to successfully perform an attack \Reference{Erickson2003}. 
%
It can be a special formed shellcode, an application or just a set of instructions \Reference{NISTGlossary2011}, \Reference{Anley2007}, \Reference{Heelan2009}. 
%
Exploits are always aimed at use by a malicious person of some security defect, which attacked software contains. 