\Paragraph{Causes of software security problems}
%
There are three main factors that sufficiently influence to software security. 
%
These include complexity, extensebility, and connectivity \Reference{McGraw2004}. 
%
In the area of software security the factors are known as \Definition{Trinity of Trouble} \Reference{McGraw2006}. 

\Paragraph{Complexity}
%
When developing a software, complexity brings to a big number of problems including security problems. 
%
It is explained by the fact that the complexity has the character of non-randomness and non-linearity, depending on the developing software size \Reference{Brooks1995}. 
%
McConnell thinks that the complexity is the main programming imperative \Reference{McConnell2004}. 
%
With the developing software size increasing, \A chance to make a mistake is increased considerably. 
%
Including an error that influences on security of the software. 
%
This dependence is represented on \OnReferenceToFigure{software_security_error_density}. 

\ReproducedImageFigure{Dependance of errors on software size}{software_security_error_density}{McConnell1997}

\Paragraph{Extensibility}
%
Rise of a project is often is often carried out at the expense of extensibility. 
%
Modern software has a very high degree of extensibility there to be positive effects on the fact that the software is deployed as rapidly as possible in order to gain market share \Reference{McGraw2004}. 
%
But this has in turn leads to a risk of vulnerability occurrence in an software plug-in. 
%
Thereby the software security assurance process becomes more sophisticated. 

\Paragraph{Connectivity}
%
Connectivity also influences on software security considerably. 
%
A vulnerability occurrence in a such software compromises security of all installed and connected to a network copies of the software \Reference{McGraw2004} \Reference{McGraw2006}. 
%
It promotes to a growth of attacks and ease of their carrying out in automatic manner \Reference{ISO17799}. 
%
Thus attacks related to capability of attacked software to connectivity may have a character of massiveness. 

\Paragraph{Secondary software security assurance factors}
%
There are also secondary factors that influences on software security. 
%
These include used during development principles and practices, tools, acquired third-party components, execution environment, and many others \Reference{Winograd2008}. 
%
Ignoring any security factors may cause serious consequences, so software can be exposed. 