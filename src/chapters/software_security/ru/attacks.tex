\Paragraph{Определение}
\Sentence
Атакой на \RussianAbbreviation{ПО} называется \EnglishText{an intentional act by which an entity 
attempts to evade security services and violate the security policy of a system} 
\Reference{RFC4949}.
\Sentence
Лицо, которое проводит такого рода атаки, называется \Emphasis{злоумышленником} 
\Reference{Stallings2008}. 
\Sentence
Целями проведения атаки могут являться сбор, уничтожение или отказ в работе информационных 
системных ресурсов или самой информации \Reference{NISTGlossary2011}. 
\Sentence
То, что является целью проведения атаки для злоумышленника, называется 
\Emphasis{\EnglishText{asset}}. 
\Sentence
В настоящий момент большое количество видов атак, а вместе с тем и их классификаций.

\Paragraph{Классификация атак по их природе}
\Sentence
По своей природе атаки делятся на четыре вида: перехват, модификация, фальсификация и вмешательство 
\Reference{Talukder2007}. 
\Sentence
\Emphasis{Перехватом} является \EnglishText{an unathorized party gaining access to an asset}.
\Sentence
\Emphasis{Модификация} представляет из себя \EnglishText{an unauthorized party gaining control of 
an asset and tampering with it}.
\Sentence
К \Emphasis{фальсификации} относится \EnglishText{an unauthorized party inserts counterfeited 
objects into the system}.
\Sentence
\Emphasis{Вмешательство} происходит, когда \EnglishText{An asset is destroyed or made unusable}.

\Paragraph{Классификация по принципу действия}
\Sentence
По принципу действия атаки классифицируются по следующим категориям \Reference{Goertzel2007}.
\Sentence
\Emphasis{\EnglishText{Reconnaissance}-атаки} помогают злоумышленнику узнать больше важной 
информации об атакумой системе и окружении, в которой она функционирует.
\Sentence
К \Emphasis{\EnglishText{enabling}-атакам} относятся те действия, осуществление которых позволяет 
злоумышленнику провести атаки других видов. 
\Sentence
\Emphasis{\EnglishText{Disclosure}-атаки} нацелены на получение конфиденциальных данных. 
\Sentence
\Emphasis{\EnglishText{Subversion}-атаки} направлены на изменение хода работы атакуемой системы. 
\Sentence
К \Emphasis{\EnglishText{sabotage}-атакам} относятся такие атаки, целью которых отказ в работе или 
доступе системы для обычных пользователей. 

\Paragraph{Классификация по моменту появления уязвимости}
\Sentence
Атаки также классифицируются по тому, на какой стадии жизненного цикла разработки проекта была 
допущена ошибка, приведшая к возможности проведения атаки \Reference{Graff2003}.
\Sentence
Если она была допущена на стадии проектирования, то могут быть проведены следующие атаки: 
человек посередине, состояние гонки, \EnglishText{replay}-атака, \EnglishText{sniffer}-атака и 
другие \Reference{Stallings2008}.
\Sentence
Если ошибка допущена во время написания кода, то могут быть проведены атаки, связанные с 
переполнением буфера или проверкой входных данных, а также с использованием бэкдора 
\Reference{Langweg2004}.
\Sentence
Наконец, атака может быть проведена в связи с небезопасным развёртыванием \RussianAbbreviation{ПО}. 
Наиболее распространённой атакой такого вида является \EnglishText{DoS}-атака 
\Reference{Stallings2008}. 

\Paragraph{Понятие эксплойта}
\Sentence
Атаки на \RussianAbbreviation{ПО} проводятся с помощью \Emphasis{эксплойта}. 
\Sentence
Эксплойтом называется способ, который позволяет злоумышленнику успешно провести атаку 
\Reference{Erickson2003}. 
\Sentence
Это может быть специально сформированный шеллкод, программа или просто набор инструкций 
\Reference{Anley2007}, \Reference{NISTGlossary2011}, \Reference{Heelan2009}. 
\Sentence
Эксплойты всегда направлены на использование злоумышленником какого-либо дефекта безопасности, 
который содержится в атакуемом \RussianAbbreviation{ПО}. 

%\ImageFigure{Типы атак на программное обеспечение}{software_security_attacks}
