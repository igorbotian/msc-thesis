\Paragraph{Определение}
%
\Important{Атакой} на \RussianAbbreviation{ПО} называется \EnglishText{an intentional act by which an entity attempts to evade security services and violate the security policy of a system} \Reference{RFC4949}.
%
Лицо, которое проводит такого рода атаки, называется \Important{злоумышленником} \Reference{Stallings2008}. 
%
Целями проведения атаки могут являться сбор, уничтожение или отказ в работе информационных системных ресурсов или самой информации \Reference{NISTGlossary2011}. 
%
В настоящий момент существует большое количество видов атак, а вместе с тем и их \Important{классификаций}.

\Paragraph{Классификация атак по их природе}
%
По своей природе атаки делятся на четыре вида: перехват, модификация, фальсификация и вмешательство \Reference{Talukder2007}. 
%
\Emphasis{Перехватом} является \EnglishText{an unathorized party gaining access to an asset}.
%
\Emphasis{Модификация} представляет из себя \EnglishText{an unauthorized party gaining control of an asset and tampering with it}.
%
К \Emphasis{фальсификации} относится \EnglishText{an unauthorized party inserts counterfeited objects into the system}.
%
\Emphasis{Вмешательство} происходит, когда \EnglishText{an asset is destroyed or made unusable}.

\Paragraph{Классификация по принципу действия}
%
По принципу действия атаки классифицируются по следующим категориям \Reference{Goertzel2007}.
%
\Emphasis{\EnglishText{Reconnaissance}-атаки} помогают злоумышленнику узнать больше важной информации об атакумой системе и окружении, в которой она функционирует.
%
К \Emphasis{\EnglishText{enabling}-атакам} относятся те действия, осуществление которых позволяет злоумышленнику провести атаки других видов. 
%
\Emphasis{\EnglishText{Disclosure}-атаки} нацелены на получение конфиденциальных данных. 
%
\Emphasis{\EnglishText{Subversion}-атаки} направлены на изменение хода работы атакуемой системы. 
%
К \Emphasis{\EnglishText{sabotage}-атакам} относятся такие атаки, целью которых отказ в работе или доступе системы для обычных пользователей. 

\Paragraph{Классификация по моменту появления уязвимости}
%
Атаки также классифицируются по тому, на какой стадии жизненного цикла разработки проекта была допущена ошибка, приведшая к возможности проведения атаки \Reference{Graff2003}.
%
Если она была допущена на стадии проектирования, то могут быть проведены следующие атаки: человек посередине, состояние гонки, \EnglishText{replay}-атака, \EnglishText{sniffer}-атака и другие \Reference{Stallings2008}.
%
Если ошибка допущена во время написания кода, то могут быть проведены атаки, связанные с переполнением буфера или проверкой входных данных, а также с использованием бэкдора \Reference{Langweg2004}.
%
Наконец, атака может быть проведена в связи с небезопасным развёртыванием \RussianAbbreviation{ПО}. 
%
Наиболее распространённой атакой такого вида является \EnglishText{DoS}-атака 
\Reference{Stallings2008}. 

\Paragraph{Понятие эксплойта}
%
Атаки на \RussianAbbreviation{ПО} проводятся с помощью \Important{эксплойта}. 
%
Эксплойтом называется способ, который позволяет злоумышленнику успешно провести атаку 
\Reference{Erickson2003}. 
%
Это может быть специально сформированный шеллкод, программа или просто набор инструкций \Reference{NISTGlossary2011}, \Reference{Anley2007}, \Reference{Heelan2009}. 
%
Эксплойты всегда направлены на использование злоумышленником какого-либо \Emphasis{дефекта} безопасности, который содержится в атакуемом \RussianAbbreviation{ПО}. 
