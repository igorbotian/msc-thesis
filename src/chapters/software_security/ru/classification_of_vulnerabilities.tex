\Paragraph{Рост количества уязвимостей}
%
По данным \EnglishText{CERT} за период с 1995 по 2008 (3 квартала) годы, количество обнаруженных уязвимостей \RussianAbbreviation{ПО} \Emphasis{непрерывано растёт} (см. \ReferenceToFigure{software_security_vulnerabilities_reported_to_cert}) \Reference{CERTStatistics2008}. 
%
Данный факт говорит о всё возрастающем риске возможных атак как в настоящем, так и в будущем. 
%
А это приводит к росту степени критичности обеспечения безопасности \RussianAbbreviation{ПО} в целом. 

\ReproducedImageFigure{Уязвимости, данные о которых получены CERT}
	{software_security_vulnerabilities_reported_to_cert}{CERTStatistics2008}

\Paragraph{Классификация уязвимостей ПО}
%
\Emphasis{Следствием непрерывного роста} новых уязвимостей за последние 20 лет является появление большого количества их классификаций. 
%
Цель каждой такой классификации заключается в помощи разработчикам \RussianAbbreviation{ПО} и специалистам по безопасности \RussianAbbreviation{ПО} понять общие ошибки кодирования, которые затрагивают безопасность \Reference{McGraw2006}. 
%
Первые классификации стали появляться ещё в 70-х годах, когда только начал изучаться вопрос безопасности \RussianAbbreviation{ПО} \Reference{Goertzel2007}.  
%
В настоящий момент наиболее широко используемыми классификациями являются \Term{OWASP} \Reference{OWASPWebTop10}, \Term{Seven Pernicious Kingdoms} \Reference{Tsipenyuk2005}, \Term{Fortify Taxonomy of Software Security Errors} \Reference{FortifyTaxonomy2009} и ряд других.

\SubSubSectionTitle{Общая классификация}
	{software_security_vulnerabilities_general_classification}

\Paragraph{Как проводится классификация}
%
При классификации уязвимостей \RussianAbbreviation{ПО} специалист исследует, каким образом, в какой момент и в каких частях системы появляется каждая уязвимость \Reference{Landwehr1993} \Reference{Landwehr1994} \Reference{Dowd2006}. 
%
То есть в общем случае уязвимости могут быть классифицированы по происхождению, по времени появления и по месторасположению. 
%
Причём одна и та же уязвимость может быть классифицирована сразу по нескольким признакам. 

\Paragraph{Классификация по происхождению}
%
\Important{По происхождению} уязвимости классифицируются на основе того, каким образом они появляются в системе: намеренно или ненамеренно со стороны разработчика \Reference{Landwehr1993}. 
%
К намеренно оставленным в системе уязвимостям можно отнести бэкдор, необходимый разработчику для проведения удалённой отладки. 
%
Уязвимости, появившиеся в системе ненамеренно, по своей сути представляют из себя обычные ошибки, которые могут быть допущены на стадии составления требований, разработки спецификации, написания кода или внедрения. 

\Paragraph{Классификация по времени появления}
%
Классификация уязвимостей на основе того, в какой \Important{момент} уязвимость появилась в системе, тесно связана с этапами жизненного цикла разработки \RussianAbbreviation{ПО}. 
%
По этому критерию уязвимости могут появиться в системе во время составления требований, разработки спецификации, написания кода или внедрения \Reference{Landwehr1994} \Reference{Dowd2006}. 

\Paragraph{Классификация по месторасположению}
%
Уязвимости также могут классифицироваться по их \Important{месторасположению} \Reference{Landwehr1993}. 
%
Они могут появиться в \RussianAbbreviation{ПО} операционной системы, в прикладных приложениях, сторонних модулях или приложениях. 

\SubSubSectionTitle{Seven Pernicious Kingdoms}
	{software_security_vulnerabilities_seven_pernicious_kingdoms}

\Paragraph{Seven Pernicious Kingdoms}
%
Одна из наиболее широко известных и популярных классификаций уязвимостей, называемая \Term{Seven Pernicious Kingdoms}, была разработана Гэри МакГроу \Reference{McGraw2006} \Reference{Goertzel2007} \Reference{Tsipenyuk2005}. 
%
Согласно данной классификации все уязвимости рассматриваются как общие ошибки кодирования с позиции злоумышленника, а не разработчика. 
%
Её основным \Important{преимуществом} является то, что уязвимости каждой категории могут быть легко найдены с помощью автоматизированных средств. 
%
Основным \Important{недостатком} является её неполнота, так она содержит только уже известные уязвимости. 
%
В данной классификации все уязвимости делятся на семь \Important{категорий}, которые автором называются 
\Emphasis{королевствами}:  
\begin{description}
	\leftskip2em%
	\setlength{\itemsep}{0pt}%
	\setlength{\parsep}{0pt}%

	\item[Проверка входных данных и их представление.] Переполнение буфера, внедрение шеллкода, \Abbreviation{XSS}, уязвимости форматных строк, переполнение целочисленных данных, неверное значение указателя, внедрение SQL-кода, использование относительных путей к файлам и др..
	\item[Неправильное использование API.] Нарушение контракта элементов \Abbreviation{API}, неверная стратегия обработки ошибок, отсутствие проверки возвращаемого функцией значения и т.п..

	\item[Средства безопасности.] Полное или частичное отсутствие механизма безопасности: контроля доступа, авторизации и аутентификации и пр..

	\item[Время и состояние.] Отсутствие или неверное использование примитивов синхронизации, состояния гонки, дедлоки и т.п.
	
	\item[Ошибки.] Отсутствие стратегии обработки ошибок или её некорректность, двойное освобождение памяти, утечки памяти, разыменование \SourceCode{NULL}-указателя и др..
	
	\item[Инкапсуляция.] Раскрытие лишней информации, недостаточное знание основ \RussianAbbreviation{ООП}.
	
	\item[Программное окружение.] Неправильное установление прав доступа к файлам, хранение паролей и другой конфиденциальной информации в незашифрованном виде.
\end{description}

\SubSubSectionTitle{Базы уязвимостей}
	{software_security_vulnerabilities_databases}

\Paragraph{Базы данных уязвимостей}
%
Все существующие на данные момент классификации уязвимостей используются в структере \Emphasis{баз данных уязвимостей}.
%
На настоящий момент существует несколько таких баз. 
%
Наиболее известными являются \Term{CWE} \Reference{CWE}, \Term{BugTraq} \Reference{BugTraq}, \Term{ICAT} \Reference{ICAT} и другие. 
%
Такое разнообразение приводит к \Emphasis{избыточности}.
%
С одной стороны, информация об уязвимости может дублироваться в каждой базе.
%
С другой стороны, чтобы получить исчерпывающую информацию, необходимо воспользоваться как можно большим количеством баз.
%
Более того, каждая из перечисленных выше баз имеет \Emphasis{собственную классификацию} уязвимостей.

\Paragraph{Наиболее опасные уязвимости}
%
Наиболее важным \Important{преимуществом} существования и использования баз данных уязвимостей можно назвать возможность проведения статистических расчётов: какие уязвимости встречаются чаще всего, какие из них являются наиболее актуальными и т.п.
%
Благодаря этому можно получить данные о том, какие уязвимости являются наиболее опасными. 

\Paragraph{Наиболее опасные уязвимости согласно CWE}
%
Например, согласно \Term{CWE}, они классифицируются тремя \Emphasis{категориями}: небезопасное взаимодействие между компонентами системы, рискованное управление ресурсами и слабый механизм защиты \Reference{CWETop25DangerousErrors2011}.
%
Уязвимости из первой категории связаны с небезопасностью используемых методов передачи данных между компонентами как приложения, так и между самим приложением и его программным окружением. 
%
Уязвимости из второй категории связаны с небезопасным созданием, использованием и освобождением важных системных ресурсов. 
%
Уязвимости третьей категории связаны с некорректным выбором защитных методик или отсутствием их использования.