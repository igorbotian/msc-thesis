\Paragraph{Понятие ошибки в ПО}
%
Разработчики часто допускают ошибки во время разработки \RussianAbbreviation{ПО}. 
%
Они могут быть допущены во время любой стадии жизненного цикла разработки \RussianAbbreviation{ПО}: 
в требованиях, на стадии проектирования, написания кода, тестирования или внедрения. 
%
В общем смысле ошибкой называется действие человека, которое приводит к некорректному результату 
(например, \RussianAbbreviation{ПО}, содержащее дефект) \Reference{Landwehr1994}. 
%
Последствием наличия ошибки в \RussianAbbreviation{ПО} является появление в нём дефекта. 
%
К дефектам можно отнести отности некорректныю инструкцию, процесс, или определение данных 
в компьютерной программе \Reference{Landwehr1994}. 
%
Дефект, в свою очередь, может привести к отказу \RussianAbbreviation{ПО} от работы. 
%
Отказом называется невозможность \RussianAbbreviation{ПО} или его отдельного компонента 
в выполнении требуемых операций \Reference{Landwehr1994}.

\Paragraph{Причины появления ошибок в ПО}
%
Появление в \RussianAbbreviation{ПО} ошибки, связанной с безопасностью, может произойти по ряду 
причин. 
%
Это может быть как недостаток мотивации или знаний со стороны разработчика, так и 
недостаток обеспечения безопасности со стороны применяемой им технологии \Reference{Goertzel2007}. 
%
Граф выделяет три вида факторов, влияющих на недостаток обеспечения безопасности со стороны 
разработчика \Reference{Graff2003}: технические, психологические и обыденные. 
%
К технологическому фактору относится в первую очередь сложность современного 
\RussianAbbreviation{ПО}. 
%
К психологическим факторам относятся неправильная оценка рисков, связанных с безопасностью, а также 
различие в мышлении и отношении к \RussianAbbreviation{ПО} со стороны разработчика и атакующего. 
%
Наконец, к обыденным факторам можно отнести источник используемого исходного кода, наличие строгих 
сроков разработки проекта или отсутствие требований к его безопасности.

\Paragraph{Понятие уязвимости ПО}
%
Но не каждая ошибка, допущенная разработчиков, влияет на безопасность разрабатываемого им в 
\RussianAbbreviation{ПО}. 
%
Уязвимостью называется такая ошибка спецификации, разработки или конфигурации ПО, которая может 
привести к нарушению политики безопасности и последующему эксплойтированию \RussianAbbreviation{ПО} s
\Reference{Alhazmi2006} \Reference{RFC4949} \Reference{Heelan2009} \Reference{McGraw2004}.
%
Согласно результатам исследования, проведённого Доудом, большинство уязвимостей связано с 
некорретными обработкой входных данных, уровнем доверия между компонентами системы, 
\Abbreviation{API}, взаимодействием с программным окружением и другими приложениями 
\Reference{Dowd2006}. 
