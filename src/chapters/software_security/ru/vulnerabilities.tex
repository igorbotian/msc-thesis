\Paragraph{Понятие ошибки в ПО}
%
Разработчики часто допускают ошибки во время разработки \RussianAbbreviation{ПО}. 
%
Они могут быть допущены во время любой стадии жизненного цикла разработки \RussianAbbreviation{ПО}: в требованиях, на стадии проектирования, написания кода, тестирования или внедрения. 
%
В общем смысле ошибкой называется действие человека, которое приводит к некорректному результату (например, \RussianAbbreviation{ПО}, содержащее дефект) \Reference{Landwehr1994}. 
%
Последствием наличия ошибки в \RussianAbbreviation{ПО} является появление в нём дефекта. 
%
К дефектам можно отнести отности некорректныю инструкцию, процесс, или определение данных в компьютерной программе \Reference{Landwehr1994}. 
%
Дефект, в свою очередь, может привести к отказу \RussianAbbreviation{ПО} от работы. 
%
Отказом называется невозможность \RussianAbbreviation{ПО} или его отдельного компонента в выполнении требуемых операций \Reference{Landwehr1994}.

\Paragraph{Причины появления ошибок в ПО}
%
Появление в \RussianAbbreviation{ПО} ошибки, связанной с безопасностью, может произойти по ряду причин. 
%
Это может быть как недостаток мотивации или знаний со стороны разработчика, так и недостаток обеспечения безопасности со стороны применяемой им технологии \Reference{Goertzel2007}. 
%
Граф выделяет три вида факторов, влияющих на недостаток обеспечения безопасности со стороны разработчика \Reference{Graff2003}: технические, психологические и обыденные. 
%
К технологическому фактору относится в первую очередь сложность современного \RussianAbbreviation{ПО}. 
%
К психологическим факторам относятся неправильная оценка рисков, связанных с безопасностью, а также различие в мышлении и отношении к \RussianAbbreviation{ПО} со стороны разработчика и атакующего. 
%
Наконец, к обыденным факторам можно отнести источник используемого исходного кода, наличие строгих сроков разработки проекта или отсутствие требований к его безопасности.

\Paragraph{Понятие уязвимости ПО}
%
Но не каждая ошибка, допущенная разработчиков, влияет на безопасность разрабатываемого им в \RussianAbbreviation{ПО}. 
%
Уязвимостью называется такая ошибка спецификации, разработки или конфигурации \RussianAbbreviation{ПО}, которая может привести к нарушению политики безопасности и последующему эксплойтированию \RussianAbbreviation{ПО} \Reference{Alhazmi2006} \Reference{RFC4949} \Reference{Heelan2009} \Reference{McGraw2004}.
%
Согласно результатам исследования, проведённого Доудом, большинство уязвимостей связано с некорретными обработкой входных данных, уровнем доверия между компонентами системы, \Abbreviation{API}, взаимодействием с программным окружением и другими приложениями \Reference{Dowd2006}. 

\Paragraph{Уязвимости языков С и С++}
%
В настоящий момент подавляющая часть известных на данный момент уязвимостей обнаружена в \RussianAbbreviation{ПО}, написанном с использованием языков среднего уровня \WebSite{ICAT} \WebSite{CWE} \WebSite{BugTraq}. 
%
С одной стороны, это можно объяснить тем, на уровне дизайна данных языков предусмотрено меньшее количество ограничений по сравнению с языками более высокого уровня \Reference{Seacord2005}. 
%
К примеру, наиболее характерной чертой языков среднего уровня является механизм ручного управления памятью. 
%
С другой стороны, языки среднего уровня С \Reference{CStandard1999} и С++ \Reference{CppStandard2003} являются одними из наиболее популярных языков программирования на настоящий момент \Reference{TIOBE}.

\Paragraph{Риск, связанный с языками более высокого уровня}
%
Но даже используя языки более высокого уровня, нельзя быть уверенным в безопасности разрабатываемого \RussianAbbreviation{ПО}, так как среды выполнения языков высокого уровня сами написаны с использованием языков среднего уровня (\EnglishText{PHP} \WebSite{PHP}, \EnglishText{Python} \WebSite{Python}, \EnglishText{Java} \WebSite{JVM} и др.). 
%
Тогда при появлении уязвимости в среде выполнения существует высокий риск безопасности \RussianAbbreviation{ПО}, выполняемого в ней.

\Paragraph{Наиболее опасные уязвимости}
%
Чаще всего уязвимости, специфичные для языков С/С++, являются следствием допущения ошибок, связанных с недостаточным пониманием со стороны программиста того, как данные располагаются в памяти.
%
Наиболее опасные и часто встречающиеся \Reference{CERTStatistics2008} уязвимости описываются в следующих подразделах. 

% ------------------------------------------------------------------------------------------------

\SubSubSectionTitle{Ошибки целочисленных данных}{software_security_vulnerabilities_integer_errors}

\Paragraph{Описание}
%
Ошибкам, связанным с работой с целочисленными данным, подвержены практически все известные языки программирования. 
%
Проблема состоит в том, каждый целочисленный тип данных имеет ограниченное множество его возможных значений, поэтому имеет наибольшее и наименьшее значение. 
%
Причина такого ограничения является архитектурной, так как каждое число представляется в памяти в виде фиксированного количества бит \Reference{Chess2007}.

\Paragraph{Атаки}
%
Существует всего несколько видов ошибок, связанных с целочисленными типами данных: переполнение целочисленных данных, ошибки при работе со знаковыми и беззнаковыми числами и ошибки при приведении одного целочисленного типа к другому \Reference{Younan2004} \Reference{Chase2005} 
\Reference{Seacord2005} \Reference{Chess2007}. 
%
В общем случае они могут привести лишь к отказу в работе или логическим ошибкам. 
%
То есть они сами по себе не могут быть эксплойтированы, но они могут привести к появлению эксплойтированных уязвимостей другого вида, например, к переполнению буфера \Reference{Howard2005}. 

\Paragraph{Пример}
%
В листинге ниже представлен пример программы, которая проверяет, является ли число, введённое пользователем, простым или нет. 
%
Данная программа содержит ряд уязвимостей, которыми потенциальный злоумышленник может воспользоваться в своих целях. 
%
Одна из них связана с ошибкой работы с целочисленными данными.
%
Число, введённое пользователем, хранится в переменной \SourceCode{number} типа \SourceCode{unsigned char} и имеет наибольшее допустимое значение \SourceCode{UCHAR\_MAX}, равное 255.
%
Если пользователь введёт число, величина которого больше наибольшего допустимого, то произойдёт переполнение переменной \SourceCode{number}. 
%
В данной программе это повлияет лишь на корректность работы программы. 
%
Но в некоторых случаях ошибки такого рода могут привести к изменению хода работы \RussianAbbreviation{ПО}, отказу в работе или утечке конфиденциальной информации.

\Listing{Пример программы, содержащей уязвимости}{C}
	{listings/vulnerabilities/primality_test.c}

\Paragraph{Меры по предотвращению}
%
Для нахождения ошибок целочисленных данных можно воспользоваться несколькими методами: проведение проверки кода и использование инструментов статического анализа, а также средств, предоставляемых компилятором \Reference{Howard2005} \Reference{Younan2004}.

% ------------------------------------------------------------------------------------------------

\SubSubSectionTitle{Переполнение буфера}{software_security_vulnerabilities_buffer_overflow}

\Paragraph{Описание}
%
Переполнение буфера считается наиболее опасной и часто встречающейся уязвимостью, связанной с языками программирования среднего уровня \Reference{Howard2005}. 
%
В таких языках не предусмотрены встроенные средства для проверки целостности границ массивов, поэтому операции для работы с ними являются потенциально небезопасными. 

\Paragraph{Атака}
%
Чаще всего целью использования переполнения буфера для злоумышленника является выполнение специального кода \Reference{McGraw2004}. 
%
Для этого сначала злоумышленник сначала формирует входные данные, которые ведут к переполнению буфера, а затем получает контроль над ходом работы программы и выполнет необходимые ему операции. 

\Paragraph{Пример}
%
В коде программы, представленной в  \InReferenceToSection{software_security_vulnerabilities_integer_errors}, содержится ошибка, которая может привести к переполнению буфера.
%
Если пользователь введёт число, большее чем 9999, то есть которое имеет 5 или более цифр, то это приведёт к переполнению буфера \SourceCode{number\_buf\_size}.
%
Присутствие такой ошибки даёт злоумышленнику возможность получить контроль хода работы программы и выполнить произвольный набор операций.
%
Особенно это опасно, если программа-жертва запущена с привилегированным уровнем доступа.
%
Подробно техники получения контроля хода работы программы с использованием уязвимости переполнения буфера рассматриваются Эриксоном и другими \Reference{Erickson2003} {Lhee2003}.

\Paragraph{Меры по предотвращению}
%
Для обнаружения данной уязвимости могут использоваться следующие методы: проведение инспекция кода, использование средства компилятора, проведение нагрузочного тестирования, использование инструментов автоматического анализа кода \Reference{Howard2005} \Reference{Chase2005} \Reference{Younan2004}.

% ------------------------------------------------------------------------------------------------

\SubSubSectionTitle{Уязвимости форматных строк}{software_security_vulnerabilities_format_strings}

\Paragraph{Описание}
%
Причиной появления уязвимостей форматных строк является использование входных данных без проверки. 
%
При использовании функций форматированного вывода, которые записывают данные в массив символов, например, \SourceCode{sprintf()} языка С, предполагается, что буфер имеет произвольно большой размер. 
%
Это может привести к его переполнению, что показано на рис. из предыдушего раздела.

\Paragraph{Атака}
%
В языках С/С++ данные уязвимости могут быть использованы злоумышленником для записи в произвольные участки памяти или для утечки информации \Reference{Howard2005}. 
%
Приложение может быть эксплойтировано для аварийного завершения, просмотра содержимого стека, просмотра содержимого памяти или для перезаписи участка памяти \Reference{Seacord2005} \Reference{Gallagher2006} \Reference{Lhee2003}. 

\Paragraph{Пример}
%
В коде программы, представленной в \InReferenceToSection{software_security_vulnerabilities_integer_errors}, содержится уязвимость, связанная с форматной строкой.
%
При выводе результата используется функция форматирования строк \SourceCode{printf()}. 
%
Но при её вызове опущен первый аргумент, который задаёт формат выводимой строки. 
%
Это позволяет злоумышленнику задать свой собственный формат, причём подобрав такое значение входных данных, которое может привести к изменению хода работы программы и выполнению произвольных операций 
\Reference{Erickson2003}. 

\Paragraph{Меры по предотвращению}
%
Для поиска уязвимостей данного вида применяются следующие методы \Reference{Howard2005} 
\Reference{Lhee2003}: 
\begin{itemize}

	\item проведение инспекции кода, использующего функции форматированного вывода 
	
	\item использование средств автоматического статического анализа (\Term{RATS} \WebSite{RATS}, \Term{Flawfinder} \WebSite{Flawfinder}, \Term{PScan} \WebSite{PScan} и др.) 
	
	\item использование дополнительных библиотек (\Term{FormatGuard} \WebSite{FormatGuard}, \Term{libformat} \WebSite{Libformat}, \Term{libsafe} \WebSite{Libsafe} и др.) 
\end{itemize}

% ------------------------------------------------------------------------------------------------

\SubSubSectionTitle{Уязвимости указателей}{software_security_vulnerabilities_pointers}

\Paragraph{Описание}
%
Указатель -- это переменная, которая хранит адрес функции, элемента массива или другой структуры данных \Reference{Seacord2005}. 
%
Работа с указателями является достаточно сложной и зачастую подвержена появлению ошибок. 
%
Наиболее распространённой ошибкой является двойное освобождение участка памяти, на которую указывает указатель. 
%
Также существуют уязвимости указателей на функции или данные.

\Paragraph{Атака}
%
\EnglishText{\Emphasis{Pointer subterfuge}} является общим термином для экскплойтов, которые изменяют значение указателя \Reference{Seacord2005}. 
%
Указатели на данные также могут быть изменены для запуска произвольного кода. 
%
Если указатель на данные используется как цель для последующего присвоения, то злоумышленник может контролировать его адрес, чтобы изменить другие участки памяти \Reference{Seacord2005}. 

\Paragraph{Пример}
%
В коде программы, представленной в \InReferenceToSection{software_security_vulnerabilities_integer_errors}, введённые пользователем данные хранятся в массиве.
%
Доступ к ним осуществляется посредством указателя \SourceCode{number\_buf\_size}. 
%
При определённом значении вводимых данных, злоумышленник может выполнить произвольные операции. 
%
Эксплойтирование непосредственно осуществляется через этот указатель на данные.

\Paragraph{Меры по предотвращению}
%
Основной мерой по предотвращению появления уязвимостей такого рода являются тщательная ревизия кода, в котором производится работа с указателями. 
%
Дополнительным средством является использование средств компилятора или его расширений, которые направлены на поиск ошибок данного вида. 
%
\Term{Mudflap} является примером такого технического средства \Reference{Eigler2003}. 

% ------------------------------------------------------------------------------------------------

\SubSubSectionTitle{Неправильная обработка ошибок и исключительных ситуаций}
	{software_security_vulnerabilities_errors_and_exceptions}

\Paragraph{Описание}
%
Данная ошибка может привести к аварийному завершению или отказу \RussianAbbreviation{ПО} в работе, а также переходу в незащищённое состояние и к появлению других уязвимостей \Reference{Howard2005}. 
%
В общем случае она может появиться в следующих случаях \Reference{McConnell2004}:
\begin{itemize}
	\item приложение выдаёт слишком много информации в сообщениях об ошибке 
	\item игнорирование ошибок 
	\item неправильная обработка ошибки 
	\item обработка не всех возможных исключений 
	\item обработка исключения на некорректном уровне абстракции 
\end{itemize}

\Paragraph{Пример}
%
В листинге ниже приведён фрагмент кода программы на языке Java, который содержит \Abbreviation{API} для получения данных об ассортименте интернет-магазина. 
%
Программа содержит ошибку, связанную с неверным выбором уровня абстракции исключения, которое может сгенерировать метод. 
%
В сигнатуре метода \SourceCode{getTShirts()} содержится информация о том, что метод может сгенерировать исключение типа \SourceCode{SQLException}. 
%
Это стандартное исключение, входящее в базовый пакет Java и сигнализируюшее об ошибках, связанных 
с \RussianAbbreviation{БД}. 
%
Такая сигнатура с точки зрения безопасности некорректна. 
%
Во-первых, злоумышленник узнает детали внутреннего устройства программы, что недопустимо. 
%
Во-вторых, по тексту ошибки, которое содержит это исключение, он может получить важную информацию, что облегчит возможность его последующего проникновения в систему. 

\Listing{Пример элементов API, декларирующих исключения с некорректным уровнем абстракции}
	{Java}
	{listings/vulnerabilities/exceptions.java}

\Paragraph{Меры по предотвращению}
%
Уязвимости подобного рода очень сильно зависят от контекста, поэтому их найти и устранить можно лишь путём тщательной проверки кода. 
%
Для их предотвращения используются следующие практики \Reference{Howard2005}: 
\begin{itemize}
	
	\item корректная обработка всех возможных исключений с учётом уровня абстракции 
	
	\item проверка возвращаемого функцией значения, если оно может сигнализировать об ошибке 
	
	\item использование средств протоколирования вмест практики включения подробной информации об исключительной ситуации в сообщение об ошибке 
\end{itemize}

% ------------------------------------------------------------------------------------------------

\SubSubSectionTitle{Утечка информации}{software_security_vulnerabilities_information_leakage}

\Paragraph{Описание}
%
Утечка информации может быть использовано злоумышленников для дальнейшего проведения атаки на систему или данные \Reference{Howard2005}. 
%
Конфигурация системы и другая важная информация  может содержаться в сообщениях об ошибке, которое выдаётся пользоавтелю \Reference{Chase2005}. 
%
Конфиденцальные данные могут храниться в незашифрованном виде как в исходном коде программы, так и в ресурсах или других ресурсах, доступных для чтения пользователю \Reference{Gallagher2006}. 
%
Может осуществляться небезопасная передача пользовательской  информации, в том числе конфиденциальной, в процессе сетевого или межпроцессного взаимодействия \Reference{Howard2005}.

\Paragraph{Пример}
%
В листинге ниже представлен фрагмент программы, осуществляющей функционирование интернет-магазина. 
%
Метод \SourceCode{getTShirts()} является небезопасным, так как содержит уязвимость, которая может привести к утечке информации.
%
При неудаче выполнения операции получения данных о товаре из \RussianAbbreviation{БД} методом генерируется исключение типа \SourceCode{FanStoreException}. 
%
Но помимо текста об ошибке исключение содержит подробную информацию о причине неудачи выполнения операции. 
%
Это небезопасно, так как злоумышленнику предоставляется информация о конфигурации системы и деталях её внутреннем устройстве. 

\Listing{Пример кода, содержащего утечку важной информации}
	{Java}
	{listings/vulnerabilities/information_leakage.java}

\Paragraph{Меры по предотвращению}
%
В качестве основным мер по предотвращению могут выступать следующие: использование средств логирования для хранения детальной информации об ошибке, сообщения об ошибках её не содержат:
\begin{itemize}
	
	\item хранение конфиденциальной информации в зашифрованном виде
	
	\item осуществление межпроцессного или сетевого взаимодействия с использованием защищённых каналов
	
	\item передача пользовательских данных в зашифрованном виде при их отсутствии;
\end{itemize}

% ------------------------------------------------------------------------------------------------

\SubSubSectionTitle{Файловая система}{software_security_vulnerabilities_file_system}

\Paragraph{Описание}
%
Проблемы безопасности данного вида тесно связаны с взаимодействием приложения с файловой системой. 
%
С точки зрения пользователя, каждый файл имеет, собственно, содержимое, а также набор атрибутов (имя, информация о владельце, атрибуты доступа и др.). 
%
Поэтому риски безопасности файлов связаны не только с возможностью чтения или изменения его содержимого, но также и с корректностью заданных ему атрибутов \Reference{Drepper2009}.

\Paragraph{Уязвимости}
%
С некорректной работой с файловой системой связан ряд уязвимостей. 
%
Некорректно выставленные атрибуты могут способствовать к несанкционированным чтению или изменению содержимого файла \Reference{Seacord2008}. 
%
При одновременном использовании файла как разделяемого ресурса между несколькими процессами или потоками возможно возникнование состояния гонки (см. подраздел ниже) \Reference{Howard2005}. 
%
Наконец, в содержимом файла может храниться конфиденциальная информация \Reference{Seacord2008}. 
%
Таким образом можно сделать вывод, что приложение является ответственно за содержимое файлов и за доступ к ним.

\Paragraph{Меры по предотвращению}
%
Для предотвращения появления уязвимостей такого рода необходимо можно выполнять следующие действия: хранить конфиденциальную информацию в зашифрованном виде, корректно выставлять атрибуты доступа ко всем используемым файлам, рассматривать файлы как потенциально разделяемые ресурсы в условии многозадачности программной среды.

% ------------------------------------------------------------------------------------------------

\SubSubSectionTitle{Проверка входных данных}{software_security_vulnerabilities_input_data}

\Paragraph{Описание}
%
С позиции атакуемого \RussianAbbreviation{ПО} взаимодействие с ним злоумышленника происходит посредством обработки входных данных из источников, используемых приложением. 
%
Видами таких источников могут являться стандартный поток ввода, файловая система, программное окружение, файловая система, пользовательский интерфейс и др. \Reference{Wheeler2003}. 

\Paragraph{Атака}
%
Большинство атак со стороны злоумышленника происходит путём составления специально составленных входных данных для атакуемого приложения \Reference{Seacord2005}. 
%
Вследствие ошибки при обработке этих данных приложение может придти в некорректное состояние и выполнить действия, необходимые злоумышленнику. 

\Paragraph{Пример}
%
В коде программы, представленной в \InReferenceToSection{software_security_vulnerabilities_integer_errors}, содержатся уязвимости переполнения буфера, форматных строк и другие. 
%
С помощью проверки данных, вводимых пользователем, можно снизить как вероятность, так и последствие эксплойтирования данных уязвимостей. 

\Paragraph{Методы по предотвращению}
%
Основной мерой по предотвращению данной уязвимости является осуществление проверки входных данных на всех уровнях приложениях, реализуя принцип “безопасность в глубине” \Reference{Viega2003}. 
%
Принцип проверки входных данных на любом уровне приложения называется техникой защитного програмирования \Reference{McConnell2004}. 
%
Обеспечение защиты от некорректных входных данных заключается в выполнении следующих действий 
\Reference{Chess2007}: 
\begin{enumerate}

	\item идентификация всех источников входных данных на всех уровнях 

	\item выбор стратегии отличия корректных данных от некорректных на каждом уровне 

	\item попытка уменьшения возможности ввода некорректных данных посредством \Abbreviation{API} или других источников
\end{enumerate}

\Paragraph{Достоинства и недостатки техники защитного программирования}
%
Однако техника защитного программирования имеет свои достоинства и недостатки 
\Reference{McConnell2004}. 
%
Основными недостатками являются увеличение сложноности приложения и снижение скорости его работы после добавления кода проверки входных данных. 
%
Причём этот код сам может содержать ошибки. 
%
Очевидным преимуществом является быстрое обнаружение ошибок и их дальнейшее устранение.

% ------------------------------------------------------------------------------------------------

\SubSubSectionTitle{Cостояние гонки}{software_security_vulnerabilities_race_conditions}

\Paragraph{Понятие}
%
Некоторые операции требуют атомарности, то есть непрерывности своего выполнения \Reference{Eckel2006}. 
%
Если они начинают выполняться, то не могут быть приостановлены для работы другого потока до своего завершения. 
%
Ошибки случаются, когда в приложении условие атомарности не выполняется при параллельном выполнении таких операций \Reference{Dowd2006}. 
%
Состоянием гонки является ошибка проектирования многозадачной программы, при которой работа операции зависит от того, в каком порядке выполняются части её кода \Reference{Netzer1992}. 
%
То есть не выполняется необходимое условие атомарности.

\Paragraph{Более подробное описание}
%
Состояние гонки независимо от языка программирования. 
%
Чаще всего оно происходит тогда, когда два различных выполняющихся потока или процесса имеют возможность изменять какой- либо обший ресурс и из-за этого вмешиваются в работу друг друга \Reference{Howard2005}. 
%
Такими разделяемых ресурсов, могут  быть разделяемая область памяти, файловая система, база данных или переменная. 

\Paragraph{Атака}
%
Экслойтирование приложений, содержащих уязвимость состояния гонки, происходит посредством изменения разделяемых ресурсов, которые они используют \Reference{Stallings2008}. 
%
Причём сама атака проводится в определённое временное окно, в котором по ошибке программиста данные ресурсы доступны для изменения. 
%
Последствиями проведения атаки являются разрушение разделяемого ресурса, аварийное завершение программы-жертвы, подмена данных с последующей их обработкой программой-жертвой на уровне своих привилегий. 

\Paragraph{Пример}
%
В листинге ниже представлен фрагмент кода программы, в котором выполняется использование ресурса. 
%
Данный фрагмент содержит уязвимость состояния гонки. 
%
По замыслу программиста, после положительной проверки доступности ресурса (метод \SourceCode{isResourceAvailable()}) можно обработать данные (метод \SourceCode{processResourceData()}), которые в нём содержатся. 
% 
С точки зрения безопасности, это некорректно, так как между выполнением тел этих методов существует зазор, во время которого менеджер потоков может приостановить выполнение кода программы. 
%
А при его возобновлении может возникнуть ситуация, когда данные ресурса ещё не готовы к обработке, но она будет проведена. 

\Listing{Пример кода, содержащего уязвимость состояния гонки}
	{Java}{listings/vulnerabilities/race_condition.c}

\Paragraph{Способы предотвращения}
%
Для правильной работы приложения в многопоточной среде, необходимо выполнение одного из двух условий в зависимости от контекста \Reference{Netzer1992} \Reference{Dowd2006}: 
\begin{itemize}
	
	\item состояние процесса не должно зависеть от состояния разделяемого ресурса
	
	\item разделяемый ресурс должен быть неизменяемым, или доступ к нему должен быть 
		синхронизованным 
\end{itemize}

\Paragraph{Долнительный способ поиска}
%
Дополнительным способом поиска и устранения состояния гонки в рамках файловой системы, является использование средств статического и динамического анализа \Reference{Seacord2005}. 
%
Наиболее известными такими средствами статического анализа являются \Term{ITS4} \WebSite{ITS4}, \Term{PVS Studio} \WebSite{PVSStudio} и \Term{RATS} \WebSite{RATS}, а динамического – \Term{Helgrind} \WebSite{Helgrind} и \Term{Inspector XE} \WebSite{InspectorXE}. 