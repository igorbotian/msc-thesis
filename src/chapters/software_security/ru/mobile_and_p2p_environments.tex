\Paragraph{Введение}
\Sentence
Исследование, проводимое в данной работе, связано с поиском уязвимостей в мобильном пиринговом 
окружении. 
\Sentence
Поэтому необходимо рассмотрение не только общих уязвимостей, но и тех, которые характерны мобильным 
и пиринговым окружениям. 

\SubSubSectionTitle{Безопасность програмнного обеспечения в мобильном окружении}
	{software_security_mobile_environment}

\Paragraph{Понятие мобильного окружения}
\Sentence
Мобильная среда отличается mobile computing и тем, что доступ к данным и информационным сервисам 
в ней осуществляется при помощи портативных беспроводных устройств \Reference{Karki2008}. 
\Sentence
В качестве таких мобильных устройств могут выступать ноутбуки, \RussianAbbreviation{КПК}, пейджеры, 
смартфоны и обычные мобильные телефоны и многие другие.
\Sentence
Мобильная среда характеризуется гетерогенностью вычислительного окружения, а также неустойчивостью 
качества сервиса, \Abbreviation{QoS}, осуществляемого нижележащей коммуникационной инфраструктурой 
\Reference{Davies1994}. 
\Sentence
Также отмечаются такие неотъемлемые черты данной среды, как мобильность пользователей и сетевых 
элементов, беспроводной характер коммуникационных устройств \Reference{Asokan1995}. 

\Paragraph{Общие проблемы безопасности в мобильном окружении}
\Sentence
С точки зрения безопасности мобильное окружение по своей природе гораздо более уязвимо, чем обычное 
сетевое окружение. 
\Sentence
Следствием это является увеличение общей сложности обеспечения в неё безопасности. 
\Sentence
Ли приводит следующие проблемы безопасности \Reference{Li2004}: 
\begin{description}
	\leftskip2em%
	\setlength{\itemsep}{0pt}%
	\setlength{\parsep}{0pt}%

	\item[Отсутствие границ безопасности.] Мобильное окружение становится более подверженным 
		к таким атакам, как пассивный перехват, активное вмешательство в процесс передачи данных, 
		утечка конфиденциальной информации, отказ в работе (\Abbreviation{DoS}).
	\item[Угроза со стороны скомпрометированных узлов.] Получение контроля над одним или 
		несколькими узлами в сети позволит провести более широкомасштабные атаки на другие узлы. 
	\item[Отсутствие централизованного средства управления.] Приводит к усложнению обнаружения и 
		предотвращения проведения атак по причине автономности узлов.
	\item[Ограниченное электропитание.] Способствует проведению \DoS-атак и подрыву операций, 
		выполняющихся в кооперации несколькими узлами. 
\end{description}

\Paragraph{Проблемы безопасности мобильного кода}
\Sentence
В настоящее время в мобильной среде популярно использование мобильного кода и контента, что также 
в некоей степени влияет на безопасность. 
\Sentence
Гёртцель отмечает следующие проблемы, касающиеся мобильного кода и контента: установление доверия, 
проблема целостности во время доставки, безопасность среды выполенения \Reference{Winograd2008}.
\Sentence
К наиболее критичным рискам, связанным с мобильным кодом, можно отнести отказ в работе (\DoS), 
изменение данных в системе, утечка конфиденциальной информации, перехват данных 
\Reference{Bian2005}.
\Sentence
Для решения указанных проблем могут применяться следующие методы: цифровая подпись мобильного кода, 
выполнение кода в песочнице, использование средств статического и динамического анализа и ряд 
других \Reference{Rubin1998} \Reference{Bian2005} \Reference{Winograd2008} 
\Reference{OWASPMobileTopTen2011}. 

\Paragraph{Угрозы безопасности в мобильной среде}
\Sentence
Асокан приводит общую классификацию угроз безопасности в мобильной среде, основываясь на понятии 
\CIATriad (см. \ReferenceToSection{software_security_concept}) \Reference{Asokan1995}.
\Sentence
Согласно ей, угрозы безопасности делятся на категории в зависимости от того, с каким элементом 
\CIATriad они связаны. 
\Sentence
Угроза \Important{доступности} состоит в возможности проведении \Emphasis{\DoS-атаки}.
\Sentence
Угроза \Important{конфиденциальности} состоит в возможности проведении \Emphasis{анализа трафика} и  
\Emphasis{перехвата данных}.
\Sentence
Угроза \Important{целостности} состоит в возможности проведения \Emphasis{атаки посередине}, а 
также \Emphasis{захват сессии}.

\Paragraph{Переход к проблемам в пиринговых окружениях}
\Sentence
В настоящее время популярно создание спонтанных и зачастую кратковременных мобильных окружений. 
\Sentence
В этом случае каждый участник сети предоставляет информацию для других участников. 
\Sentence
Такой способ взаимодействия привёл к применению пиринговой архитектуры в мобильных окружениях. 

% -------------------------------------------------------------------------------------------------

\SubSubSectionTitle{Безопасность программного обеспечения в пиринговом окружении}
	{software_security_peer_to_peer_environment}

\Paragraph{Определение пирингового окружения}
\Sentence
Сетевое окружение является пиринговым (\Abbreviation{P2P}), если оно имеет распределённую 
архитектуру, а участники окружения разделяют часть их собственных ресурсов(вычислительную мощность, 
принтеры, сервисы и др.) \Reference{Schollmeier2001}.
\Sentence
Характерной чертой пиринговых окружений является отсутствие централизованных серверов, и участники 
окружения взаимодействуют друг с другом напрямую. 
\Sentence
При этом каждый участник сети может выступать в роли как поставщика сервисов, так и потребителя. 

\Paragraph{Виды пиринговых окружений}
\Sentence
Наиболее часто пиринговая сетевая архитектура используется в случаях осуществления обмена файлами, 
проведения распределённых вычислений, \EnglishText{overlay multicast} и др. 
\Reference{Schollmeier2001}.
%\Sentence
%В зависимости от цели использования пиринговых сетей, они могут классифицироваться как чистые и 
%гибридные, структурированные и неструктурированные \Reference{Schollmeier2001} \Reference{RFC5765}.
\Sentence
В зависимости от предназначения пирингового окружения злоумышленник может преследовать разные цели 
\Reference{RFC5765} \Reference{Wallach2002}. 
\Sentence
Низкий уровень отслеживаемости атак в таких сетях способствует лёгкому распространению вирусов и 
другого зловредного \RussianAbbreviation{ПО}. 
\Sentence
Создавая ботнет из участников пиринговой сети, злоумышленник может как проводить анализ траффика, 
так и проводить широкомасштабные \DoS-атаки. 
\Sentence
Поэтому в качестве жертвы атаки злоумышленника может являться участник сети, предоставляемый им 
сервис или передаваемые данные. 

\Paragraph{Угрозы безопасности в пиринговом окружении}
\Sentence
Чтобы понять, каким образом злоумышленник может проводить атаки в пиринговом окружении, 
необходимо рассмотреть, что может являться причиной возможности проведения атаки.
\Sentence
В области безопасности они тесно связаны с понятием угрозы.
\Sentence
Литературные источники классифицируют угрозы безопасности в зависимости от того, с какими 
элементами \CIATriad (см. \ReferenceToSection{software_security_concept}) они связаны 
\Reference{RFC5765} \Reference{Barcellos2008} \Reference{Wallach2002}. 
\Sentence
Угроза, связанная с \Emphasis{доступностью}, включают себя возможность проведения атаки, которая 
направлена на отказ какого-либо сервиса в сети или низким уровнем его функционирования. 
\Sentence
Угроза, связанная с \Emphasis{целостностью}, включают в себя возможность повреждения данных, 
предоставляемых сервисом. 
\Sentence
Стоит отметить, что этом случае злоумышленник может выдать себя в роле их источника. 
\Sentence
Наконец, к угрозе, связанной с \Emphasis{конфиденциальностью}, можно отнести тот факт, что все 
сервисы, функционирующие в сети, доступны всем без исключения участникам.

\Paragraph{Виды атак в пиринговых окружениях}
\Sentence
В связи с отличительными свойствами пирингового окружения, возможность проведения некоторых видов 
атак гораздо выше, чем в других сетевых окружениях. 
\Sentence
К таким атакам относятся \Emphasis{"человек посередине"} и \Emphasis{саморепликация} 
\Reference{Damiani2002}. 
\Sentence
По причине отсутствия централизованного сервера в пиринговых окружениях довольно сложно точно 
определить настоящего отправителя или получателя какого-либо сообщения.
\Sentence
Каждый участник сети потенциально может выдать себя за другого, что позволяет злоумышленнику 
использовать это в своих целях.  
\Sentence
Предоставляя свои ресурсы, каждый участник сети выступает при выполнении некоторых операций может 
выступать в роли посредника между потребителем и получателем.
\Sentence
В связи с данным фактом злоумышленнику, выступающему в качестве такого посредника, легко провести 
атаку типа "человек посередине". 

\Paragraph{Способы защиты от атак}
\Sentence
Для защиты от упомянутых атак был разработан ряд следующих способов. 
\Sentence
Один из них основан на раннем обнаружении злоумышленника и имеет две разновидности, в зависимости 
от того, в каком режиме происходит защита: проактивном или реактивном \Reference{RFC5765}. 
\Sentence
\Emphasis{Проактивный} режим защиты представляет собой периодическую проверку активности участников сети на её 
подозрительность. 
\Sentence
\Emphasis{Реактивный} режим защиты представляет из себя проверку активности какого-либо участника 
сети на подозрительость другим участником сети во время их взаимодействия. 
\Sentence
Принципиально другой способ защиты от атак основан на системе репутации всех участников сети 
\Reference{RFC5765} \Reference{Damiani2002}.
\Sentence
В этом случае участник сети, ведущий злоумышленную деятельность, имеет низкую репутацию, о чём 
будут знать другие обычные участники. 
