\Paragraph{Введение}
%
Исследование, проводимое в данной работе, связано с поиском уязвимостей в мобильном пиринговом окружении. 
%
Поэтому необходимо рассмотрение не только общих уязвимостей, но и тех, которые характерны мобильным и пиринговым окружениям. 

% -------------------------------------------------------------------------------------------------

\SubSubSectionTitle{Безопасность програмнного обеспечения в мобильном окружении}
	{software_security_mobile_environment}

\Paragraph{Понятие мобильного окружения}
%
Мобильная среда отличается тем, что доступ к данным и информационным сервисам в ней осуществляется при помощи \Emphasis{портативных беспроводных устройств} \Reference{Karki2008}. 
%
В качестве таких мобильных устройств могут выступать ноутбуки, \RussianAbbreviation{КПК}, пейджеры, смартфоны и обычные мобильные телефоны и многие другие.
%
Мобильная среда характеризуется \Emphasis{гетерогенностью} вычислительного окружения, а также неустойчивостью качества сервиса, \Abbreviation{QoS}, осуществляемого нижележащей коммуникационной инфраструктурой \Reference{Davies1994}. 
%
Также отмечаются такие неотъемлемые черты данной среды, как \Emphasis{мобильность пользователей и сетевых элементов}, \Emphasis{беспроводной характер коммуникационных устройств} \Reference{Asokan1995}. 

\Paragraph{Общие проблемы безопасности в мобильном окружении}
%
С точки зрения безопасности мобильное окружение по своей природе гораздо \Emphasis{более уязвимо}, чем обычное сетевое окружение. 
%
\Emphasis{Следствием} этого является усложнение обеспечения в ней безопасности. 
%
Ли приводит следующие проблемы безопасности \Reference{Li2004}: 
\begin{description}
	\leftskip2em%
	\setlength{\itemsep}{0pt}%
	\setlength{\parsep}{0pt}%

	\item[Отсутствие границ безопасности.] Мобильное окружение становится более подверженным к таким атакам, как пассивный перехват, активное вмешательство в процесс передачи данных, утечка конфиденциальной информации, отказ в работе (\Abbreviation{DoS}).

	\item[Угроза со стороны скомпрометированных узлов.] Получение контроля над одним или несколькими узлами в сети позволит провести более широкомасштабные атаки на другие узлы. 

	\item[Отсутствие централизованного средства управления.] Приводит к усложнению обнаружения и предотвращения проведения атак по причине автономности узлов.

	\item[Ограниченное электропитание.] Способствует проведению \EnglishText{DoS}-атак и подрыву операций, выполняющихся в кооперации несколькими узлами. 
\end{description}

\Paragraph{Проблемы безопасности мобильного кода}
%
В настоящее время в мобильной среде популярно использование \Important{мобильного кода и содержимого}, что также в некоей степени влияет на безопасность. 
%
Гёртцель отмечает следующие проблемы, касающиеся мобильного кода и содержимого: установление доверия между отправителем и получателем данных, проблема целостности данных во время доставки, безопасность среды выполнения \Reference{Winograd2008}.
%
К наиболее критичным рискам, связанным с мобильным кодом, можно отнести отказ в работе (\EnglishText{DoS}), изменение данных в системе, утечка конфиденциальной информации, перехват данных \Reference{Bian2005}.
%
Для решения указанных проблем могут применяться следующие методы: цифровая подпись мобильного кода, выполнение кода в песочнице, использование средств статического и динамического анализа и ряд других \Reference{Winograd2008} \Reference{Bian2005} \Reference{Rubin1998} \Reference{OWASPMobileTopTen2011}. 
%
Цифровая подпись кода представляет из себя прилагаемую к коду информацию, позволяющую идентифицировать автора кода и проверить данный код на целостность. 
%
Выполнение кода в песочнице представляет из себя запуск приложения в специальном изолированном виртуальном окружении для уменьшения потенциальных последствий, к которым может привести запуск зловредного кода.

\Paragraph{Угрозы безопасности в мобильной среде}
%
Асокан приводит общую \Important{классификацию угроз} безопасности в мобильной среде, основываясь на понятии \Term{CIA Triad} (см. \ReferenceToSection{software_security_concept}) \Reference{Asokan1995}.
%
Согласно ей, угрозы безопасности делятся на категории в зависимости от того, с каким элементом \Term{CIA Triad} они связаны. 
%
\Emphasis{Угроза доступности} состоит в возможности проведении \EnglishText{DoS}-атаки.
%
\Emphasis{Угроза конфиденциальности} состоит в возможности проведении анализа трафика и перехвата данных.
%
\Emphasis{Угроза целостности} состоит в возможности проведения атаки посередине, а также захват сессии.

\Paragraph{Переход к проблемам в пиринговых окружениях}
%
В настоящее время популярно создание \Emphasis{спонтанных} и зачастую \Emphasis{кратковременных} мобильных окружений. 
%
В этом случае каждый участник сети предоставляет информацию для других участников. 
%
Такой способ взаимодействия привёл к применению пиринговой архитектуры в мобильных окружениях. 

% -------------------------------------------------------------------------------------------------

\SubSubSectionTitle{Безопасность программного обеспечения в пиринговом окружении}
	{software_security_peer_to_peer_environment}

\Paragraph{Определение пирингового окружения}
%
Сетевое окружение является \Important{пиринговым} (\Abbreviation{P2P}), если оно имеет распределённую архитектуру, а участники окружения разделяют часть их собственных ресурсов (вычислительную мощность, принтеры, сервисы и др.) \Reference{Schollmeier2001}.
%
Характерной чертой пиринговых окружений является \Emphasis{отсутствие централизованных серверов}, и участники окружения взаимодействуют друг с другом \Emphasis{напрямую}. 
%
При этом каждый участник сети может выступать в роли как поставщика сервисов, так и потребителя. 

\Paragraph{Виды пиринговых окружений}
%
Наиболее часто пиринговая сетевая архитектура используется в случаях осуществления обмена файлами, проведения распределённых вычислений, \EnglishText{overlay multicast} и др. \Reference{Schollmeier2001}.
%
%В зависимости от цели использования пиринговых сетей, они могут классифицироваться как чистые и гибридные, структурированные и неструктурированные \Reference{Schollmeier2001} \Reference{RFC5765}.
%
В зависимости от предназначения пирингового окружения злоумышленник может преследовать разные \Important{цели} \Reference{RFC5765} \Reference{Wallach2002}. 
%
\Emphasis{Низкий уровень отслеживаемости атак} в таких сетях способствует лёгкому распространению вирусов и другого зловредного \RussianAbbreviation{ПО}. 
%
Создавая ботнет из участников пиринговой сети, злоумышленник может как проводить \Emphasis{анализ траффика}, так и проводить \Emphasis{широкомасштабные \EnglishText{DoS}-атаки}. 
%
Поэтому в качестве \Emphasis{жертвы} атаки злоумышленника может являться участник сети, предоставляемый им сервис или передаваемые данные. 

\Paragraph{Угрозы безопасности в пиринговом окружении}
%
Чтобы понять, каким образом злоумышленник может проводить атаки в пиринговом окружении, необходимо рассмотреть, что может являться \Important{причиной} возможности проведения атаки.
%
В области безопасности они тесно связаны с понятием угрозы.
%
Литературные источники классифицируют угрозы безопасности в зависимости от того, с какими элементами \Term{CIA Triad} (см. \ReferenceToSection{software_security_concept}) они связаны \Reference{RFC5765} \Reference{Wallach2002} \Reference{Barcellos2008}. 
%
Угрозы, связанные с \Emphasis{доступностью}, включают в себя возможность проведения атаки, которая направлена на отказ какого-либо сервиса в сети или низким уровнем его функционирования. 
%
Угрозы, связанная с \Emphasis{целостностью}, включают в себя возможность повреждения данных, предоставляемых сервисом. 
%
Стоит отметить, что этом случае злоумышленник может выдать себя в роле их источника. 
%
Наконец, к угрозе, связанной с \Emphasis{конфиденциальностью}, можно отнести тот факт, что все сервисы, функционирующие в сети, доступны всем без исключения участникам.

\Paragraph{Виды атак в пиринговых окружениях}
%
В связи с отличительными свойствами пирингового окружения, возможность проведения некоторых видов \Important{атак} гораздо выше, чем в других сетевых окружениях. 
%
К таким атакам относятся \Emphasis{``человек посередине''} и \Emphasis{саморепликация} \Reference{Damiani2002}. 
%
По причине отсутствия централизованного сервера в пиринговых окружениях довольно сложно точно определить настоящего отправителя или получателя какого-либо сообщения.
%
Каждый участник сети потенциально может выдать себя за другого, что позволяет злоумышленнику использовать это в своих целях.  
%
Предоставляя свои ресурсы, каждый участник сети при выполнении некоторых операций может выступать в роли посредника между потребителем и получателем.
%
В связи с данным фактом злоумышленнику, выступающему в качестве такого посредника, легко провести атаку типа ``человек посередине''. 

\Paragraph{Способы защиты от атак}
%
Для \Important{защиты} от упомянутых атак был разработан ряд следующих способов. 
%
Один из них основан на \Emphasis{раннем обнаружении} злоумышленника и имеет две разновидности, в зависимости от того, в каком режиме происходит защита: проактивном или реактивном \Reference{RFC5765}. 
%
\Emphasis{Проактивный} режим защиты представляет собой периодическую проверку активности участников сети на её подозрительность. 
%
\Emphasis{Реактивный} режим защиты представляет из себя проверку активности какого-либо участника сети на подозрительность другим участником сети во время их взаимодействия. 
%
Принципиально другой способ защиты от атак основан на \Emphasis{системе репутации} всех участников сети \Reference{RFC5765} \Reference{Damiani2002}.
%
В этом случае участник сети, ведущий злоумышленную деятельность, имеет низкую репутацию, о чём будут знать другие обычные участники. 

% -------------------------------------------------------------------------------------------------

\SubSubSectionTitle{Резюме}{software_security_summary}

\Paragraph{Безопасность в мобильном пиринговом окружении}
%
Обеспечение безопасности в мобильном пиринговом окружении достаточно трудно. 
%
Это объясняется \Emphasis{природой} самого окружения, что предоставляет злоумышленнику возможность проведения атак, в том числе широкомасштабных, а также значительно усложняет их обнаружение другими пользователями сети.

\Paragraph{Угрозы безопасности}
%
Рассматривая \Important{угрозы безопасности} в мобильном пиринговом окружении как угрозы \Term{CIA Triad}, можно сделать вывод, что угрозы \Emphasis{доступности} связаны с возможностью отказа функционирующего в окружении сервиса в работе, угрозы \Emphasis{конфиденциальности} связаны с высокой вероятностью перехвата данных и утечки конфиденциальной информации, угрозы \Emphasis{целостности} связаны с возможностью подделки передаваемых данных, захватом сесии или с выдачей злоумышленником себя за другого пользователя сети. 

\Paragraph{Атаки в мобильном пиринговом окружении}
%
Архитектура мобильного пирингового окружения позволяет злоумышленнику проводить большое количество разнообразных \Important{атак}, наиболее популярными из которых являются \Emphasis{``человек по середине''} и \Emphasis{саморепликация}. 
%
Возможность \Important{отражения} нападения от перечисленных атак на функционирующие в окружении сервисы целиком связана с уровнем их безопасности. 
%
Дополнительными мерами защиты могут являться технология цифровой подписи, введение системы репутации между участниками сети, раннее обнаружение злоумышленника с использованием анализа его действий и выполнение операций в песочнице.