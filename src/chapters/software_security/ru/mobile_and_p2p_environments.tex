\Paragraph{Введение}
\Sentence
Исследование, проводимое в данной работе, связано с поиском уязвимостей в мобильном пиринговом 
окружении. 
\Sentence
Поэтому необходимо рассмотрение не только общих уязвимостей, но и тех, которые характерны мобильным 
и пиринговым окружениям. 

\SubSubSectionTitle{Безопасность програмнного окружения в мобильном окружении}
	{software_security_mobile_environment}

\Paragraph{Понятие мобильного окружения}
\Sentence
Мобильная среда отличается mobile computing и тем, что доступ к данным и информационным сервисам 
в ней осуществляется при помощи портативных беспроводных устройств \Reference{Karki2008}. 
\Sentence
В качестве таких мобильных устройств могут выступать ноутбуки, \RussianAbbreviation{КПК}, пейджеры, 
смартфоны и обычные мобильные телефоны и многие другие.
\Sentence
Мобильная среда характеризуется гетерогенностью вычислительного окружения, а также неустойчивостью 
качества сервиса, \Abbreviation{QoS}, осуществляемого нижележащей коммуникационной инфраструктурой 
\Reference{Davies1994}. 
\Sentence
Также отмечаются такие неотъемлемые черты данной среды, как мобильность пользователей и сетевых 
элементов, беспроводной характер коммуникационных устройств \Reference{Asokan1995}. 

\Paragraph{Общие проблемы безопасности в мобильном окружении}
\Sentence
С точки зрения безопасности мобильное окружение по своей природе гораздо более уязвимо, чем обычное 
сетевое окружение. 
\Sentence
Следствием это является увеличение общей сложности обеспечения в неё безопасности. 
\Sentence
Ли приводит следующие проблемы безопасности \Reference{Li2004}: 
\begin{description}
	\leftskip2em%
	\setlength{\itemsep}{0pt}%
	\setlength{\parsep}{0pt}%

	\item[Отсутствие границ безопасности.] Мобильное окружение становится более подверженным 
		к таким атакам, как пассивный перехват, активное вмешательство в процесс передачи данных, 
		утечка конфиденциальной информации, отказ в работе (\Abbreviation{DoS}).
	\item[Угроза со стороны скомпрометированных узлов.] Получение контроля над одним или 
		несколькими узлами в сети позволит провести более широкомасштабные атаки на другие узлы. 
	\item[Отсутствие централизованного средства управления.] Приводит к усложнению обнаружения и 
		предотвращения проведения атак по причине автономности узлов.
	\item[Ограниченное электропитание.] Способствует проведению \DoS-атак и подрыву операций, 
		выполняющихся в кооперации несколькими узлами. 
\end{description}

\Paragraph{Проблемы безопасности мобильного кода}
\Sentence
В настоящее время в мобильной среде популярно использование мобильного кода и контента, что также 
в некоей степени влияет на безопасность. 
\Sentence
Гёртцель отмечает следующие проблемы, касающиеся мобильного кода и контента: установление доверия, 
проблема целостности во время доставки, безопасность среды выполенения \Reference{Winograd2008}.
\Sentence
К наиболее критичным рискам, связанным с мобильным кодом, можно отнести отказ в работе (\DoS), 
изменение данных в системе, утечка конфиденциальной информации, перехват данных 
\Reference{Bian2005}.
\Sentence
Для решения указанных проблем могут применяться следующие методы: цифровая подпись мобильного кода, 
выполнение кода в песочнице, использование средств статического и динамического анализа и ряд 
других \Reference{Rubin1998} \Reference{Bian2005} \Reference{Winograd2008} 
\Reference{OWASPMobileTopTen2011}. 

\Paragraph{Угрозы безопасности в мобильной среде}
\Sentence
Асокан приводит общую классификацию угроз безопасности в мобильной среде, основываясь на понятии 
\CIATriad (см. \ReferenceToSection{software_security_concept}) \Reference{Asokan1995}.
\Sentence
Согласно ей, угрозы безопасности делятся на категории в зависимости от того, с каким элементом 
\CIATriad они связаны. 
\Sentence
Угроза \Important{доступности} состоит в возможности проведении \Emphasis{\DoS-атаки}.
\Sentence
Угроза \Important{конфиденциальности} состоит в возможности проведении \Emphasis{анализа трафика} и  
\Emphasis{перехвата данных}.
\Sentence
Угроза \Important{целостности} состоит в возможности проведения \Emphasis{атаки посередине}, а 
также \Emphasis{захват сессии}.

\Paragraph{Переход к проблемам в пиринговых окружениях}
\Sentence
В настоящее время популярно создание спонтанных и зачастую кратковременных мобильных окружений. 
\Sentence
В этом случае каждый участник сети предоставляет информацию для других участников. 
\Sentence
Такой способ взаимодействия привёл к применению пиринговой архитектуры в мобильных окружениях. 

% -------------------------------------------------------------------------------------------------

\SubSubSectionTitle{Безопасность программного обеспечения в пиринговом окружении}
	{software_security_peer_to_peer_environment}
