\Paragraph{Что такое угроза безопасности ПО}
%
Согласно RFC 4949, \EnglishText{threat is a potential for violation of security, which exists when there is a circumstance, capability, action, or event that could breach security and cause harm} \Reference{RFC4949}. 
%
\EnglishText{That is, a threat is a possible danger that might exploit a vulnerability}.
%
Обычно угрозы классифицируются по их происхождению.

\Paragraph{Классификация}
%
Угроза безопасности может проявиться на любой стадии жизненного цикла разработки \RussianAbbreviation{ПО}. 
%
Данное обстоятельство используется при классификации угроз безопасности, исходя из их происхождения. 
%
Согласно ей, угрозы делятся на следующие категории \Reference{Winograd2008}: 
\begin{description}
	\item[Unintentional] Это может быть игнорирование практик безопасности программирования, неучитывание безопасности на стадии развёртывания, некорректное с точки зрения безопасности требование к проекту.

	\item[Intentional but not malicious] Это может быть отсутствие проверки входных пользовательских данных.

	\item[Malicious] Это может быть намеренное создание бэкдора в программе, использование паролей по умолчанию при установке приложения, возможность проведения атак внедрения.
\end{description}

\Paragraph{Переход к атакам}
%
\RussianAbbreviation{ПО} не может считаться безопасным, если оно содержит угрозы безопасности. 
%
Появление хотя бы угрозы безопасности приводит к возможности проведения атаки на данное \RussianAbbreviation{ПО} со стороны злоумышленника. 