\Paragraph{Причины появления проблем безопасности ПО}
%
Существует три основных \Important{фактора}, существенно влияющих на безопасность \RussianAbbreviation{ПО}. 
%
К ним относятся \Definition{сложность}, \Definition{расширяемость} и \Definition{способность к взаимодействию} \Reference{McGraw2004}.
%
В литературе по безопасности они известны как \Term{Trinity Of Trouble} \Reference{McGraw2006}.

\Paragraph{Фактор сложности ПО}
%
При разработке \RussianAbbreviation{ПО} \Important{сложность} приводит к большому количеству проблем, в том числе и к проблемам безопасности.
%
Это объясняется тем, что она имеет характер \Emphasis{неслучайности} и \Emphasis{нелинейности} в зависимости от размера самого \RussianAbbreviation{ПО} \Reference{Brooks1995}.
%
МакКоннелл считает сложность \Emphasis{основным императивом} программирования \Reference{McConnell2004}.
%
При росте размера разрабатываемого \RussianAbbreviation{ПО} существенно увеличивается шанс допустить ошибку, в том числе ошибку, влияющую на безопасность.
%
Данная зависимость показана на \OnReferenceToFigure{software_security_error_density}. 
%
\Remark{Объяснить график, привести пример}

\ReproducedImageFigure{Зависимость количества ошибок от размера проекта}
	{software_security_error_density}{McConnell1997}

\Paragraph{Фактор расширяемости ПО}
%
\Remark{Текст слишком тяжёл для понимания}
%
Зачастую рост проекта осуществляется при помощи средств \Important{расширяемости}. 
%
Современное \RussianAbbreviation{ПО} имеет очень высокую степень расширяемости, что благотворно сказывается на скорости разработки и более скорое появление на рынке \Reference{McGraw2004}.
%
Но в свою очередь, это приводит к риску появления \Emphasis{уязвимости} безопасности в каком-либо из расширений.
%
Тем самым усложняется процесс обеспечения безопасности разрабатываемого \RussianAbbreviation{ПО}.

\Paragraph{Фактор способности к взаимодействию ПО}
%
\Important{Способность \RussianAbbreviation{ПО} к взаимодействию} также существенно влияет на его безопасность.
%
Появление уязвимости в таком \RussianAbbreviation{ПО} ставит под угрозу безопасность всех установленных его копий и подключённых к сети \Reference{McGraw2006} \Reference{McGraw2004}. 
%
Это способствует росту числа возможных \Emphasis{атак}, а также простоте их проведения в автоматизированном способом \Reference{ISO17799}.
%
Поэтому атаки, связанные с способностью атакуемого \RussianAbbreviation{ПО} к взаимодействию, могут иметь характер \Emphasis{массовости}.

\Paragraph{Второстепенные факторы безопасности ПО}
%
Существуют также \Important{второстепенные факторы}, влиящие на безопасность \RussianAbbreviation{ПО}.
%
К ним относятся используемые при разработке принципы и практики, средства разработки, приобретённые сторонние компоненты, среда выполнения и другие \Reference{Winograd2008}. 
%
Игнорирование любого из факторов безопасности может привести к серьёзным последствиям, а само \RussianAbbreviation{ПО} может быть подвержено \Emphasis{угрозе безопасносности}. 