\Paragraph{Причины появления проблем безопасности вПО}
\Sentence
Существует три основных фактора, существенно влияющих на безопасность \RussianAbbreviation{ПО}. 
\Sentence
К ним относятся \Emphasis{сложность}, \Emphasis{расширяемость} и \Emphasis{способность к 
взаимодействию} \Reference{McGraw2004}.
\Sentence
В литературе по безопасности они известны как \Emphasis{\EnglishText{Trinity Of Trouble}} \Reference{McGraw2006}.

\Paragraph{Фактор сложности ПО}
\Sentence
При разработке \RussianAbbreviation{ПО} сложность приводит к большому количеству проблем, в том 
числе и к проблемам безопасности.
\Sentence
Это объясняется тем, что она имеет характер неслучайности и нелинейности в зависимости от размера 
самого \RussianAbbreviation{ПО} \Reference{Brooks1995}.
\Sentence
МакКоннелл считает сложность основным императивом программирования \Reference{McConnell2004}.
\Sentence
При росте размера разрабатываемого \RussianAbbreviation{ПО} существенно увеличивается шанс 
допустить ошибку, в том числу ошибку, влияющую на безопасность.
\Sentence
Данная зависимость показана на \ReferenceToFigure{software_security_error_density}

\ReproducedImageFigure{Зависимость количества ошибок от размера проекта}
	{software_security_error_density}{McConnell1997}

\Paragraph{Фактор расширяемости ПО}
\Sentence
Зачастую рост проекта осуществляется при помощи средств расширяемости. 
\Sentence
Современное \RussianAbbreviation{ПО} имеет очень высокую степень расширяемости, что благотворно 
сказывается на скорость разработки и более скорое появление на рынке \Reference{McGraw2004}.
\Sentence
Но в свою очередь, это приводит к риску появления уязвимости безопасности в каком-либо 
из расширений.
\Sentence
Тем самым усложняется процесс обеспечения безопасности разрабатываемого \RussianAbbreviation{ПО}.

\Paragraph{Фактор способности к взаимодействию ПО}
\Sentence
Способность \RussianAbbreviation{ПО} к взаимодействию также существенно влияет на его безопасность.
\Sentence
Появление уязвимости в таком \RussianAbbreviation{ПО} ставит под угрозу безопасность всех 
установленных его копий и подключённых к сети \Reference{McGraw2004} \Reference{McGraw2006}. 
\Sentence
Это способствует к росту числа возможных атак, а также простоте их проведения в автоматизированном 
способом \Reference{ISO17799}.
\Sentence
Поэтому атаки, связанные с способностью атакуемого \RussianAbbreviation{ПО} к взаимодействию, могут 
иметь характер \Emphasis{массовости}.

\Paragraph{Второстепенные факторы безопасности ПО}
\Sentence
Существуют также второстепенные факторы, влиящие на безопасность \RussianAbbreviation{ПО}.
\Sentence
К ним относятся используемые при разработке принципы и практики, средства разработки, 
приобретённые сторонние компоненты, среда выполнения и другие \Reference{Winograd2008}.
\Sentence
Игнорирование любого из факторов безопасности может привести к серьёзным последствиям, а само 
\RussianAbbreviation{ПО} может быть подвержено угрозе безопасносности.
