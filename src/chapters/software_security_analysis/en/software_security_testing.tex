\Paragraph{The software security analysis concept}
%
Software security analysis is a variety of software testing. 
%
But unlike other varieties of software testing it is not based on requirements to software, it is not functional \Reference{Winograd2008}. 
%
The main reason of this is the fact that success of tests and fulfilment of all requirements can not say that \The tested software does not contain security vulnerabilities. 
%
This is a complexity of assessment of security of \A software. 

\Paragraph{Software security analysis characteristics}
%
Software security testing is intended for search of vulnerablities, not traditional errors. 
%
They are differed qualitatively. 
%
Traditional errors are detected by developer accidentally and generally does not influence on other users. 
%
On the contrary, unlike a normal user a malicious person has some security knowledge and experience, that is interacts with software differently and looks for security defectes intentionally. 
%
Consequences of his futher attack often influences on other users utilizing the attacking software. 
%
For this reason, software security testing is differed to traditional testing methods. 
%
Michael enumerates the following differences \Reference{Michael2005}: 
\begin{itemize}
	%\leftskip2em%
	\setlength{\itemsep}{0pt}%
	%\setlength{\parsep}{0pt}%

	\item code operating properly is not alwasy secure
	\item most of security requirements can not be checked using traditional software testing methods
	\item software security testing is intended for checking the fact that software should not do, whereas traditional software testing methods are intended for checking correctness of functionality of the software. 
\end{itemize}

\Paragraph{The goals of software security analysis}
%
From a formal point of view software security testing consists in verification of security-related characteristics. 
%
They are following \Reference{Goertzel2007}: 
\begin{enumerate}
	%\leftskip2em%
	\setlength{\itemsep}{0pt}%
	%\setlength{\parsep}{0pt}%

	\item behaviour of software is predictable and secure
	\item software does not contain known security vulnerabilities
	\item software is tolerant to defects and operates correctly in case of exception
	\item software meets all non-function security requirements
	\item software does not violate no one security factor
\end{enumerate}

\Paragraph{Software security analysis approaches}
%
There are two approaches to make software security analysis: testing of software security mechanisms and risk-based security testing (simulation of attacker's behaviour) \Reference{Potter2004}. 
%
The first approach does not differ to traditional testing methods but is aimed to security mechanisms. 
%
The second one is intended to decrease a level of risks to detect security vulnerability in software. 

\Paragraph{Software security analysis characteristics}
%
While performing risk-based software security testing IA Triad principles (see \ReferenceToSection{software_security_concept}) can be utilized. 
%
The security analysis is performed in terms of security risks and a set of security requirements, if they exist. 
%
Sucessful peforming of security analysis, meanwhile, is quite difficult because it requires taking into account many details. 
%
Moreover, a person performing the analysis should be high-skilled. 

\Paragraph{Which activities are performed during software security analysis}
%
Risk-based software security analysis consists in source code revision and utilization of various software security testing methods using appropriate tools \Reference{Goertzel2007}. 
%
Source code revision is performed from \An attacker position and aimed at most important components and interfaces beetween components, including interfaces for plug-ins \Reference{Winograd2008}. 
%
Software security methods are applied to testing techniques based on ``white-box'', ``gray-box'', and ``black-box'' principles. 

\Paragraph{White-box, gray-box, and black-box testing methods}
%
The methods mentioned above have such names because they depend on artifacts available at the security analysis performing. 
%
While performing black-box testing only binary code is applied. 
%
While performing gray-box testing both source and binary code are applied. 
%
Finally, While performing white-box testing only source code is applied. 