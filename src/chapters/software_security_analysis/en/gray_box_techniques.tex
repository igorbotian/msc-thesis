\Paragraph{The gray box testing concept}
%
Gray box testing, in some sense, is a hybrid solution that involves elements of white box and black box testing techniques. 
%
Therefore it can be utilized as an additional tool for these kind of testing techniques. 
%
It's main advantages are availability and high degree of code coverage by tests \Reference{Sutton2007}. 
%
The main disadvantage consists in the fact that gray box testing is quite difficult to perform \Reference{Sutton2007}. 

% ------------------------------------------------------------------------------------------------

\SubSubSectionTitle{Dynamic code analysis}
	{software_security_analysis_gray_box_techniques_dynamic_code_analysis}

\Paragraph{What is dynamic code analysis}
%
Dynamic code analysis by it's nature is execution of tested software using special tools. 
%
The goal is to detect abnormal actions and to notify about them instantly \Reference{Winograd2008}. 
%
General operational principle of these tools is to run tested software in special environment that is used as a layer between the tested application and it's environment. 
%
After that, the tools analyze all operations performed by the tested application. 
%
If source code of the application is available, information about location in source code of each anomaly is displayed. 
%
Gray box testing tools often can exactly recognize an anomaly as a consequence of an error. 

\Paragraph{Characteristics}
%
As opposed to static analysis, dynamic analysis makes it possible for a tester to search security vulnerabilities which can be detected by interaction of tested software with user, environment, or it's own component. 
%
By performing dynamic code analysis, one can find the following kind of errors: buffer overflow, format string function vulnerabilities, pointer vulnerabilities, and others \Reference{Newsome2005}. 
%
The following well-known tools can be utilized: \IT{StackGuard} \WebSite{StackGuard}, \IT{Libsafe} \WebSite{Libsafe}, \IT{FormatGuard} \WebSite{FormatGuard}, \IT{Valgrind} \WebSite{Valgrind}, \IT{Helgrind} \WebSite{Helgrind}. 

% ------------------------------------------------------------------------------------------------

\SubSubSectionTitle{Source code fault injection}
	{software_security_analysis_gray_box_techniques_fault_injection}

\Paragraph{What is source code fault injection}
%
Source code fault injection is a method used to test fault-tolerance of software \Reference{Madeira2000}. 
%
The method by it's nature consists in manual injection of various defects to source code of tested software. 
%
The aim of this is, for example, to verify correctness of exception processing strategy \Reference{Winograd2008}. 
%
Manual fault injection makes it possible to check correctness of work of both whole tested application and it's component under conditions of fault existence in environment or in one of it's other component. 
%
Software fault emulation makes it possible to determine degree of fault-tolerance of tested software. 

\Paragraph{What it makes it possible to detect}
%
Precise fault emulation makes it possible to determine aftereffects of an error presence. 
%
Threrefore it can be said that support of precise fault emulation is the main task of this kind of testing \Reference{Madeira2000}. 
%
Fault injection is useful in case of searching errors, such as incorrect usage of pointers and arrays, race conditions, and others. 
%
The method makes it possible to find erros at development stage. 

\Paragraph{Disadvantages}
%
By-turn, the method has some disadvantages \Reference{Hsueh1997}. 
%
It can not be utilized if source code of tested software is not available. 
%
Other disadvantages are related to possibility of normal work-load violation and changes in structure of original copy of tested software. 