\Paragraph{Secure that should be secure}
%
The security problem is one of the most critical for a variety of software. 
%
A user trust such software to perform operations most important from security point of view, for example, processing of private data \Reference{Winograd2008}. 
%
Examples of software that should be secure are following: servers available from network, web applications, applets, daemons, applications that have appropriate access rights to perform privileged operations as well. 

\Paragraph{The concept of secure software}
%
To be secure software should be designed, implemented, configured, and maintained in such a way as to operate properly under attack of a malicious person and to reduce effects of software defects that are independent from the software \Reference{Goertzel2007}. 
%
Secure software has the following properties: reliability, trustworthiness and reducibility \Reference{Winograd2008}. 
%
Reliable software operates properly under any circumstances\T{,} including attacks. 
%
Software that has trustworthiness property does not contain operations that could be used to injury the software by a malicious person. 
%
Software that has reduciblity property is tolerant to attacks and can recover it's proper operating after performing of an attack. 

\Paragraph{The concept of software security assurance}
%
As much developer takes into consideration these factors as more secure software are developed by him. 
%
It allows him to engage security assurance activities related to the software. 
%
According \Reference{Goertzel2007}, software security assurance consist of ``a planned and systematic set of activities that are intended for meeting by the software security requirements, standards and procedures''. 
%
McGraw gives three pillars of software security: risk management, developer's knowledge about software security, and compliance with practices specific to security \Reference{McGraw2006}.
%
Software security assurance are based on them. 

\Paragraph{Transition to software security analysis}
%
As it was noted in \InReferenceToSection{introduction_main_objectives}, the practical part of the thesis is comprised of software vulnerability search. 
%
It is explained by the fact that the observable project is on the testing stage of it's development lifecycle. 
%
Software vulnerability search is performed by means of it's security analysis. 