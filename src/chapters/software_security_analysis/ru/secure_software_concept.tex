\Paragraph{\RussianAbbreviation{ПО}, которое должно быть защищённым}
%
Существует большое количество различных видов \RussianAbbreviation{ПО}, для которых безопасность является наиболее критичной. 
%
Такому \RussianAbbreviation{ПО} пользователь доверяет выполнение наиболее важных с точки зрения безопасности операций, например, обработку конфиденциальных данных \Reference{Winograd2008}. 
%
Примерами \RussianAbbreviation{ПО}, которое должно быть безопасным, являются доступные из сети серверы, веб-приложения, апплеты, демоны, а также программы, которые имеют соответствующие права доступа для выполнения привилегированных операций \Reference{Wheeler2003}. 
%
Такое \RussianAbbreviation{ПО} должно быть защищённым.

\Paragraph{Понятие защищённого \RussianAbbreviation{ПО}}
%
Чтобы быть защищённым, \RussianAbbreviation{ПО} должно быть спроектировано, реализовано, сконфигурировано и поддержваться таким образом, чтобы оно могло корректно функционировать во время проведения атаки со стороны злоумышленника и ограничить последствия от ошибок, независящих от данного \RussianAbbreviation{ПО} \Reference{Goertzel2007}. 
%
Защищённое \RussianAbbreviation{ПО} обладает свойствами надёжности, гарантии безопасности и способности к восстановлению \Reference{Winograd2008}. 
%
Надёжное \RussianAbbreviation{ПО} корректно функционирует при любых условиях, включая атаку со стороны злоумышленника. 
%
\RussianAbbreviation{ПО}, гарантирующее безопасность, не содержит операций, с помощью которых злоумышленник может нанести вред. 
%
\RussianAbbreviation{ПО}, способное к восстановлению, является стойким к проведению атак и может восстановить нормальное функционирование системы после проведения атаки.

\Paragraph{Факторы, влияющие на защищённость \RussianAbbreviation{ПО}}
%
Ряд факторов влияет на степень защищённости \RussianAbbreviation{ПО} \Reference{Winograd2008}. 
%
К ним относятся принципы и практики разработки, инструменты для разработки, приобретённые сторонние компоненты, конфигурация развёртывания, программное окружение, а также опыт самого разработчика.

\Paragraph{Понятие обеспечения безопасности \RussianAbbreviation{ПО}}
%
Насколько разработчик учитывает перечисленные выше факторы, тем более защищённое \RussianAbbreviation{ПО} он разрабатывает. 
%
Это позволяет ему проводить мероприятия по обеспечению безопасности разрабатываемого им \RussianAbbreviation{ПО}. 
%
Согласно \Reference{Goertzel2007}, обеспечение безопасности \RussianAbbreviation{ПО} "состоит из спланированного и систематического ряда мероприятий, которые направлены на то, что разрабатываемое \RussianAbbreviation{ПО} удовлетворяет требованиям, стандартам и процедурам безопасности".
%
МакГроу приводит три столпа безопасности \RussianAbbreviation{ПО}, на которых основывается обеспечение безопасности \RussianAbbreviation{ПО}: управление рисками, знание разработчика о безопасности и следование практикам, направленным на увеличение безопасности \Reference{McGraw2006}.

\Paragraph{Переход к анализу безопасности}
%
Как было указано в \InReferenceToSection{introduction_main_objectives}, в данной работе практическая часть состоит из поиска уязвимостей безопасности \RussianAbbreviation{ПО}. 
%
Это объясняется тем, что исследуемый проект находится на стадии раннего тестирования. 
%
Поиск уязвимостей безопасности \RussianAbbreviation{ПО} проводится путём анализа его безопасности.