\Paragraph{Понятие тестирование по методу серого ящика}
%
Тестирование по принципу ``серого'' ящика в некоем смысле является \Emphasis{решением-гибридом}, которое включает в себя элементы тестирования по методам ``белого'' и ``чёрного'' ящиков. 
%
Поэтому оно может использоваться в качестве дополнительного средства при проведении данных видов тестирования \RussianAbbreviation{ПО}. 
%
Основными достоинствами тестирования по методу серого ящика являются доступность и высокая степень покрытия кода тестами \Reference{Sutton2007}. 
%
К основному недостатку данного вида тестирования можно отнести сложность его проведения \Reference{Sutton2007}.

% ------------------------------------------------------------------------------------------------

\SubSubSectionTitle{Динамический анализ кода}
	{software_security_analysis_gray_box_techniques_dynamic_code_analysis}

\Paragraph{Что такое динамическй анализ кода}
%
Динамический анализ кода представляет из себя \Emphasis{запуск} тестируемого \RussianAbbreviation{ПО} с помощью специальных инструментальных средств. 
%
Его \Important{целью} является  обнаружение каких-либо аномальных действий и мгновенное извещение пользователя об этом \Reference{Winograd2008}. 
%
Общий \Important{принцип работы} инструментальных средств такого вида заключается в запуске \RussianAbbreviation{ПО} в специальном окружении, служащим прослойкой между самим приложением и его программным окружением, и дальнейшем анализе действий, которое выполняет приложение. 
%
При наличии исходного кода пользователю также выдаётся информация о местоположении в коде, в котором обнаружена аномалия. 
%
Зачастую инструментальное средство имеет возможность точного определения данной аномалии как последствие ошибки.

\Paragraph{Характеристика}
В отличие от статического анализа динамический анализ позволяет тестировщику проверить \RussianAbbreviation{ПО} на наличие уязвимостей, которые могут быть обнаружены при взаимодействии с пользователем, программным окружением или собственным компонентом. 
%
С помощью проведения динамического анализа исходного кода можно выявить такие ошибки, как переполнение буфера, уязвимость форматных строк, уязвимость указателей и другие \Reference{Newsome2005}. 
%
Для этого можно применить такие инструменты, как \IT{StackGuard} \WebSite{StackGuard}, \IT{Libsafe} \WebSite{Libsafe}, \IT{FormatGuard} \WebSite{FormatGuard}, \IT{Valgrind} \WebSite{Valgrind}, \IT{Helgrind} \WebSite{Helgrind}.

% ------------------------------------------------------------------------------------------------

\SubSubSectionTitle{Внедрение неисправностей в исходный код}
	{software_security_analysis_gray_box_techniques_fault_injection}

\Paragraph{Что такое внедрение неисправностей в исходный код}
%
Внедрение неисправностей в исходный код является методом, используемым для тестирования \Emphasis{отказоустойчивости} \RussianAbbreviation{ПО} \Reference{Madeira2000}. 
%
Метод предполагает искусственное внесение разного рода неисправностей в исходный код тестируемого приложения. 
%
Основной \Important{целью} такого действия является, например, проверка корректности механизма обработки исключительных ситуаций \Reference{Winograd2008}. 
%
Искусственное внесение неисправностей позволяет проверить корректность работы как всего приложения, так и его отдельного компонента в условии существования неисправности в программном окружении или в одном из компонентов самого приложения. 
%
Эмуляция неисправностей \RussianAbbreviation{ПО} позволяет оценить степень отказоустойчивости тестируемого приложения.

\Paragraph{Что позволяет определить}
%
Точная эмуляция неисправностей позволяет определить \Emphasis{последствия от ущерба} таких ошибок. 
%
Поэтому можно сказать, что обеспечение точности эмуляции является основной задачей при внедрении неисправностей \Reference{Madeira2000}. 
%
Внедрение неисправностей полезно для поиска таких ошибок, как некорректное использование указателей и массивов, состояние гонки и другие. 
%
Метод внедрения неисправностей позволяет находить ошибки уже на стадии разработки, поэтому часто применяется во время написания исходного кода.

\Paragraph{Недостатки}
%
В свою очередь, данный метод имеет ряд \Important{недостатков} \Reference{Hsueh1997}. 
%
Он неприменим, если невозможно внести изменение в компонент или программное окружение при отсутствии исходного кода. 
%
Другие недостатки связаны с возможностью нарушения рабочей нагрузки приложения и с изменения структуры оригинального \RussianAbbreviation{ПО}.