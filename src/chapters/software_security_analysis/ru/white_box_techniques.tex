\Paragraph{Понятие тестирования по методу белого ящика}
%
Как уже было упомянуто выше, для тестирования \RussianAbbreviation{ПО} по методу ''белого ящика'' необходимо \Emphasis{наличие исходного кода} тестируемого \RussianAbbreviation{ПО}. 
%
Методы тестирования, использующие исходный код, наиболее полезны и популярны на \Emphasis{этапе разработки \RussianAbbreviation{ПО}}. 
%
К таким методам относятся статический анализ исходного кода и тестирование, основанное на свойствах \Reference{Winograd2008}.

\Paragraph{Достоинства и недостатки}
%
Тестирование по методу белого ящика имеет свои \Important{достоинства} и \Important{недостатки} \Reference{Sutton2007}. 
%
Основным \Emphasis{достоинством} тестирования по методу белого ящика является высокая степень покрытия исходного кода благодаря его доступности. 
%
Анализируются все возможные пути выполнения программы с целью поиска потенциальных уязвимостей. 
%
Основным же \Emphasis{недостатком} является сложность проведения такого анализа. 
%
Не редким является ложное срабатывание статического анализатора, что особенно актуально при большом размере тестируемого \RussianAbbreviation{ПО}.

% -------------------------------------------------------------------------------------------------

\SubSubSectionTitle{Статический анализ кода}
	{software_security_analysis_blackbox_testing_techniques_static_code_analysis}

\Paragraph{Проблема защитного программирования}
%
Осуществляя безопасность входных данных на всех уровнях разрабатываемого \RussianAbbreviation{ПО}, применяется \Emphasis{техника защитного программирования}. 
%
Она способствует более быстрому обнаружению ошибок и повышению безопасности \RussianAbbreviation{ПО} в целом. 
%
Но её применение является недостаточным \Reference{Chess2007}. 
%
Оно направлено лишь на проверку соблюдения контракта вызывающей стороной. 
%
Безопасность передаваемых данных зачастую не учитывается. 

\Paragraph{Пример проблемы защитного программирования}
%
В листинге ниже представлен код функции \SourceCode{log()}, осуществляющей запись сообщений в файл. 
%
Специально сформированное злоумышленником значение аргумента функции \SourceCode{msg} может являться корректным, но приводящим к уязвимости форматной строки \SourceCode{fprintf()} в теле функции (см. \ReferenceToSection{software_security_vulnerabilities_format_strings}). 
%
Уязвимости такого вида могут быть найдены при помощи средств статического анализа.

\Listing{Пример небезопасной программы}{C}
	{listings/software_security_analysis/static_analysis.c}

\Paragraph{Понятие}
%
Проведение статического анализа \Important{направлено} на поиск различного рода проблем, таких как ошибок, допущенных по случайности или недосмотру со стороны программиста, общих ошибок программирования, в том числе и уязвимостей безопасности \Reference{Chess2007}. 
%
Статический анализ кода помогает снизить общее количество ошибок ещё на этапе разработки 
\RussianAbbreviation{ПО}.

\Paragraph{Статический анализ и ревизия кода}
%
Проведение статического анализа кода очень часто используется в процесе \Emphasis{ревизии кода}. 
%
В этом случае процесс ревизии кода выполняется последовательно и состоит из следующих шагов: устанавливаются цели ревизии, используются средства статического анализа, а затем по отчётам их работы проводится сама ревизия кода с последующем устранением найденных ошибки \Reference{Chess2007}. 

\Paragraph{Недостатки статического анализа}
%
Проведение статического анализа имеет ряд \Important{недостатков} \Reference{Winograd2008} \Reference{Chess2007}. 
%
Статический анализ может использоваться лишь в роле вспомогательного средства поиска ошибок, так как отсутствие ошибок в отчёте не может говорить о том, что их нет в тестируемом приложении. 
%
Например, к таким ошибкам относятся ошибки, которые явно не присутствуют в коде и проявляются лишь во время выполнения приложения. 
%
К тому отчёт, полученный инструментом статического анализа, зачастую содержит большое количество ложных срабатываний. 
%
Поэтому его анализ должен выполнить специалист, компетентный в области безопасности \RussianAbbreviation{ПО}.

\Paragraph{Известные инструменты статического анализа}
%
Наиболее известными \Important{инструментами} статического анализа являются \IT{Splint} \WebSite{Splint}, \IT{Cppcheck} \WebSite{Cppcheck}, \IT{RATS} \WebSite{RATS}, \IT{ITS4} \WebSite{ITS4}, \IT{PVS Studio} \WebSite{PVSStudio} и другие. 
%
\IT{RATS} и \IT{PVS Studio} могут быть использованы в качестве инструментов поиска ошибок параллелизма, в том числе состояние гонки. 
%
\IT{ITS4} и \IT{Splint} направлены на поиск распространённых уязвимостей, специфичных для языков \IT{C} \Reference{CStandard1999} и \IT{C++} \Reference{CppStandard2003}, например, переполнение буфера, уязвимости форматной строки и другие (см. \ReferenceToSection{software_security_vulnerabilities_format_strings}).

% -------------------------------------------------------------------------------------------------

\SubSubSectionTitle{Тестирование, основанное на свойствах}
	{software_security_analysis_blackbox_testing_techniques_property_based_testing}

\Paragraph{Описание}
%
\Important{Целью} любого вида тестирования безопасности является проверка приложений на надёжность и обеспечение его функций. 
%
Другими словами, происходит проверка семантических свойств поведения приложения. 
%
Данный вид тестирования является разновидностью техники формального анализа, поэтому применяется уже после того, как реализована функциональность приложения \Reference{Winograd2008}.

\Paragraph{Как оно проводится}
%
Для проведения тестирования, основанного на свойствах, выполняются \Important{следующие шаги} 
\Reference{Fink1997}. 
%
Сначала анализируется \Emphasis{спецификация} тестируемого приложения. 
%
На основе спецификации тестируемого приложения составляется \Emphasis{список свойств приложения}, которые необходимо проверить. 
%
Например, корректность аутентификации пользователя. 
%
Далее формализуется проверка данного свойства, и составляется тест, который её выполняет. 
%
Проверка выполняется с использованием исходного кода.

\Paragraph{Недостатки}
%
Основным \Important{недостатком} тестирования, основанного на свойствах, является то, что его проведение может занимать большое количество времени \Reference{Winograd2008}. 
%
Особенно, если анализируемое \RussianAbbreviation{ПО} имеет большой размер. 
%
Причиной этого является требование к строгой формализации тестов.