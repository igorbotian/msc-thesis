\Paragraph{Понятие анализа безопасности ПО}
%
\Important{Анализ безопасности \RussianAbbreviation{ПО}} представляет собой разновидность тестирования \RussianAbbreviation{ПО}, но его принципиальным отличием является то, что оно не основывается на требованиях и не является функциональным \Reference{Winograd2008}. 
%
Причиной этому является то, что успешность проведения тестов и удовлетворение всех требований не может говорить о том, что тестируемое \RussianAbbreviation{ПО} не содержит уязвимостей. 
%
В этом состоит сложность оценки качества проводимого тестирования. 

\Paragraph{Особенности анализа безопасности ПО}
%
Тестирование безопасности \RussianAbbreviation{ПО} направлено на \Important{поиск уязвимостей}, не традиционных ошибок.
% 
Они имеют качественные различия. 
%
Традиционные ошибки обнаруживаются пользователем случайно и обычно не влияют на работу других пользователей.
%
Напротив, злоумышленник в отличие от пользователя имеет опыт и знания о безопасности, поэтому взаимодействует с \RussianAbbreviation{ПО} принципиально другим способом и намеренно выполняет поиск дефектов безопасности. 
%
Последствия его атаки зачастую влияют на работу других пользователей атакуемого им \RussianAbbreviation{ПО}.
%
Поэтому тестирование безопасности \RussianAbbreviation{ПО} имеет \Emphasis{ряд отличий} от традиционных способов тестирования. Майкл перечисляет следующие \Reference{Michael2005}:
\begin{itemize}
	%\leftskip2em%
	\setlength{\itemsep}{0pt}%
	%\setlength{\parsep}{0pt}%

	\item корректно функционирующий код не всего является безопасным
	
	\item многие требования безопасности не могут быть проверены при помощи традиционных способов тестирования \RussianAbbreviation{ПО}

	\item тестирование безопасности \RussianAbbreviation{ПО} направлено на проверку того, что приложение не должно делать, тогда как традиционные способы тестирования \RussianAbbreviation{ПО} направлены лишь на проверку корректности функциональности тестируемого \RussianAbbreviation{ПО}
\end{itemize}

\Paragraph{Цели анализа безопасности ПО}
%
С формальной точки зрения тестирование безопасности \RussianAbbreviation{ПО} заключается в \Emphasis{верификации характеристик}, связанных с безопасностью. 
%
К ним относятся \Reference{Goertzel2007}:
\begin{enumerate}
	%\leftskip2em%
	\setlength{\itemsep}{0pt}%
	%\setlength{\parsep}{0pt}%

	\item Поведение \RussianAbbreviation{ПО} прогнозируемо и безопасно
	
	\item \RussianAbbreviation{ПО} не содержит известные уязвимости безопасности
	
	\item \RussianAbbreviation{ПО} является стойким к ошибкам и корректно ведёт себя в случае исключительных ситуаций

	\item \RussianAbbreviation{ПО} удовлетворяет всем нефункциональным требованиям безопасности

	\item \RussianAbbreviation{ПО} не нарушает ни один из факторов безопасности
\end{enumerate}

\Paragraph{Подходы к проведению анализа безопасности ПО}
%
Существует два \Important{подхода} к проведению анализа безопасности \RussianAbbreviation{ПО}: тестирование механизмов безопасности на корректность работы и тестирование \RussianAbbreviation{ПО} на безопасность, основанное на рисках (симулирование поведения злоумышленника) \Reference{Potter2004}. 
%
Первый из них ничем не отличается от проведения обычного тестирования, но направленного на механизм безопасности. 
%
Второй из них направлен на снижение рисков нахождения уязвимостей в \RussianAbbreviation{ПО}.

\Paragraph{Характеристики анализа безопасности ПО}
%
При проводении тестирования безопасности \RussianAbbreviation{ПО} на \Important{основе рисков} возможно применение принципов \Term{CIA Triad} (см. \ReferenceToSection{software_security_concept}). 
%
Анализ безопасности проводится исходя из \Emphasis{рисков безопасности} и \Emphasis{списка необходимых требований к безопасности}, если они есть. 
%
При этом успешное проведение анализа безопасности довольно сложно, так как оно требует учёта множества деталей. 
%
Более того, со стороны лица, проводящего анализ, также требуется высокая квалификация.

\Paragraph{Какие мероприятия проводятся при анализе ПО на безопасность}
%
Тестирование безопасности \RussianAbbreviation{ПО}, основанное на рисках, заключается в \Emphasis{ревизии исходного кода} и использовании различных \Emphasis{методов тестирования безопасности} с применением соответствующих инструментов \Reference{Goertzel2007}. 
%
Ревизия исходного кода проводится с позиции злоумышленника и направлена в первую очередь на компоненты, имеющих высокое значение, и интерфейсы между компонентами, включая интерфейсы для расширений \Reference{Winograd2008}. 
%
К методам тестирования безопасности относятся техники тестирования по принципу ``белого, серого и чёрного ящиков''.

\Paragraph{Понятия тестирования по методу белого, серого и чёрного ящиков}
%
Упомянутые выше методы имеют такие названия, так как исходят из того, какие \Emphasis{артефакты \RussianAbbreviation{ПО}} доступны при проведении анализа безопасности. 
%
При проведении тестирования \RussianAbbreviation{ПО} по методу \Emphasis{``чёрного ящика''} используется только бинарный код тестируемого \RussianAbbreviation{ПО}. 
%
Для проведении тестирования \RussianAbbreviation{ПО} по методу \Emphasis{``серого ящика''} необходимо наличие как исходного кода, так и бинарного тестируемого \RussianAbbreviation{ПО}. 
%
Наконец, при проведении тестирования \RussianAbbreviation{ПО} по принципу \Emphasis{``белого ящика''} используется исходный код тестируемого \RussianAbbreviation{ПО}.