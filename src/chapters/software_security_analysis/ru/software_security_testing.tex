\Paragraph{Понятие анализа безопасности ПО}
%
Анализ безопасности \RussianAbbreviation{ПО} представляет собой разновидность тестирования 
\RussianAbbreviation{ПО}, но его принципиальным отличием является то, что оно не основывается на 
требованиях и не является функциональным \Reference{Winograd2008}. 
%
Причиной этому является то, что успешность проведения тестов и удовлетворение всех требований 
не может говорить о том, что тестируемое \RussianAbbreviation{ПО} не содержит уязвимостей. 
%
В этом состоит сложность оценки качества проводимого тестирования. 

\Paragraph{Цели анализа безопасности ПО}
%
С формальной точки зрения тестирование безопасности \RussianAbbreviation{ПО} заключается 
в верификации характеристик, связанных с безопасностью. 
%
К ним относятся \Reference{Goertzel2007}:
\begin{enumerate}
	%\leftskip2em%
	\setlength{\itemsep}{0pt}%
	%\setlength{\parsep}{0pt}%

	\item Поведение \RussianAbbreviation{ПО} прогнозируемо и безопасно
	\item \RussianAbbreviation{ПО} не содержит известные уязвимости безопасности
	\item \RussianAbbreviation{ПО} является стойким к ошибкам и корректно ведёт себя в случае 
		исключительных ситуаций
	\item \RussianAbbreviation{ПО} удовлетворяет всем нефункциональным требованиям безопасности
	\item \RussianAbbreviation{ПО} не нарушает ни один из факторов безопасности
\end{enumerate}

\Paragraph{Подходы к проведению анализа безопасности ПО}
%
Существует два подхода к проведению анализа безопасности \RussianAbbreviation{ПО}: тестирование 
механизмов безопасности на корректность работы и тестирование \RussianAbbreviation{ПО} 
на безопасность, основанное на рисках (симулирование поведения злоумышленника) 
\Reference{Potter2004}. 
%
Первый из них ничем не отличается от проведения обычного тестирования, но направленного на механизм 
безопасности. 
%
Второй из них направлен на снижение рисков нахождения уязвимостей в \RussianAbbreviation{ПО}.

\Paragraph{Характеристики анализа безопасности ПО}
%
При проводении тестирования безопасности \RussianAbbreviation{ПО} на основе рисков возможно 
применение принципов \CIATriad (см. \ReferenceToSection{software_security_concept}). 
%
Анализ безопасности проводится исходя из рисков безопасности и списка необходимых требований 
к безопасности, если они есть. 
%
При этом успешное проведение анализа безопасности довольно сложно, так как оно требует учёта 
множества деталей. 
%
Более того, со стороны лица, проводящего анализ, также требуется высокая квалификация.

\Paragraph{Какие мероприятия проводятся при анализе ПО на безопасность}
%
Тестирование безопасности \RussianAbbreviation{ПО}, основанное на рисках, заключается в ревизии 
исходного кода и использовании различных методов тестирования безопасности с применением 
соответствующих инструментов \Reference{Goertzel2007}. 
%
Ревизия исходного кода проводится с позиции злоумышленника и направлена в первую очередь 
на компоненты, имеющих высокое значение, и интерфейсы между компонентами, включая интерфейсы 
для расширений \Reference{Winograd2008}. 
%
К методам тестирования безопасности относятся техники тестирования по принципу «белого, серого и 
чёрного ящиков».

\Paragraph{Понятия тестирования по методу белого, серого и чёрного ящиков}
%
Упомянутые выше методы имеют такие названия, так как исходят из того, какие артефакты 
\RussianAbbreviation{ПО} доступны при проведении анализа безопасности. 
%
При проведении тестирования \RussianAbbreviation{ПО} по методу «чёрного ящика» используется только 
бинарный код тестируемого \RussianAbbreviation{ПО}. 
%
Для проведении тестирования \RussianAbbreviation{ПО} по методу «серого ящика» необходимо наличие 
как исходного кода, так и бинарного тестируемого \RussianAbbreviation{ПО}. 
%
Наконец, при проведении тестирования \RussianAbbreviation{ПО} по принципу «белого ящика» 
используется исходный код тестируемого \RussianAbbreviation{ПО}.