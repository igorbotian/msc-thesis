\Paragraph{Понятие}
%
При тестировании безопасности \RussianAbbreviation{ПО} по принципу \Important{''чёрного'' ящика} используется только бинарный код. 
%
Различные методы в этом случае тестирования основаны на рассмотрении \RussianAbbreviation{ПО} как \Emphasis{единого целого}, отвечающего на входные данные выполнением заранее определённого действия и соответствующими выходными данными. 
%
Сам принцип тестирования основан на проверке \Emphasis{соответствия входных данных выходным} с точки зрения безопасности и с учётом изменения программного окружения \Reference{Winograd2008}.

\Paragraph{Достоинства}
%
Тестирование по методу ''чёрного'' ящика имеет ряд как \Important{достоинств} \Reference{Sutton2007}. 
%
Основным достоинством является \Emphasis{доступность}, так как его можно использовать как при наличи, так и отсутствии исходного кода тестируемого \RussianAbbreviation{ПО}. 
%
Другим достинством является его лёгкая \Emphasis{воспроизводимость}, так как оно рассматривает тестируемое \RussianAbbreviation{ПО} как единое целое без учёта его внутренних механизмов. 
%
Наконец, тестирование по методу ''чёрного'' ящика отличается своей \Emphasis{простотой}, так как оно не требует знания деталей реализации, которая может быть достаточно сложной.

\Paragraph{Недостатки}
%
В свою очередь тестирование по методу ''чёрного ящика'' имеет также ряд \Important{недостатков} \Reference{Sutton2007}. 
%
Основным недостатком является \Emphasis{трудность оценки} качества тестирования безопасности \RussianAbbreviation{ПО} в целом. 
%
Другим недостатком является \Emphasis{сложность моделирования комплексной атаки}, нацеленной на некоторую уязвимость.

% ------------------------------------------------------------------------------------------------

\SubSubSectionTitle{Fuzz-тестирование}
	{software_security_analysis_black_box_techniques_fuzz_testing}

\Paragraph{Описание}
%
\EnglishText{Fuzz}-тестирование заключается в передаче случайных некорректных данных тестирумой программе через её точки входа посредством программного окружения или сторонних компонентов и дальнейшей проверке полученных результатов \Reference{Winograd2008}. 
%
Причём обычно генерируемые входные данные основываются на изменении корректного набора входных данных. 
%
Сама передача данных осуществляется с помощью специального приложения, называемого \Emphasis{\EnglishText{fuzzer}'ом}. 
%
Обычно каждый \EnglishText{fuzzer} тесно связан с типом передачи данных, например, посредством протокола \IT{HTTP} \WebSite{HTTP}. 

\Paragraph{Методы fuzz-тестирования}
%
Существует несколько \Important{разновидностей} методов, которые применяются при \EnglishText{fuzz}-тестировании \Reference{Sutton2007}. 
%
При написании тестовых сценариев с преопределёнными данными могут быть использованы \Emphasis{граничные значения} в структурах данных, пакетах или сообщениях, которые являются входными данными для тестируемого приложения. 
%
С другой стороны, могут использоваться \Emphasis{случайно сгенерированные данные}, что является достаточно эффективным, но ресурсоёмким. 
%
Его разновидностью является полный перебор всех входных данных (\EnglishText{brute force}). 
%
Наконец, в некоторых случаях может использоваться целый \Emphasis{фреймворк}, который анализирует выходные данные тестируемого приложения и в зависимости от них определяет дальнейшие значения входных данных.

\Paragraph{Недостатки и ограничения}
%
\EnglishText{Fuzz}-тестирование имеет ряд \Important{ограничений}, связанных с типом уязвимостей, которые можно найти с помощью него \Reference{Sutton2007}. 
%
К ним относятся дефекты, связанные с механизмом контроля доступа, логические или архитектурные ошибки, поиск бэкдоров, повреждение памяти, а также сложные уязвимости, которые выявляются при проведении многошаговых комплексных атак. 
%
Основным \Emphasis{недостатком} является необходимость написания нужного \EnglishText{fuzzer}'a или поиска существующего. 
%
Причём во втором случае необходима его настройка на тестируемое приложение. 
%
Поэтому со стороны специалиста требуется достаточно глубокое знание о тестируемом приложении \Reference{Winograd2008}.

% ------------------------------------------------------------------------------------------------

\SubSubSectionTitle{Внедрение неисправностей в бинарный код}
	{software_security_analysis_black_box_techniques_fault_injection}

\Paragraph{Описание}
%
Данный метод очень наиболее часто применяется в совокупности с тестированием внедрения для получения специалистом более полной картины того, как тестируемое приложение функционирует при проведении атак \Reference{Goertzel2007}. 
%
Его суть заключается в том, чтобы внести \Emphasis{изменения в ресурсы или программное окружение} таким образом, чтобы сымитировать последствия проведения атаки. 
%
\Important{Цель} внедрения неисправностей состоит в том, чтобы оценить состояние, поведение и свойства безопасности тестируемого приложения в данных условиях \Reference{Winograd2008}. 
%
Внедрение неисправностей даёт следующие \Important{преимущества} \Reference{Winograd2008}: 
\begin{itemize}
	\leftskip2em%
	\setlength{\itemsep}{0pt}%
	\setlength{\parsep}{0pt}%

	\item возможность симулирования аномалий окружения без глубокого понимания того, как они могут проявиться впоследствии

	\item возможность выбора того, какие аномалии симулируются в конкретном тесте, без настройки всего окружения в целом

	\item лёгкость автоматизации
\end{itemize}

% ------------------------------------------------------------------------------------------------

\SubSubSectionTitle{Реверс-инжиниринг}
	{software_security_analysis_black_box_techniques_reverse_engineering}

\Paragraph{Что такое реверс-инжиниринг}
%
Данный метод представляет из себя \Emphasis{анализ выполняего кода} приложения и может быть использован для поиска уязвимостей. 
%
В этом случае специалист играет \Emphasis{роль злоумышленника} и проводит анализ тестируемой программы при помощи различного вида специальных инструментальных средств. 
%
Это позволяет оценить \Emphasis{уровень защищённости} программы «извне» и найти дефекты безопасности, видимые с этого уровня \Reference{Goertzel2007}. 
%
Существует большое количество различных инструментов, которые используются для выполнения данной операции \Reference{Eilam2005}. 
%
Каждый из них направлен на выполнение определённых действий, например, отладки, дизассемблирования и других. 
%
Поэтому в зависимости от тестируемого \RussianAbbreviation{ПО} подбирается определённый набор инструментов.

\Paragraph{Дизассемблеры}
%
Одними из наиболее распространённых инструментов, используемых при проведении реверс-инжиниринга, являются \Important{дизассемблеры}. 
%
Они \Emphasis{декодируют бинарный машинный} код в текст, пригодный для чтения человеком. 
%
Наиболее известными дизассемблерами являются \IT{IDA Pro} \WebSite{IDAPro} и \IT{ILDasm} \WebSite{ILDasm}. 
%
\IT{IDA Pro} осуществляет дизассемблирование машинных инструкций \IT{x86}, а \IT{ILDasm} -- байт-кода среды \Term{.NET} \WebSite{dotNet}.

\Paragraph{Отладчики}
%
В отличии от дизассемблеров, \Important{отладчики} могут использоваться для пошагового выполнения тестируемой программы. 
%
Они предоставляют возможность \Emphasis{получения информации о текущем состоянии} программы в любой момент, а также изменения на лету хода её работы. Наиболее известными отладчиками являются \IT{OllyDbg} \WebSite{OllyDbg} и \IT{WinDbg} \WebSite{WinDbg}.

\Paragraph{Декомпиляторы}
%
\Important{Декомпиляторы} представляют собой инструменты, которые \Emphasis{декодируют скомпилированный код} обратно в исходный код. 
%
На настоящий момент такая операция может быть выполнена лишь \Emphasis{частично}, так как в процессе компиляции большой объём информации, содержащийся в исходном коде, не используется и в скомпилированном коде не сохраняется. 
%
Наиболее известным инструментом такого вида является \IT{.NET Reflector} \WebSite{dotNETReflector}.

\Paragraph{Инструменты системого мониторинга}
%
При проведении реверс-инжиниринга одними из наиболее полезных инструментов являются инструменты \Important{системного мониторинга}. 
%
Они позволяют узнать большое количество информации о том, как тестируемое приложение \Emphasis{взаимодействует} с программным окружением. 
%
Например, они могут предоставить информации о том, какие системные функции вызываются приложением, к каким файлам обращается доступ и т. п.. 
%
Наиболее известным набором таких инструментов для \RussianAbbreviation{ОС} \IT{Windows} \WebSite{Windows} являютя утилиты от \IT{SysInternals} \WebSite{SysInternals}, а на других платформах подобную функциональность имеют встроенные утилиты типа \IT{ps}, \IT{lsof} и другие \Reference{Chess2010}.

\Paragraph{Недостатки реверс-инжиниринга}
%
Как и любой метод тестирование \RussianAbbreviation{ПО}, реверс-инжиниринг имеет ряд \Important{недостатков} \Reference{Chess2007}. 
%
Основным недостатком реверс-инжиниринга является \Emphasis{сложность} декодирования и анализа выполняего кода. 
%
Дополнительную трудность представляет собой тот факт, что для успешного проведения реверс-инжиниринга со стороны специалиста необходим \Emphasis{навык работы} с рядом различных инструментов. 
%
Причём, как уже было упомянуто выше, состав набора инструментов может сильно различаться в зависимости от тестируемого приложения.

% ------------------------------------------------------------------------------------------------

\SubSubSectionTitle{Отладка по методу чёрного ящика}
	{software_security_analysis_black_box_techniques_black_box_debugging}

\Paragraph{Описание}
%
Отладка по методу ''чёрного'' ящика включает в себя разделение тестируемой программы на части для последующего тестирования каждой из них по отдельности \Reference{Winograd2008}. 
%
\Important{Целью} такого действия является поиск уязвимых частей. Особенно данный метод полезен тогда, когда приложение использует сторонние компоненты, исходный код которых недоступен.

\Paragraph{В чём она заключается}
%
Отладка по методу ''чёрного'' ящика заключается в \Emphasis{сборе отладочной информации} о том, как приложение взаимодействует со сторонними компонентами и программным окружением. 
%
Это достигается путём мониторинга внешнего поведения программы или его составной части. 
%
Используя данную технику, возможно найти и устранить такие уязвимости, как игнорирование сигналов ошибке и другие \Reference{Whittaker2003}.

\Paragraph{Инструменты}
%
Проведение отладки по методу чёрного ящика подразумевает использование соответствующих \Important{инструментальных средств}. 
%
Получение отладочной информации о тестируемом приложении может быть произведено с помощью путём мониторинга файловой системы и вызываемых приложением системных функций, с помощью использования отладчика приложений и другими способами. 
%
К наиболее известным инструментам, которые используютя для проведения такого рода действий, можно отнести утилиты от \IT{SysInternals} \WebSite{SysInternals}, \IT{IDA Pro} \WebSite{IDAPro} и \IT{OllyDbg} \WebSite{OllyDbg}.

% ------------------------------------------------------------------------------------------------

\SubSubSectionTitle{Сканирование уязвимостей}
	{software_security_analysis_black_box_techniques_vulnerability_scanning}

\Paragraph{Описание}
%
Автоматическое сканирование уязвимостей является одни из наиболее полезных способов анализа безопасности \RussianAbbreviation{ПО} и проводится при помощи специальных инструментальных средств. 
%
Принцип работы таких инструментов состоит в том, что они взаимодействуют с тестируемым приложением, передавая на вход шаблонные данные, называемые сигнатурами \Reference{Winograd2008}. 
%
Каждый сканер уязвимостей содержит большое количество таких сигнатур, каждая из которых связана с определённым видом уязвимости.

\Paragraph{Когда проведение особенно полезно}
%
Использование сканеров уязвимостей \Important{наиболее эффективно} при анализе безопасности сторонних компонентов до момента их использования в составе тестируемого приложения, а также перед проведением тестирования внедрения для снижения количества сценариев, если будут найдены известные уязвимости в приложении \Reference{Winograd2008}. 

\Paragraph{Недостатки}
%
Использование таких сканеров уязвимостей имеет ряд \Important{недостатков} \Reference{Winograd2008}. 
%
В среднем обычный сканер уязвимости в состоянии найти только 30 \% всех типов уязвимостей. 
%
Так как эффективность работы сканеров зависит от количества сигнатур уязвимостей в его базе, то требуется её периодическое обновление.

\Paragraph{Инструменты}
%
Сканеры уязвимостей разделяются на несколько \Important{категорий}, в зависимости от того, как они взаимодействиуют с тестируемым приложением. 
%
Такими категориями являются сканеры портов, сканеры уязвимостей веб-приложений, \EnglishText{CGI}-сканеры и другие. 
%
К наиболее известным инструментам можно отнести \IT{Nmap} \WebSite{Nmap}, \IT{Nessus} \WebSite{Nessus}, \IT{ITS4} \WebSite{ITS4} и другие.

% ------------------------------------------------------------------------------------------------

\SubSubSectionTitle{Тестирование внедрения}
	{software_security_analysis_black_box_techniques_penetration_testing}

\Paragraph{Описание}
%
Тестирование внедрения является одним из наиболее популярных методов при проведении анализа безопасности. 
%
Оно выполняется на последних стадиях разработки \RussianAbbreviation{ПО} и заключается в анализе того, как тестируемое приложение ведёт себя при проведении различного рода атак \Reference{Arkin2005}.

\Paragraph{Особенности тестирования внедрения}
%
Тестирование внедрения имеет ряд \Important{особенностей} \Reference{Winograd2008}. 
%
При его проведении тестируемое приложение выполняется в обычном режиме, то есть настройка специального тестируемого окружения не производится. 
%
Тестирование внедрения в отличие от других видов тестирования в свою очередь также направлено на \Emphasis{поиск архитектурных ошибок} и ошибок, допущенных во время проектирования. 
%
Наконец, в тестах внедрения учитываются \Emphasis{самые худшие сценарии}, которые могут нанести наибольшее повреждение тестируемому приложению.

\Paragraph{Как происходит процесс тестирования}
%
Процесс тестирования внедрения обычно состоит из \Important{следующих шагов} \Reference{Thompson2005}. 
%
Первым шагом является оценка рисков безопасности, таких как отказ в работе, утечка информации, порча данных и другие. 
%
Множество рисков формируют модель рисков. 
%
Вторым шагом является построение плана тестирования: какие компоненты подлежат анализу, как будет проводиться тестирование и т.п.. 
%
Последующими шагами являются, собственно, запуск тестов и составление отчёта о найденных уязвимостях.

\Paragraph{Недостатки}
%
Основным \Important{недостатком} тестирования внедрения является тот факт, что оно проводится на последних стадиях разработки \RussianAbbreviation{ПО}, поэтому найденные с помощью него ошибки безопасности могут потребовать значительных ресурсных затрат на их устранение.