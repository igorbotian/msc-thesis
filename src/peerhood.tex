\IncludeChapterMetaInformation{peerhood}
\SectionTitle{\PeerHoodTitle}{peerhood}
%
\IncludePeerHoodOutline
%
\SubSectionTitle{\PeerHoodConcept}{peerhood_concept}
\IncludePeerHoodConcept
%
\SubSectionTitle{\PeerHoodGoalsAndRequirementsTitle}{peerhood_goals_and_requirements}
\IncludePeerHoodGoalsAndRequirements
%
\SubSectionTitle{\PeerHoodArchitectureTitle}{peerhood_architecture}
\IncludePeerHoodArchitecture
%
\SubSectionTitle{\PeerHoodSecurityProblemTitle}{peerhood_security_problem}
\IncludePeerHoodSecurityProblem
%
\begin{comment}
\Paragraph{Источники}
\begin{description}
	\item[Bishal Raj Karki]
		- определение
		- где и когда началась разработка
		- концепция (вместе с рисунком)
		- архитектура (демон, библиотека, плагины)
		- функциональные характеристики
		- взаимодействие пользовательских приложений с PeerHood
	\item[Jussi Laakkonen]
		- определение
		- для чего был разработан
		- где был разработан
		- связь с PTD
		- концепция (вместе с рисунком)
		- архитектура
		- связь между компонентами
		- цели
		- функциональные характеристики
		- текущая разработка
		- описание плагинов
	\item[Kimmo Kolehmainen]
		- определение
		- для чего предназначен
		- краткое описание функциональных характеристик
		- какие платформы поддерживает
		- где был протестирован
		- концепция
		- ключевые требования
		- высокоуровневая архитектура (демон, библиотека, плагины, приложения)
	\item[Sami Saalasti]
		- зачем был разработан проект (реализация, не концепция)
		- разработка ключевых требований и функциональных характеристик
	\item[Yevgeniy Bondarenko]
		- какую проблему решает PeerHood
		- определение
		- как приложения могут взаимодействовать с PeerHood
		- рассмотрение проблемы безопасности
		- планы по разработке
	\item[Comparison of Linux and Symbian Based Implementations of Mobile Peer-to-Peer Environment]
		- определение (в общих сетевых терминах)
		- цели
		- PeerHood и PTD
		- местоположение PeerHood в стеке пользовательского ПО
		- функциональные характеристики
		- реализации под Linux и Symbian
	\item[Enhancing Mobile Peer-to-Peer Environment with Neighborhood Information]
		- описание мобильной пиринговой среды
		- местоположение PeerHood в концепции PTD
		- компоненты, их взаимодействие
		- обмен информацией между клиентами
		- сценарии использования
		- результат при обмене информации между клиентами
	\item[Peer-to-peer Communication Approach for a Mobile Environment]
		- концепция пирингового взаимодействия в мобильном окружении
		- пример работы
		- описание реализации под Linux (архитектура, демон, библиотека, интерфейс)
	\item[Personal Trusted Device in Personal Communications]
		- примеры использования PTD в жизни
\end{description}

\begin{description}
	\item[Enhancing Mobile Peer-to-Peer Environment with Neighborhood Information]
		- обмен информацией между клиентами
		- результат при обмене информации между клиентами
\end{description}
\end{comment}
%
\begin{comment}
HOW TO ORGANIZE YOUR THESIS

A brief section giving background information may be necessary, especially if your work spans two or
more traditional fields. That means that your readers may not have any experience with some of the 
material needed to follow your thesis, so you need to give it to them. 
A different title than that given above is usually better; e.g., "A Brief Review of Frammis Algebra."
\end{comment}